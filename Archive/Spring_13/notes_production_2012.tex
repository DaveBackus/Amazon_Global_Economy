\chapter{The Production Function}\label{chp:agpf}
% killed "aggregate" to fit title
\hypertarget{production}{}

\textbf{Tools:} Cobb-Douglas production function.

\textbf{Key Words:} Productivity (TFP); constant returns to scale, diminishing marginal product, capital, labor.

\textbf{Big Ideas:}
\vspace{-0.1in}
\begin{itemize}
\item A production function relates output (real GDP) to inputs (capital and labor).
Ours have three essential properties: (i) more inputs lead to more output;
(ii) diminishing returns to capital and labor; (iii) constant returns to scale.

\item The Cobb-Douglas production function is a specific form that we'll use throughout.
\item Total factor productivity (TFP) is the overall efficiency with which inputs are transformed into outputs.
\end{itemize}

\rule{\textwidth}{1pt}

We want to understand why some countries are richer than others,
in the sense of having higher GDP per capita.
Since rich means they produce more output,
the question becomes where the output comes from.
Here, we describe a tool for answering that question:
a {\it production function\/} that relates the quantity of output produced to
(i)~the quantities of inputs and (ii)~the efficiency or productivity
with which they're used.
Doing this for an entire economy takes a leap of faith,
but the reward is a quantitative summary of the sources of
aggregate economic performance.


\section{The production function}
\label{sec:production_function}

Economic organizations transform inputs (factories, office
buildings, machines, labor with a variety of skills, intermediate
inputs, and so on) into outputs.  Boeing, for example,
owns factories, hires workers, buys electricity and avionics, and
uses them to produce aircraft. American Express's credit card
business uses computers, buildings, labor, and small amounts of
plastic to produce payment services. Pfizer hires scientists,
MBAs, and others to develop, produce, and market drugs. McKinsey
takes labor and information technology to produce consulting
services.


For an economy as a whole, we might think of all the labor and capital
used in the economy as producing real GDP,
the total quantity of goods and services.
A production function is a mathematical relation between inputs and output
that makes this idea concrete:
\begin{eqnarray*}
    Y &=& A F(K,L) ,
    \label{eq:pf-general}
\end{eqnarray*}
where $Y$ is output (real GDP), $K$ is the quantity of physical capital
(plant and equipment) used in production,
$L$ is the quantity of labor,
and $A$ is a measure of the productivity of the economy. We call $A$ \emph{total factor productivity}.
More on each of these shortly.


The production function tells us how different amounts of capital and labor
may be combined to produce output.
The critical ingredient here is the function $F$.
Among its properties are
%
\begin{enumerate}
\item \textbf{More input leads to more output.} In economic terms, the
marginal products of capital and labor are positive. In
mathematical terms, output increases in both $K$ and
$L$:
\[
    \frac{\partial Y}{\partial K}> 0, \;\; \frac{\partial Y}{\partial L}> 0.%
\]

\item \textbf{Diminishing marginal products of capital and labor.}
Increases in capital and labor lead to increases in output, but
they do so at a decreasing rate: The more labor we add, the less
additional output we get. You can see this in
Figure~\ref{fig:production}. For a given capital stock $\bar{K}$,
increasing labor by an amount $\Delta$ starting from $L_{1}$ has a larger
effect on output than increasing labor by the same amount starting
from $L_{2} > L_1$. That is:
$AF(\bar{K},L_{1}+\Delta)-AF(\bar{K},L_{1})>AF(\bar{K},L_{2}+\Delta)-AF(\bar{K},L_{2})$.
This condition translates into properties of the second
derivatives:
%
\begin{equation}
\frac{\partial^{2} Y}{\partial K^{2}}< 0,  \;\; \frac{\partial^{2} Y}{\partial L^{2}}< 0.%
\label{eq:dmp}
\end{equation}
%

%
%%%% FIGURE: PRODUCTION FUNCTION
\begin{figure}[h]
\caption{The production function.}\label{fig:production}
%\begin{center}
\centering
\begin{picture}(300,200)(-30,-20)
\footnotesize
\thicklines

% Axis
\put(-30,0){\vector(1,0){300}}%
\put(0,-20){\vector(0,1){200}}%

\put(265,-14){$L$}%
\put(-15,170){$Y$}%

% Drawing the production function
\bezier{2000}(0,0)(1,130)(190,165)%

% Vertical Lines at L_1
\bezier{40}(30,0)(30,45)(30,89)%
\bezier{60}(50,0)(50,53)(50,108)%

% Horizontal Lines at L_1
\bezier{15}(0,89)(15,89)(30,89)%
\bezier{25}(0,108)(25,108)(50,108)%

\put(26,-14){$L_{1}$}%
\put(46,-14){$L_{1}+\Delta$}%

% Vertical Lines at L_2
\bezier{80}(140,0)(140,75)(140,152)%
\bezier{90}(160,0)(160,75)(160,158)%

% Horizontal Lines at L_2
\bezier{65}(0,152)(70,152)(140,152)%
\bezier{85}(0,158)(80,158)(160,158)%

\put(136,-14){$L_{2}$}%
\put(156,-14){$L_{2}+\Delta$}%

\put(200,160){$Y=AF(\bar{K},L)$}
\end{picture}

%\end{center}
\end{figure}


\item \textbf{Constant returns to scale.}
This property says that if we (say) double all the inputs,
the output doubles, too.
More formally, if we multiply both inputs by the same number $ \lambda > 0$,
then we multiply output by the same amount:
\begin{equation}
    A F(\lambda K,\lambda L) \;=\; \lambda A F(K,L) .
    \label{eq:crs}
\end{equation}
Thus, there is no inherent advantage or disadvantage of size.
\end{enumerate}
These properties are more than we need for most purposes,
but we mention them because they play a (sometimes hidden)
role in the applications that follow.

Our favorite example of a production function is $F(K,L) = K^\alpha L^{1-\alpha}$, which leads to
\begin{equation}
    Y \;=\; A K^{\alpha} L^{1-\alpha}
    \label{eq:pf-cd}
\end{equation}
for a number (``parameter'') $\alpha$ between zero and one. Circle
this equation so that you remember it!  It's referred to as the {\it
Cobb-Douglas\/} version of the production function to commemorate
two of the earliest people to use it.  (Charles Cobb was a
mathematician.  Paul Douglas was an economist and later a US
senator.) Let's verify that it satisfies the properties we
suggested. First, the marginal products of capital and labor are
\begin{eqnarray*}
    \partial Y /\partial K &=& \alpha A K^{\alpha-1}L^{1-\alpha} \;=\; \alpha {Y}/{K}\\
    \partial Y /\partial L &=& (1-\alpha) A K^{\alpha}L^{-\alpha} \;=\; (1-\alpha) {Y}/{L} .%
\end{eqnarray*}
Note that both are positive.
Second, the marginal products are both decreasing.
We show this by differentiating the first derivatives to get second derivatives:
\begin{eqnarray*}
    \partial^{2} Y / \partial K^{2} &=&  \alpha (\alpha-1) A K^{\alpha-2} L^{1-\alpha} \\
    \partial^{2} Y / \partial L^{2} &=& - \alpha(1-\alpha) A K^{\alpha}L^{-\alpha-1} .
\end{eqnarray*}
Note that both are negative.
%
Finally, the function exhibits constant returns to scale.
If we multiply both inputs by $\lambda > 0$, the result is
\[
    A (\lambda K)^{\alpha}(\lambda L)^{1-\alpha} \;=\; A\lambda^{\alpha} K^{\alpha} \lambda^{1-\alpha} L^{1-\alpha} \;=\; \lambda AK^{\alpha}L^{1-\alpha} ,%
\]
as needed.
We will typically use $\alpha = 1/3$.
If you'd like to know why, see the ``Review questions'' at the end of the chapter.


\section{Capital input}

The capital input (or capital stock) $K$
is the total quantity of plant and equipment used in production.
We value different kinds of capital (machines, office buildings, computers)
at their base-year prices, just as we do with real GDP in the National Income and Product Accounts.
It's somewhat heroic to combine so many different kinds of capital into one
number, but that's the kind of people we are.

Fine points:
\begin{itemize}
\item \textbf{How does capital change over time?}
Typically, capital increases with investment (purchases of new plant and equipment)
and decreases with depreciation.
Mathematically, we might write
\begin{equation}
    K_{t+1} \;=\; K_{t} - \delta_t K_{t} + I_{t} ,
    \label{eq:lom-k}
\end{equation}
where $\delta_t$ is the rate of depreciation between $t$ and $t+1$.
On average, the capital stock depreciates about 6 percent a year,
but this is an average of depreciation rates for structures
(which depreciate more slowly) and equipment
(for example, computers, which depreciate more quickly).
In practice, we use (\ref{eq:lom-k}) to construct
estimates of the capital stock from investment data.

Digression.
{\footnotesize
 Note that we've used a different timing convention
than financial accountants.
Capital at time $t$ is the amount available for production
during the period.
We use the amount available at the start of the period,
which in financial statements would be the end of the previous
period.
Why do we do this?  Because, otherwise,
current production would depend on last period's
capital stock, which seems a little strange.
Note, too, that for a period like a year, this is a moving target:
The amount of capital available in December is likely to
be different from the amount available in March.
That's not a big deal, because the capital stock is slow to change,
so any changes within a period are likely to be small relative to the total.}

\item \textbf{Quality.}
In principle, we want to take into account changes over time
in the quality of capital.
Computers, for example, are more productive than they were
ten years ago, so a computer today should count as more capital
than a computer ten years did.
Ideally, this happens when we construct our real investment series:
the national income and product accountants consider changes
in quality when they divide investment into price and quantity components.
In recent times, the effect of this has been a sharp decrease
in the price of investment goods,
particularly equipment,
so that a given dollar expenditure
results in greater additions to capital than in the past.


\item \textbf{Wars and natural disasters.}
Wars can have an impact on the capital stock---natural disasters, too,
although their impact is rarely as big.
Experts estimate that the German and Japanese capital stocks declined
by about 50 percent  between the start and end of World War II.
In modern times, the impact is almost always negligible.
September 11 and Hurricane Katrina, for example, had
enormous effects on New York City and New Orleans, respectively,
but the impact on the US capital stock was tiny in both cases.

\item \textbf{Does land count?}  The short answer:  No.
In principle, maybe it should, but in modern economies, land is far less important
than plant and equipment.
For very poor agricultural economies, land and livestock are important
inputs to production, but they're not typically included in our
measures of the capital stock.
%To the extent it matters, its impact shows up in $A$.

\item \textbf{Intangibles.}
Capital here consists solely of physical capital.
We do not include ``investments'' in such things
as research and development, patents, brands, and databases.
These aren't part of traditional measures of capital yet,
but there's been some progress towards including them.

\end{itemize}



\section{Labor input}

The next component of our production function is labor.
The first-order approximation is simply the number of people employed ($L$),
which is a number we can find for most countries.
(It's not as easy as you might think to measure employment,
especially in countries with a large informal sector.)
In some cases, we also include measures of the quality of labor
(``human capital'' $H$) and hours worked ($h$).
If we include both, our measure of labor input becomes
$hHL$.


The starting point for the labor input is, of course, the population.
Populations of countries differ not only in quantity,
but also in their age distribution and its evolution.
Right now, for example, China has a relatively young population,
but with a low birth rate, it is aging rapidly.
The US has a younger population than Europe or Japan,
the result of a higher birth rate (more young people!)
and a higher immigration rate (immigrants tend to be young, too).
These demographic issues are interesting in their own right.
They play an important role in government policy---many countries have state-supported pension and health-care systems,
for example, so changes in the age distribution can have
a significant impact on government budgets.
They're also a critical input in product decisions,
telling you, for example, whether you should be selling
diapers or walkers.


Our focus, however, is on the quantity and quality of labor.
There's no question that individuals differ in skill.
Derek Jeter's skills earn him
\$15m a year as a shortstop for the New York Yankees baseball team,
but most of us would be worth far less in the same job.
American workers earn more than Mexican and Chinese workers,
in part because their skills are better.
There are many skills we might want to measure.
One that's relatively easy to measure is the level of education of the
workforce.
In 2000, the average Korean worker had 10.46 years of
schooling, and the average Mexican worker had 6.73 years. We know
that individuals with more education have higher salaries, on
average, so we might guess that Koreans have higher average skills
than Mexicans. We call this school-based difference in skill
human capital and take it into account by putting
it into our production function:
\[
    Y \;=\; A F(K,HL),
\]
where $H$ is a measure of human capital.


There are two common measures we could use,
both tied to the number of years of school $S$ of the workforce.
The first is to set human capital equal to average years of school:
\[
    H \;=\; S .
\]
This seems to be a relatively good approximation for most purposes,
but it leads to unreasonably large percentage increases in $H$
at low levels of schooling.
For example, workers in India had an average level of schooling
of 1.7 years in 1960,
so one additional year of school increases $H$ by 59 percent.
Another approach, based on a huge body of evidence,
is to credit each year of school with (say) a given percentage
increase in skill.
Mathematically, we might say
\[
    H \;=\; \exp( \sigma S ) ,
\]
where $\sigma$ is the extra value of a year of school.
A good starting point is $\sigma = 0.07$,
which means that every year of school increases
human capital by 7 percent.


A second refinement of our measure of labor input focuses on quantity:
the number of hours worked.
Curiously enough, there are substantial differences
in average hours worked across countries.
If we use $h$ to represent hours worked, our state-of-the-art
modified production function is
\begin{equation}
    Y \;=\; \;=\; A F(K,hHL)
      \;=\;  A K^{\alpha} (hHL)^{1-\alpha}.
      \label{eq:pf-mod}
\end{equation}



\section{Productivity}

The letter $A$ in the production function plays a central role
in this course. We refer to it as total factor productivity or TFP,
but what is it?  Where does it come from?


The word productivity is commonly used to mean several different
things. The most common measure of productivity is the ratio of
output to labor input, which we'll call the {\it average product
of labor\/}. This is typically what government agencies mean when
they report productivity data.  It differs from the {\it marginal
product of labor\/} for the same reason that average cost differs
from marginal cost. {\it Total factor productivity\/},
the letter $A$ in the production function, measures the overall
efficiency of the economy in transforming inputs into outputs.

Mathematically, the three definitions are:
\begin{eqnarray*}
    \mbox{Average Product of Labor} &=& {Y}/{ L} \\
    \mbox{Marginal Product of Labor} &=& {\partial Y}/{\partial L}\\
    \mbox{Total Factor Productivity} &=& {Y}/{ F(K,L)} .
\end{eqnarray*}
%They differ primarily in their treatment of capital.
For the Cobb-Douglas production function they are:
\begin{eqnarray*}
    \mbox{Average Product of Labor} &=& A \left({K}/{L}\right)^\alpha  \\
    \mbox{Marginal Product of Labor} &=& (1-\alpha) A \left({K}/{L}\right)^\alpha \\
    \mbox{Total Factor Productivity} &=& A  .
\end{eqnarray*}
Holding $A$ constant,
the first two increase when we increase the ratio of capital to labor.
Why?
You can be more productive if you have
(say) more equipment to work with.
TFP is an attempt to measure productivity independently
of the amount of capital each worker has.
That allows us to tell whether the US is more productive
than India because it has more and better capital (higher $K$)
or uses the labor and capital it has more effectively (higher TFP $A$).
In practice, both play a role,
and this allows us to tell which effect is larger.


In practice, we measure total factor productivity as a residual:
We measure $A$ by taking a measure of output (real GDP $Y$)
and comparing it to measures of capital and labor inputs.
In the simplest case (without corrections to labor), we solve
\[
    A \;=\; {Y}/{(K^\alpha L^{1-\alpha})}.
\]
As a result, anything that leads the same inputs to produce more
output results in higher TFP.
What kinds of things might do this?
One example is innovation. If we invent the computer chip or a drug
that cures cancer, they will clearly increase measured productivity
(or one would hope they would).
But there are many other examples.
Another example is security. If we establish personal safety and security,
then individuals can spend more time working productively,
and less time worrying about being robbed or murdered.
Another is competition.  If the economic system
reallocates resources from less-productive to more-productive
firms, that will lead to an increase in country-wide productivity.
Capital and labor-market laws and regulations play a clear role here.
In short, anything that affects the allocation of resources can
have an impact on total factor productivity.


\section{Marginal products}

In competitive markets, labor and capital are paid their marginal products.
We could show that, but for now would prefer to simply take it on faith.
That, in turn, tells us where payments to labor and capital come from.

Consider payments to labor.
Firms hire workers until the marginal product of an additional unit of labor
equals its cost, the wage $w$.
We'll go into this in more detail when we study labor markets,
but for now note that this bit of logic can be represented mathematically by
\[
    w \;=\; \mbox{MPL} \;\equiv\; {\partial Y}/{\partial L} ,
\]
where MPL means the marginal product of labor.
With our basic Cobb-Douglas production function (\ref{eq:pf-cd}), this becomes
\[
    w \;=\; (1-\alpha) A K^\alpha L^{-\alpha} \;=\; (1-\alpha) A \left({K}/{L}\right)^{\alpha}
    \;=\; \ (1-\alpha) {Y}/{L}.
\]
We can now ask ourselves:  What do we need to generate high wage rates?
The answer:  High total factor productivity and/or high capital-labor ratios.
In words, workers are more productive, at the margin, if TFP is high and
if they have more capital to work with.


Note that high wages are a good thing for an economy:
they reflect (for example) high productivity.
Often, countries with high TFP also have high capital per worker, so the two
terms drive wages in the same direction.
It doesn't seem fair, but it happens because the same productivity that makes workers
valuable also raises the return on capital, as we see next.


The market return on capital ($r$, say) equals the marginal product of capital.
In this case, there's an additional adjustment for depreciation, so we have
\[
    r \;=\; \mbox{MPK} \;=\; \alpha A \left({K}/{L}\right)^{\alpha-1} - \delta \;=\; \ \alpha {Y}/{K} - \delta.
\]
The right-hand side here is the net marginal product of capital---net because
we have netted out depreciation.
Without that term, we have the gross marginal product of capital,
because our measure of output is gross of depreciation (the G in GDP).

In short, the productive value of labor and capital
(ie, their marginal products)
depends in large part on total factor productivity.
To understand this, it's important that you be able to distinguish between
total factor productivity (the letter $A$ in the production function)
and the marginal products of labor and capital.


\section*{Executive summary}

\setlength{\leftmargini}{.5\oldleftmargini}
\begin{enumerate}
\item A production function links output to inputs.

\item Inputs include physical capital (plant and equipment)
and labor (possibly adjusted for skill and hours worked).

%\item Labor varies in quantity (number of people working, numbers of hours)
%and quality (skill, education).

\item Total Factor Productivity (TFP) is a measure of overall productive efficiency.
\end{enumerate}
\setlength{\leftmargini}{\oldleftmargini}


\section*{Review questions}

\setlength{\leftmargini}{.5\oldleftmargini}
\begin{enumerate}

\item Components of the production function.  A small country invests a large fraction of GDP in a major
infrastructure project, which later turns into a ``white elephant''
(that is, it's not used).
How does this affect the components of the
production function?

Answer.
The investment will raise the stock of capital $K$,
but since it's not used, we would expect no increase in output $Y$.
We would, therefore, expect measured productivity to fall.

\item Computing TFP.  Suppose an economy has the production function
\[
    Y \;=\; A K^{1/4} L^{3/4} .
\]
If $ Y= 10$, $K=15$, and $L = 5$,
what is total factor productivity $A$?

Answer.  $ A = Y / (K^{1/4} L^{3/4}) = 1.520$.


\item Diminishing returns.  Suppose the production function is
\[
    Y \;=\; 2 K^{1/4} L^{3/4}
\]
and $K=L = 1$. How much output is produced? If we reduced $L$ by
10 percent, how much would $K$ need to be increased to produce the same
output?

Answer.  With $K=L=1$, $Y = 2$.
If $L$ falls to $0.9$, $K = 1/0.9^3 = 1.372$
(a 37 percent increase in $K$).
The reason for the difference between the magnitudes in the changes in
$K$ and $L$ is the difference in their exponents in the production function.


\item Human capital 1. Worker 1 has ten years of education, worker 2 has 15.  How much
more would you expect worker 2 to earn?  Why?

Answer.  If $H = $ years of education, then one hour of worker 2's
time is equivalent to 1.5 ($=15/10$) hours of worker 1's time, so
we'd expect her to be paid 50 percent more. A more complex answer is
that skill may increase in a more complicated way with years of
education, and that types of education may differ in their impact
on earning power (an MBA may be worth more in this sense than a
PhD in cultural anthropology, however interesting the latter may
be).

\item Human capital 2. Consider the augmented production function
\[
    Y \;=\;  K^{1/3} (HL)^{2/3} .
\]
If $K=10$, $H=10$, and $L=5$, what is the average product of labor?
How much does the average product increase if $H$ rises to 12?

Answer.  Output is $Y = 29.24$ so $Y/L = 5.85$.  If $H$ rises to 12,
$Y/L = 6.60 $.

\item Production function conditions.
Conditions 2 and 3 [equations (\ref{eq:dmp}) and (\ref{eq:crs})]
seem to contradict each other.
One says increases in inputs have a declining impact on output, while
the other says that proportional increases in capital and labor
lead to the same proportional increase in output.  What's going on here?

Answer.  This is a subtle issue, but the answer is that
the conditions are different.
Condition 2 concerns increases in one input,
{\it holding constant the other input\/}.
Condition 3 concerns increases in both inputs at the same time.
Different concepts, different properties.

\item One-third.  Why does $\alpha=1/3$?

Answer.  If we look at the income side of the National Income
and Product Accounts, about two-thirds is paid to labor
and one-third to capital.
We'll see later that firms will hire labor until its
marginal product equals the wage.
For our Cobb-Douglas production function,
\[
    w \;=\; \mbox{MPL} \;=\; (1-\alpha) A K^\alpha L^{-\alpha} .
\]
Total payments to labor are the product of the wage and labor:
\[
    w L  \;=\; (1-\alpha) A K^\alpha L^{1-\alpha} \;=\; (1-\alpha) Y.
\]
So we set $1-\alpha = 2/3$, as stated.
\end{enumerate}
\setlength{\leftmargini}{\oldleftmargini}


\section*{Symbols used in this chapter}

\begin{table}[H]
\centering
\caption{Symbol table.}
\begin{tabular*}{0.98\textwidth}{l@{\extracolsep{\fill}}l}
\toprule
Symbol & Definition\\
\midrule
$Y$                            &Output (real GDP)\\
$A$                            &Total factor productivity (TFP)\\
$K$                            &Stock of physical capital (plant and equipment)\\
$L$                            &Quantity of labor (number of people employed)\\
$F(K,L)$                    &Production function of K and L\\
$\partial F(K,L)/\partial K $             &    Partial derivative of $F(K,L)$ with respect to $K$\\
$\partial F(K,L)/\partial L $             &    Partial derivative of $F(K,L)$ with respect to $L$\\
$\Delta$                     &Infinitesimal number\\
$\overline{K}$                &Given capital stock\\
$\lambda$                     &Constant\\
$\alpha$                     &Exponent of $K$ in Cobb-Douglas production function \\
                            &(= capital share of income)\\
$\delta$                     &Rate of depreciation of physical capital\\
$I$                            &Investment (purchases of new plant and equipment)\\
$H$                            &Human capital\\
$h$                            &Hours worked\\
$hHL$                        &Volume of labor input\\
$S$                            &Years of school of workforce\\
$\sigma$                     &Extra value of a year    of school\\
$w$                            &Wage\\
$r$                            &Market return on capital (or rental cost of capital)\\
\bottomrule
\end{tabular*}
\end{table}

\section*{If you're looking for more}

The methodology described in this chapter has been applied,
with lots of variations,
to countries, industries, and even firms.
The biggest challenge in most studies is coming up with capital stocks.
We use the Penn World Table and construct country-level capital stocks ourselves,
which we can then use to compute productivity.
If you're interested, and can't find it online, send us an email.

Lots of other organizations do their own calculations.
Two of the most useful public sources are
\begin{itemize}
\item The Bureau of Labor Statistics.
They  report what they call
``multifactor productivity,'' both levels and growth rates,
for the US private business sector and a number of industries:

\vspace*{\parskip}
\centerline{\url{http://www.bls.gov/bls/productivity.htm}.}

\item The Conference Board.  Their Total Economy Database includes
growth rates of aggregate TFP for a number of countries:

\vspace*{\parskip}
\centerline{\phantom{xxxxx}\url{http://www.conference-board.org/data/economydatabase/}.}
\end{itemize}

\begin{comment}
MBA 11 alum Matthew Cedergren
supplies these
links to interactive graphs of Cobb-Douglas production functions:
%
\begin{itemize}
\item  Manfred Gartner's
\href{http://www.fgn.unisg.ch/eurmacro/tutor/cobb-douglas.html}
{eurmacro site}.

\item
\href{http://demonstrations.wolfram.com/CobbDouglasProductionFunctions/}
{Wolfram demonstrations}.

\end{itemize}
\end{comment}
