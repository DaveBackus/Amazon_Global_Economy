\part{Long-Term Economic Performance}

\chapter*{Long-Term Overview}
\hypertarget{lrp}{}

\rule{\textwidth}{1pt}

This outline covers key concepts from the first part of the course:
long-term economic performance.
It is not exhaustive, but is meant to help you
(i)~anticipate what is coming and
(ii)~organize your thoughts later on.

\medskip
\hyperref[chp:agpf]{\textbf{\underline{The Aggregate Production Function}}}

\textbf{Tools:} Cobb-Douglas production function.

\textbf{Key Words:} Productivity (TFP); constant returns to scale, diminishing marginal product, capital, labor.

\textbf{Big Ideas:}
\vspace{-0.1in}
\begin{itemize}
\item A production function relates output (real GDP) to inputs (capital and labor).
Ours have three essential properties: (i) more inputs lead to more output;
(ii) diminishing returns to capital and labor; (iii) constant returns to scale.

\item The Cobb-Douglas production function is a specific form that we'll use throughout.
\item Total factor productivity (TFP) is the overall efficiency with which inputs are transformed into outputs.
\end{itemize}


\hyperref[chp:solo]{\textbf{\underline{The Solow Model}}}

\textbf{Tools:} Capital accumulation dynamics; Cobb-Douglas production function.

\textbf{Key Words:} Investment; saving; depreciation; steady state; convergence.

\textbf{Big Ideas:}
\vspace{-0.1in}
\begin{itemize}
\item The Solow model connects saving and investment with economic growth.
\item In the Solow model without productivity (TFP) growth,
capital accumulation does not generate long-run growth.
The reason is diminishing returns to capital:  
the impact of additional capital declines the more you have.
    As a result, differences in saving rates have only modest effects on output per worker 
    and none at all on its long-run growth rate.  
\item TFP growth generates long-run growth in output per worker.
\end{itemize}


\textbf{\hyperref[chp:grth]{\underline{Sources of Economic Growth}} and \hyperref[chp:insp]{\underline{Institutions and Policies}}}

\textbf{Tools:} Cobb-Douglas production function; level and growth accounting;
continuously-compounded growth rates.

\textbf{Key Words:} Productivity (TFP); institutions; time consistency; governance; property rights. 

\textbf{Big Ideas:}
\vspace{-0.1in}
\begin{itemize}
\item Level and growth accounting allow us to \emph{quantify} the sources of growth:
  the contributions of capital, labor, and total factor productivity (TFP) to growth in real GDP.
\item TFP accounts for most of the cross-country differences in output per worker 
and in differences in the growth rate of output per worker.
\item Cross-country differences in productivity (TFP)
  are connected to differences in institutions that shape productivity and policy.  
\end{itemize}


\hyperref[chp:lbmk]{\textbf{\underline{Labor Markets}}}

\textbf{Tools: }Labor supply and labor demand diagrams; simple model of unemployment dynamics.

\textbf{Key Words:} Labor force; unemployment rate; vacancy rate; accession rate; separation rate.

\textbf{Big Ideas:}
\vspace{-0.1in}
\begin{itemize}
\item Unemployment and vacancy rates tell us about excess supply and demand in labor markets. Unemployment arises from the time it takes to match a worker and a firm.

\item Institutions influence accession and separation rates, which influence unemployment. High accession rates lead to lower unemployment and faster transitions; accession rates vary across countries and over the business cycle and currently are very low in the US.
\end{itemize}

\hyperref[chp:fnmk]{\textbf{\underline{Financial Markets}}}

\textbf{Key Words:} Time consistency; information asymmetry.

\textbf{Big Ideas:}
\vspace{-0.1in}
\begin{itemize}
\item Effective financial markets require strong institutional support.
\item Good institutions deal with information asymmetries and time consistency issues.
\end{itemize}

\hyperref[chp:intr]{\textbf{\underline{International Trade}}}

\textbf{Tools:} Ricardo's model of trade; consumption and production possibility frontiers.

\textbf{Key Words:} Absolute advantage; comparative advantage; autarky.

\textbf{Big Ideas:}
\vspace{-0.1in}
\begin{itemize}
\item Trade is a positive-sum game:  both countries benefit.
\item Gains from trade are similar to increases in TFP:  trade increases aggregate consumption opportunities.
\item Trade creates winners and losers, but the winners win more than the losers lose.
 Trade affects the kind of jobs that are available, not the number of jobs.
%\item Opposition to trade is often thinly disguised self-interest.
\end{itemize}



