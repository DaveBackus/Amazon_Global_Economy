\chapter{Exchange-Rate Regimes}\label{chp:fxr}
\hypertarget{fxregimes}{}

\textbf{Tools:} Central bank balance sheet.

\textbf{Key Words:} Convertibility; capital mobility; capital controls; fixed and flexible exchange rates;
foreign exchange reserves; sterilization.

\textbf{Big Ideas:}
\vspace{-0.1in}
\begin{itemize}
\item Countries adopt different exchange rate regimes: fixed, floating, and in between.
\item The trilemma limits our policy options:  we can choose only two of
(i)~fixed exchange rate, (ii)~free flow of capital,
and (iii)~discretionary monetary policy.
\item Fixed exchange-rate regimes must be defended through open market operations 
and are vulnerable to speculative attack.
\end{itemize}
\rule{\textwidth}{1pt}

The term ``exchange-rate regimes'' refers to the various
arrangements that governments around the world make about
international transactions.
We'll see (i)~how central banks intervene
in currency markets to fix the price
and (ii)~how such fixed exchange-rate systems sometimes blow up.


\section{A catalog of foreign-exchange arrangements}

Governments follow a wide range of policies toward their currencies.
One aspect of policy is whether people and businesses can freely
exchange their local currency for another:
whether the currency is {\it convertible\/}.
The US dollar, for example, is convertible.
You can walk into most banks in New York and use dollars to buy dozens of foreign currencies.
Or you can use your credit card abroad and have the currency transaction
done for you.
The renminbi, however, has limited convertibility.
You need approval from the Chinese central bank to buy or sell Chinese currency.

A related issue is {\it capital controls\/}:
whether the government restricts movements of capital (funds)
in and out of the country.
In the US, capital is generally free to move in and out of the country,
although there are restrictions on foreign ownership of companies
in some industries (banks, media, airlines).
In China, there are limits on foreign investments that vary (as in the US)
by industry and type (direct investment is easier than buying securities).
And there are restrictions that limit the amount of money that Chinese citizens can take out of the country.
These controls are typically enforced through convertibility:
since you can't convert renminbi to (say) dollars, you can't
take it out of the country.


There's nothing unusual about this.
Many countries limit convertibility and capital flows,
particularly during times of stress.
Malaysia imposed capital controls during the Asian crisis of 1997,
and Argentina did the same in 2002.

Another aspect of foreign-exchange policy
is whether the price of the currency is set by the government,
allowed to float freely, or something in between.
If the price is determined in a free market, we say that we have a
{\it flexible\/} or {\it floating\/} exchange-rate regime.
If the government sets the price, we say it has a
{\it fixed\/} or {\it pegged\/} exchange-rate regime.
A {\it managed float\/} is somewhere in between.


\section{Fixed exchange rates}

Many countries have fixed exchange-rate regimes of one sort or other.
Colombia uses US dollars, so its currency is fixed by design. %We say that Colombia has ``dollarized.''
The countries of the European Monetary Union use a common currency, the euro.
Other countries have their own currencies, but intervene to
fix the price.
Probably the most prominent current example is the Chinese renminbi,
which has been quasi-fixed for more than a decade.


How does a central bank set the exchange rate if the currency
 is convertible?
Can it simply announce a rate?
Probably not.
You can state a price, but you can't make people trade at it.
You could claim, for example, that your apartment is worth
\$10m, but if no one is willing to buy it for that price,
the statement is meaningless.
For the same reason, a central bank must back up its
claim to fix the exchange rate by buying and selling as much
foreign currency as people want at the stated price.


Let's think through how this might work.
Suppose the New York City government decided to fix the price of beer at \$2 a sixpack (cheap even if you live outside NYC).
It supports this price by buying or selling
any amount at the quoted price.  Can they keep the price this low?
Our guess is that at this price,
beermakers would not find it profitable to make any
(at least not any that we'd be willing to call beer).  People would then flood the government with requests for beer, which the government would
not be able to meet.  When the government reneged on its promise to buy or sell at \$2, the price
would rise above \$2 to its market level, either officially or on the black market.
Unless the government has enough beer to back up the price,
the system will collapse.
Alternatively, suppose that the government set the price at \$20.
Beermakers would flood the government with beer at this price, leaving the
government with a huge surplus.  This is roughly what Europeans do with agriculture, where artificially high prices have left the EU with
``mountains of butter," ``lakes of wine," and so on.
The point is that the government can fix a price
only if it is willing and able to buy and sell at that price --- or outlaws market transactions altogether.

The same logic applies to currencies.
If the People's Bank of China were to support an excessively high price for
the renminbi, then it would be flooded with offers from traders
selling renminbi for (say) dollars.
Its balance sheet would look something like this:
%
\begin{center}
\begin{tabular}{lr|lr}
               Assets  &     &     Liabilities                     \\
               \hline
               FX Reserves &  20 &     Money &  200   \\
               Bonds   & 180 & \\
\end{tabular}
\end{center}
%
We made these numbers up, but they give us the right idea.
The central bank has the usual liabilities,
``money" and government bonds,
and also holds some foreign currency reserves,
which you might think of as dollars.
The PBOC intervenes in the currency market by trading renminbi for dollars,
and vice versa, depending on market conditions.


Suppose, for example, that Nike wanted to convert
\$2m to renminbi to build a new plant in China.
It would do this through a Chinese bank.
If the bank had no countervailing trades, it would
go to the PBOC and exchange the \$2m for renminbi at the going rate --- say ten yuan per dollar, to make the arithmetic simple.
The PBOC's balance sheet would then show an increase of
20m yuan worth of foreign currency
and a comparable increase in its monetary base:
%
\begin{center}
\begin{tabular}{lr|lr}
               Assets  &     &     Liabilities                     \\
               \hline
               FX Reserves &  40 &     Money &  220   \\
               Bonds  & 180 &
\end{tabular}
\end{center}
%
Note that the transaction doesn't make the PBOC any richer.
%It's simply a portfolio shift.
Its net worth is unchanged,
since it has exchanged assets with equal value.

The difference, then, between fixed and flexible exchange-rate regimes
is that the former obligates the central bank to buy and sell
currencies at the stated price.

\section{Sterilization}

You might have noticed that when a central bank buys and sells foreign currency,
its money supply changes.
In the example above, the purchase of 20 worth of foreign currency increased the money supply by the same amount.
It's automatic:  when the central bank purchases foreign currency, it offers domestic
currency in return.

Central banks often want to reverse this impact of foreign exchange intervention
by engaging in an offsetting open market operation.
We refer to this as {\it sterilization\/}.

In our example, the central bank would like to reduce the money supply by 20,
offsetting the impact of buying foreign currency.
It does so with an open market sale of government bonds.
The sale of bonds is paid for in local currency, which is now held
by the central bank.
Its balance sheet is now
%
\begin{center}
\begin{tabular}{lr|lr}
               Assets  &     &     Liabilities                     \\
               \hline
               FX Reserves &  40 &     Money &  200   \\
               Bonds  & 160 &
\end{tabular}
\end{center}
%
In China's case, this has happened to such an extent that the bond
position is negative:  the PBOC is an issuer of bonds, not an investor.


\section{The trilemma}

Exchange-rate policy is, evidently, a dimension of monetary policy
since it involves management of the central bank's balance sheet.
Is it another tool a central bank can use to manage the economy?

Both logic and experience tell us that the central bank's choices are limited.
The sharpest example is the {\it trilemma\/}.
You can choose, at most, two of the following:
% ?? honest, smart, get elected
\begin{itemize}
\item fixed exchange rate
\item free international flow of capital
\item discretionary monetary policy
\end{itemize}
If you try for all three, something will give, probably the exchange rate.

The US lets the exchange rate float, which allows it to have a discretionary
monetary policy and free movements of capital.
China limits the international flow of capital,
which allows it to have a fixed exchange rate and some degree of monetary policy discretion.
The UK, in 1992, tried for all three, and it blew up, driving them out
of the European Monetary System, the precursor of the European Monetary Union.
%Mexico let its real exchange rate appreciate in 1994,
%only to see it depreciate sharply at the end of the year.


\section{Exchange-rate crises}

As a matter of experience, fixed exchange-rate systems often collapse ---
sometimes spectacularly --- when the central bank runs out of reserves.

We can illustrate the mechanics with the central bank's balance sheet.
Suppose that it looks like the one above, with ``fx reserves'' of 40.
And suppose, further, that investors
would like to exchange 50 worth of pesos for the same value in dollars.
Once the central bank runs out of dollars, it can no longer support
the exchange rate, which becomes (more or less automatically) floating.

It's the same issue we illustrated earlier with beer: If people would prefer to buy foreign currency at the official
exchange rate,
and the currency is convertible,
the central bank may find that its supply of foreign reserves is not
enough to meet the market demand.
(The market for currencies is enormous, so you need a lot of reserves.)
For that reason, currency traders often look closely at the central bank's
foreign currency reserves to measure its ability to maintain a fixed rate.

What invites ``speculative attacks" on a currency with a fixed exchange
rate? Often, it's a problem of \hyperref[sec:time_cons]{time consistency}. A fixed exchange rate is a
policy promise to exchange one currency for another at a specified price
without limit into the future. If investors today expect that a future
policymaker will alter that price, what will stop them from selling the
``expensive" currency today? In a foreign-exchange market that transacts
about four trillion dollars daily, few governments have adequate foreign
reserves to fend off a run on a fixed exchange rate.

One classic currency run occurred in 1992 in the United Kingdom.
As part of the European Exchange Rate Mechanism (ERM), the UK had effectively
fixed its currency, sterling, to Germany's Deutsche Mark (DM).
But Germany was in its post-unification economic boom and needed
high interest rates to limit inflation, while the UK was in a deep recession
and needed low interest rates. Doubting that UK policymakers
would keep interest rates high just to maintain the fixed exchange rate,
speculators sold sterling. They made a fortune when the UK exited
the ERM in September 1992 and sterling plunged versus the DM.

You might ask: Should the UK have considered capital controls instead of
devaluing sterling? One practical obstacle was that any hint of controls
would have further encouraged investors to flee sterling
before they could no longer do so. Where time consistency is lacking --- in this case, in currency policy --- instability often follows.

There's a big-picture question lurking behind the scenes here:
whether fixed exchange rate regimes reduce volatility.
With flexible rates, we tend to see a lot of short-run volatility.
With fixed exchange rates, short-run volatility is low most of the time,
but we occasionally have spikes in volatility when the system collapses.
Neither seems completely appealing, but that's the choice we're given.


\section{Strong fixes}

The tendency for fixed exchange rates to blow up has led to two
competing lines of thought.
One is to let them float --- let off the pressure, so to speak.
The other is to reinforce the fixed-exchange-rate system and nail the lid down tighter.
Nothing has proved foolproof to date, but you never know.

One way to reinforce a fixed exchange rate is with a currency board.
The idea is to start off with a large reserve of foreign currency
and limit issues of domestic currency to this amount.
That way, you should not run out of foreign currency when people trade
in their local currency.
Argentina set up a system like this in the 1990s, and established
an exchange rate of one dollar per peso.
But it was dissolved in a currency crisis ten years later.
Hong Kong has had such a system since 1983, with the Hong Kong dollar
pegged to the US dollar.
As a result, interest rates in Hong Kong mirror those in the US:
it has, in a sense, inherited US monetary policy.


A more extreme arrangement is a common currency.
EMU (the euro area) is the most ambitious effort along these lines to date.
But it has been under stress for years, and it remains unclear whether
it will survive in its current form.


\section*{Executive summary}

\setlength{\leftmargini}{.5\oldleftmargini}
\begin{enumerate}
\item ``Convertibility'' and ``capital mobility'' refer to
policies limiting currency transactions and international capital flows.

\item Foreign currency reserves are an indicator of the
government's ability to maintain a fixed exchange rate.

\item The trilemma says you can have, at most, two of the following three
things:
(i)~fixed exchange rates;
(ii)~international capital mobility; and
(iii)~discretionary monetary policy.
\end{enumerate}
\setlength{\leftmargini}{\oldleftmargini}

%\begin{comment}
\section*{Review questions}

\setlength{\leftmargini}{.5\oldleftmargini}
\begin{enumerate}
\item Foreign exchange market intervention.
Use a hypothetical central bank balance sheet to show how
purchases of foreign currency affect the bank's assets and liabilities. What does this purchase do to the supply of money (currency)?

Answer.  When a central bank buys foreign currency,
it receives it from private owners and gives them
    domestic currency in return.
    The latter is an increase in the domestic money supply.
    Suppose, for example, that the central bank
    starts with the balance sheet
    %
\begin{center}
\begin{tabular}{lr|lr}
               Assets  &     &     Liabilities                     \\
               \hline
               FX Reserves &  100 &     Money  &  200   \\
               Bonds   & 100 & \\
\end{tabular}
\end{center}
%
The purchase of 25 worth of foreign currency
changes the balance sheet to
%
\begin{center}
\begin{tabular}{lr|lr}
               Assets  &     &     Liabilities                     \\
               \hline
               FX Reserves &  125 &     Money  &  225   \\
               Bonds   & 100 & \\
\end{tabular}
\end{center}
%

\item Sterilization.  Suppose that the central bank has increased the money supply by purchasing foreign currency, as described above.
    How might it offset this impact on the money supply (sterilize it, so to speak)?

Answer.  It does an equal sale of bonds, accepting money in return.
If it sells 25 worth of bonds, the balance sheet changes to
\begin{center}
\begin{tabular}{lr|lr}
               Assets  &     &     Liabilities                     \\
               \hline
               FX Reserves &  125 &     Money  &  200   \\
               Bonds   & 75 & \\
\end{tabular}
\end{center}
%
The net result of the two trades is that its liabilities are now more heavily weighted in foreign currency.
If the foreign currency rises, it makes money; if it falls, it loses money.
This posture is designed as protection against a sharp fall in local currency (or rise in foreign currency), and it does that.

\item Hong Kong's trilemma.
Use the trilemma to explain why Hong Kong has inherited US monetary policy.

Answer.  Hong Kong has (i) a fixed exchange rate against the US dollar and
(ii)~international capital mobility.
The trilemma then tells us that it can't have its own monetary policy.
Should they want their own monetary policy, either (i) or (ii) has to go.
\end{enumerate}
\setlength{\leftmargini}{\oldleftmargini}


\section*{If you're looking for more}

The International Monetary Fund's
{\it Annual Report on Exchange Arrangements\/}
is the definitive guide to exchange-rate arrangements:  fixed,
flexible, capital controls, and so on.

