\chapter{Aggregate Supply and Demand}\label{chp:asad}
\hypertarget{asad}{}

\textbf{Tools:} Aggregate supply and demand (AS/AD) graph.

\textbf{Key Words:} Short- and long-run aggregate supply; sticky wages;
supply and demand shocks; Keynesian.

\textbf{Big Ideas:}
\vspace{-0.1in}
\begin{itemize}
\item The AS/AD model relates output and prices in the short and long runs.
The model is composed of (i) an upward-sloping short-run aggregate supply curve,
which inherits its shape from sticky wages;
(ii) a vertical long-run aggregate supply curve;
and (iii) a downward-sloping aggregate demand curve.
\item Shocks to the aggregate supply and demand curves
have different effects on inflation and output.
\end{itemize}
\rule{\textwidth}{1pt}

We've seen that economic fluctuations follow regular patterns
and that these patterns can be used to forecast the future.
For some purposes, that's enough.
But if we want to make sense of analysts' discussions of business conditions,
of monetary policy,
or of situations that don't fit historical patterns,
we need a theoretical framework.
The aggregate supply and demand model is the analyst's standard,
the implicit model behind most popular macroeconomic analysis.
It's not the answer to all questions,
but it's a useful tool for organizing our thoughts.
Think of these notes as the executive summary;
textbooks devote hundreds of pages to the same topic.
If you'd like to go deeper, see the references at the end of this chapter.

We take one common shortcut.
In our theory, the variables are real output and the price level.
In practice, we refer to output growth and inflation.
There's a modest disconnect here between theory and practice,
but it's something that can be worked out.
You can thank us later for saving you the trouble of
doing it explicitly.


\section{Aggregate supply}

The standard business-cycle model used by analysts
is an adaptation of the supply and demand diagram.
We refer to the curves as aggregate supply and aggregate
demand to emphasize that we're talking about the whole economy
rather than a single market --- and to remind ourselves that the analogy with supply and demand
in a single market is imperfect.
Figure \ref{fig:asad} is an example.
The aggregate supply curve (AS) represents combinations of
output (real GDP, which we denote by $Y$) and the price level
(an index $P$ such as the GDP deflator)
consistent with the decisions of producers and sellers (``supply'').
The aggregate demand curve (AD)
represents combinations of output and the price
level consistent with the decisions of buyers (``demand'').

Here's how aggregate supply works.
%How do workers and firms decide how much to work and produce?
We start with the production function,
\begin{eqnarray}
    Y_t &=& A_t K_t^\alpha L_t^{1-\alpha} ,
    \label{eq:pf}
\end{eqnarray}
where (as usual) $Y$ is output, $A$ is total factor productivity,
$K$ is the stock of physical capital (plant and equipment),
$L$ is labor, and $ \alpha = 1/3$.
%Everything but $\alpha$ can in principle change with time $t$.

In the simplest ``classical'' theory, $Y$ is fixed
because nothing on the right-hand side of (\ref{eq:pf}) changes in the short run.
Although current productivity $A$ and capital $K$ can change over time,
we often assume that their current values are set --- that is
they can't be changed by current decisions.
Capital, for example, takes time to produce:
We can increase next year's capital stock by investing today,
but today's capital stock is whatever we happen to have.
Labor $L$ is determined by supply and demand in the labor market,
which gives us a level of work that reconciles firms'
demand with individuals' supply.
The result is a vertical aggregate supply curve,
such as AS$^*$ in the figure,
in which output does not depend on the price level $P$.


%%%%%%%%%%%%%%%%%%%%%%%%%%%%%%%%%%%%%%%%%%%%%%%%%%%%%%%%%%%%%%%%%%%%%%%%%%%%
%  Supply and demand diagram
\begin{figure}[h]
\caption{Aggregate supply and demand.}
\label{fig:asad}
%
\centering
\setlength{\unitlength}{0.075em}
\begin{picture}(250,200)(0,0)
%\footnotesize
\thicklines

% horizontal axis
\put(-30,0){\vector(1,0){300}}
\put(255,-16){$Y$}
\put(142,-16){$Y^*$}

% vertical axis
\put(0,-20){\vector(0,1){200}}
\put(-15,155){$P$}

% demand
\put(25,165){\line(4,-3){200}}\put(230,10){AD}

% supply
\put(25,13){\line(4,3){200}} \put(230,160){AS}
\put(146.4,0){\line(0,1){170}} \put(138,175){AS$^*$}

% equilibrium labels
\put(105,85){\footnotesize A}
\put(135,60){\footnotesize B}
%\put(138,64){\footnotesize C}
% dotted lines
%\qbezier[31]{(133,0)(133,46)(133,92)}
%\qbezier[45]{(0,92)(67,92)(133,92)}
%\qbezier[45]{(0,72)(67,72)(133,72)}
\end{picture}
\begin{minipage}{0.6\textwidth}
\vspace{0.45in}
\footnotesize{The horizontal axis is real GDP;
the vertical axis is the price level.
The lines represent aggregate supply (two versions, AS and AS$^*$)
and demand (AD).}
\end{minipage}

\end{figure}

%%%%%%%%%%%%%%%%%%%%%%%%%%%%%%%%%%%%%%%%%%%%%%%%%%%%%%%%%%%%%%%%%%%%%%%%%%%%

A vertical aggregate supply curve was the state of the art until the 1930s,
when John Maynard Keynes (pronounced ``canes'')
decided that the Depression demanded a new theory.
He and his followers (the ``Keynesians'')
argued that the aggregate supply curve should be upward-sloping.
Why?  The most popular argument is that wages or prices are ``sticky'':
They do not adjust quickly enough
to equate supply and demand for labor or goods. One version of this story is that
wages are sticky downward (that is, wages generally don't fall when
demand for labor declines).
The sticky wage analysis goes something like this:
If the nominal wage $W$ is fixed (i.e., very sticky),
then an increase in the price level reduces the real wage $W/P$,
making labor more attractive to firms, who hire more workers,
which raises output.
(That is, we continue to use the labor demand curve.)

The Keynesian sticky wage/price story leads to
an aggregate supply curve that is upward-sloping,
since an increase in the price level leads to a lower
real wage and, therefore, more people hired by firms.
We refer to it as the short-run aggregate supply curve,
because wages and prices are not thought to be sticky forever.
Eventually they adjust, putting us on the vertical aggregate
supply curve AS$^*$,
which we refer to as long-run aggregate supply.


That's what aggregate supply looks like,
but what kinds of things make it shift over time?
Here's a list:
%
\begin{itemize}
\item Productivity:  An increase in $A$ shifts AS to the right.
You can see why from the production function (\ref{eq:pf}) --- an increase in productivity $A$ raises output $Y$.


\item Capital:  ditto an increase in $K$.
\item Price of imported oil:  This is a more subtle one, but
an increase in the price of oil works like a negative productivity shock
in an oil-importing country like the US and shifts AS to the left.
Why?  Because production involves energy, as well as capital and labor, as an input.
In our measurement system, GDP consists only of value added by capital
and labor.
If the price of oil rises, then a larger fraction of total production
is paid to oil producers, leaving less for capital and labor and, therefore, reducing GDP.
There's very little question that this is what happens in practice: An increase in the price of oil leads to larger payments to oil producers
and smaller payments to capital and labor.
Oil producers benefit; oil consumers do not.
\end{itemize}
%
To simplify matters, we'll assume that these factors shift both AS and AS$^*$
by the same amount.
%We could consider alternatives, but we won't.
%Trust me, it's not worth the effort.
% ?? think about this


\section{Aggregate demand}

% QT first?

The primary role of the Keynesian aggregate supply curve is to
allow demand to influence output;
if supply is vertical, then changes in demand affect the price level,
but not output.
That's what we assumed when we discussed the quantity theory --- that changes in the money supply affect prices, but not output.
But if aggregate supply is flatter,
shifts in demand affect both prices and output.

But what is the aggregate demand curve and where does it come from?
Recall that aggregate demand refers to the purchase of goods and services.
The aggregate demand curve tells us how much demand for output
there is at each price level.

The simplest version follows from the quantity theory,
\begin{eqnarray}
    M_t V_t &=& P_t Y_t ,
    \label{eq:qt}
\end{eqnarray}
which you might recall from Chapter \ref{chp:mpin}.
The aggregate demand curve is the relation between $P$ and $Y$ for a
given supply of money $M$,
determined presumably by the monetary authority.
For simplicity, we will assume that velocity $V$, is constant.
That gives us an inverse relation between $P$ and $Y$,
as shown in Figure \ref{fig:asad}.
Changes in $M$ lead to shifts in AD.
If we increase $M$, then we need either an increase in output $Y$
or an increase in the price level $P$ to satisfy (\ref{eq:qt}),
so AD shifts up or to the right.

There's a more complex version in which monetary policy operates
through the interest rate.
The idea is that an increase in the money supply reduces
the interest rate,
which stimulates interest-sensitive components of demand,
such as business investment and housing.

Other movements in aggregate demand come from the expenditure components:
demand by consumers ($C$); firms ($I$); government ($G$);
and the rest of the world ($\NX$).
You often read, for example, that high demand by consumers
leads to higher output --- more on this shortly.
Ditto increases in government purchases
(wars are the biggest examples historically)
and investment by firms (remember, investment is the most volatile
expenditure component).
Keynes thought investment fluctuations stemmed,
in part, from the ``animal spirits'' of business people,
which you might think of as shifts in investment demand
driven by psychological factors.


%Our guess is that this leaves you breathless.
%That's ok, our goal was to skim over the details.
%Just take our word for it that aggregate demand involves an
%inverse relation between the price level and output,
%and that a variety of obvious things increase demand:
%increasing the supply of money, government purchases,
%and so on.

Let's summarize.
The aggregate demand curve is downward-sloping, reflecting
the decline in the real money supply as the price level increases.
The following factors (``shocks'')
shift the aggregate demand curve to the right:
%
\begin{itemize}
\item An increase in the supply of money.
\item An increase in government purchases.
\item An increase in consumer demand:  For given levels of income,
consumers decide to spend more.
\item An increase in investment demand:
For given levels of interest rates and output,
firms decide to invest more (animal spirits).
\end{itemize}

\section{Aggregate supply and demand together}

Now let's put supply and demand together.
Equilibrium is where supply and demand cross.
The only difference in this respect from
the traditional supply and demand analysis is that we have two
supply curves.
The short-run equilibrium is where AD and AS cross.
The long-run equilibrium is where AD and AS$^*$ cross.

Here's how that works.
The short-run equilibrium is where aggregate supply AS and aggregate
demand AD cross:  the point labeled A in Figure \ref{fig:asad}.
In this example, the economy's output is below its long-run
value $Y^*$, indicated by AS$^*$.
The reason is that the real wage is too high, leading firms to demand
less work than individuals would like to offer at the going real wage.

The long-run equilibrium is where aggregate demand AD crosses
long-run aggregate supply AS$^*$:  point B in the figure.
How do we get there?
Eventually, the real wage adjusts (either $W$ falls or $P$ rises),
shifting the AS curve down, until
we get to the point where AD crosses the long-run aggregate supply
curve AS$^*$.
The dynamics are governed by the speed of adjustment of wages.
%[Suggestion: show in the figure how AS shifts until we reach B.]


%%%%%%%%%%%%%%%%%%%%%%%%%%%%%%%%%%%%%%%%%%%%%%%%%%%%%%%%%%%%%%%%%%%%%%%%%%%%
%  Supply and demand diagram
\begin{figure}[h!]
\caption{The impact of an increase in the money supply.}
%
\centering
\setlength{\unitlength}{0.075em}
\begin{picture}(250,200)(0,0)
%\footnotesize
\thicklines

% horizontal axis
\put(-30,0){\vector(1,0){300}}
\put(255,-16){$Y$}
\put(142,-16){$Y^*$}

% vertical axis
\put(0,-20){\vector(0,1){200}}
\put(-15,155){$P$}

% demand
\put(25,165){\line(4,-3){200}}\put(230,10){AD}
\put(65,165){\line(4,-3){200}}\put(270,10){AD$'$}

% supply
\put(25,13){\line(4,3){200}} \put(230,160){AS}
%\put(65,13){\line(4,3){200}} \put(270,160){AS$'$}
\put(146.4,0){\line(0,1){170}} \put(138,175){AS$^*$}

% equilibrium labels
\put(105,85){\footnotesize A}
\put(150,115){\footnotesize B}
%\put(138,64){\footnotesize C}
% dotted lines
%\qbezier[31]{(133,0)(133,46)(133,92)}
%\qbezier[45]{(0,92)(67,92)(133,92)}
%\qbezier[45]{(0,72)(67,72)(133,72)}

\end{picture}
\begin{minipage}{0.6\textwidth}
\vspace{0.45in}
{\footnotesize Aggregate demand AD shifts right to AD$'$,
moving the short-run equilibrium from A to B.}
\end{minipage}
\label{fig:asad_m}
\end{figure}
%%%%%%%%%%%%%%%%%%%%%%%%%%%%%%%%%%%%%%%%%%%%%%%%%%%%%%%%%%%%%%%%%%%%%%%%%%%%


Let's put our theory to work:

\textbf{What is the impact of an increase in the supply of money?}
Consider a short-run equilibrium at a point like A in Figure \ref{fig:asad_m}.
Could the monetary authority do something to raise output to its long-run
level?
An increase in the money supply will shift AD to the right,
as in the new aggregate demand curve AD$'$.
If we increase the money supply by the right amount,
we can establish a new
equilibrium at B, where output is exactly its long-run value.
[We recommend that you work through all these shifts of curves on your own
to make sure you follow the argument.]
How was this accomplished?
The increase in the supply of money increased the price level.
Since the nominal wage is fixed, the real wage falls,
making labor more attractive to firms
and thereby increasing employment and output.

Alternatively, suppose that we're at the long-run equilibrium.
What are the short- and long-run effects of increasing the
money supply?
The short-run impact is to raise prices and output, as
we move up the aggregate supply curve.
[You should work this out yourself in a diagram.]
But the long-run effect is to increase prices,
with no impact on output.
Why?  Because monetary policy doesn't affect
the economy's long-term ability to produce.
That's what we saw with the quantity theory.

\textbf{What is the impact of an increase in government purchases?}
You might recognize this as an example of a stimulus program.
Suppose that we start, as we did above, with output below
its long-term value,
such as point A in Figure \ref{fig:asad_m}.
Then, an increase in government purchases shifts aggregate demand
to the right, as illustrated in the same figure.

% ?? extensions: timing and g, tax cuts

\textbf{What is the impact of an increase in the price of oil?}
Consider an initial short-run and long-run
equilibrium, such as point A in Figure \ref{fig:asad_oil}.
An increase in the price of oil
shifts the aggregate supply curve to the left.
Remember:  we shift both supply curves horizontally
by the same amount.
The new short-run equilibrium is at point B,
where AD and AS$^{'}$ cross.
Eventually the short-run aggregate supply curve shifts until
it crosses AD at AS$^{*'}$, giving us long-run equilibrium at C.


%%%%%%%%%%%%%%%%%%%%%%%%%%%%%%%%%%%%%%%%%%%%%%%%%%%%%%%%%%%%%%%%%%%%%%%%%%%%
%  Supply and demand diagram
\begin{figure}[h]
\caption{The impact of an increase in the price of oil.}
\label{fig:asad_oil}
%
\centering
\setlength{\unitlength}{0.075em}
\begin{picture}(250,200)(0,0)
%\footnotesize
\thicklines

% horizontal axis
\put(-30,0){\vector(1,0){300}}
\put(255,-16){$Y$}
\put(142,-16){$Y^*$}
\put(102,-16){$Y^{*\prime}$}

% vertical axis
\put(0,-20){\vector(0,1){200}}
\put(-15,155){$P$}

% demand
\put(25,165){\line(4,-3){200}}\put(230,10){AD}
%\put(65,165){\line(4,-3){200}}\put(270,10){AD$'$}

% supply
\put(65,13){\line(4,3){200}} \put(270,160){AS}
\put(25,13){\line(4,3){200}} \put(230,160){AS$'$}
\put(146.4,0){\line(0,1){170}} \put(138,175){AS$^*$}
\put(106.4,0){\line(0,1){170}} \put(98,175){AS$^{*\prime}$}

% equilibrium labels
\put(150,55){\footnotesize A}
\put(122,75){\footnotesize B}
\put(95,94){\footnotesize C}
\put(95,76){\footnotesize D}
%\put(95,54){\footnotesize D}
% dotted lines
%\qbezier[31]{(133,0)(133,46)(133,92)}
%\qbezier[45]{(0,92)(67,92)(133,92)}
%\qbezier[45]{(0,72)(67,72)(133,72)}

\end{picture}
\begin{minipage}{0.6\textwidth}
\vspace{0.45in}
{\footnotesize Aggregate supply curves shift left from AS/AS$^*$ to AS$'$/AS$^{*\prime}$,
moving the short-run equilibrium from A to B.}
\end{minipage}

\end{figure}
%%%%%%%%%%%%%%%%%%%%%%%%%%%%%%%%%%%%%%%%%%%%%%%%%%%%%%%%%%%%%%%%%%%%%%%%%%%%

Note that this adverse supply shock reduces output but raises the price level.
People sometimes refer to this combination as ``stagflation,''
a term coined to describe what seemed to be a surprising or implausible outcome.  In fact, it's a natural result of leftward
shifts in the aggregate supply curve.
Note that ``supply shocks'' are inflationary,
in the sense that they raise the price
level unless something is done to aggregate demand to offset them.

\textbf{Where do business cycles come from?}
It's obvious that we can generate economic fluctuations by shifting
the aggregate supply and demand curves around.
Less obvious is what kinds of shifts are most common,
or how they lead to the patterns we see in the data.
As a rule, shifts in AD move price and output in the same direction, while shifts in AS move price and output in opposite directions.
By looking at the statistical relation between prices and output,
we can get a sense of whether supply or demand shifts are more
prevalent.



\section{Beyond supply and demand}

This is a nice model, relatively easy to use,
and applicable to lots of things.
But it's theory, not the real world.
Eliza Doolittle notwithstanding,
it's often a mistake to fall in love with your own creation.
Economists have learned to be humble; some might say that we have a lot to be humble about.

The aggregate supply and demand framework can be developed
further, but it has two weaknesses that are hard to overcome.
The first is what we might term general equilibrium:
Many things affect both supply and demand, so it seems
artificial to separate them as we have.
The second is dynamics:  The impact of many shocks depends
not only on what happens now, but also on what we expect to happen
in the future.
That's a hard thing to model explicitly, more so in a single diagram.

Consider the interaction of supply and demand.
If productivity rises, is that a shift of supply or demand?
Obviously it shifts supply:  The production function shifts,
which directly affects producers.
But consumers are also producers.
An increase in productivity  raises wages,
which gives them more income, perhaps
leading them to consume more.
It also raises the demand for capital goods,
since capital is now more productive.
Are these shifts in demand?
In the sense that they change consumption and investment
decisions, the answer is yes.
The closer you look, the harder it is to separate
supply from demand.

Another example is the financial crisis.
Did it shift supply or demand?
Think about it and let us know what you come up with.

Or consider dynamics.
One issue is causality.
Think about popular comments to the effect that
consumer demand is driving the economy.
A journalist might say:  ``High consumer demand led to an economic boom.''
The logic is perfectly consistent with our AS/AD analysis,
but is that really what's going on?
If we think about consumption,
one of our first thoughts should be to think about our future income.
If we expect to have much higher income in the future
(that MBA is really paying off!),
we might consume more now.
But think about what that does to causality. For the economy as a whole, has output gone up because we consumed
more, or did we consume more because we expected output to go up?
It's not easy to tell the difference between the two mechanisms.

Investment is similar.  Firms make investment decisions
based on their assessment (i.e., guess)
of market conditions years down the road.
That's why ``institutions'' are so important. Good institutions give firms some assurance that the rules won't change
in ways that make the investment less attractive.
With respect to business cycles, we could ask the same question
we asked of consumption: Did high investment lead to a booming economy,
or did expectations of a booming economy lead to high investment?
If we're forecasting, we may not care, as long as the two go
together.
But if we want to understand what's going on, we need to address
this issue one way or the other.
Fed minutes and analysts' reports
are filled with conjectures over exactly this kind of issue.

A related issue is what we might call context:
the set of assumptions people use to think about
the connection between current and future events.
Monetary policy is a good example.
In most developed countries,
central banks have worked hard to convince people
that if they increase the money supply now,
it does not signal future increases in the money supply.
If it did, people might immediately demand higher wages,
which would lead to higher prices and inflation.
But if they regard a current increase in the money supply as temporary,
they might very well be content with wages and prices
where they are.
That's one of the reasons that monetary policy is different
in the US and Argentina:
The contexts for understanding current events are different.
It's also an important practical issue for monetary policy: Central banks must signal not only their current policies,
but their likely future policies.
That's exactly what the Fed is struggling with right now --- how to do that effectively.




\section*{Executive summary}

\setlength{\leftmargini}{.5\oldleftmargini}
\begin{enumerate}
\item In the long run, output is determined by the production function:
the productivity of the economy and the behavior and institutions
that determine investment and employment.

\item In the short run, output may respond to changes in
aggregate demand (e.g., the money supply) because of sticky wages or prices
or possibly other market frictions.

\item This is theory, not reality.
There's no substitute for adding some common sense.
\end{enumerate}
\setlength{\leftmargini}{\oldleftmargini}

%\begin{comment}
\section*{Review questions}

\setlength{\leftmargini}{.5\oldleftmargini}
\begin{enumerate}
\item AS/AD review.  Get out a piece of paper and do the following
without looking at the text:
\begin{enumerate}
\item Draw the aggregate demand curve on a piece of paper.
Why does it slope downward?
\item Draw the short-run aggregate supply curve on a piece of paper.
Why does it slope upward?
\item Draw the long-run aggregate supply curve where the two cross.
Why is it vertical?
\item What happens in the short run if we increase the money supply?
In the long run?
\end{enumerate}

Answer. You may refer to Figure \ref{fig:asad_m}.
\begin{enumerate}
\item The AD curve slopes down because a given quantity of money
can support
a high price level $P$ or high real output $Y$, but not both.
See the quantity theory equation (\ref{eq:qt}).
\item The idea is that a higher price level leads, at a fixed wage rate,
to lower real wages, leading firms to hire more workers and expand output.
\item In the long run, wages adjust, and output and employment do not depend
on the price level.
\item If we increase the money supply, aggregate demand shifts to the right.
The immediate impact is to raise output and prices as we move along
the short-run aggregate supply curve.
In the long run wages adjust, leading output to revert to its long-run equilibrium
level and prices to rise.

You might notice that the long-run impact duplicates our analysis of hyperinflation,
where money affects prices but not output.
What we've added is the possibility that monetary policy can affect
output in the short run.
\end{enumerate}

\item Supply or demand?
Suppose exports increase sharply.
Is this a shift in supply or demand?

Answer.  The question is whether this has to do with the production (supply)
or purchase (demand) of goods and services.
Exports are sales, so it's a purchase, hence demand.
We would approach it the same way we approached the increase
in the money supply in the previous question.

%\item How does our analysis of the movement from short-run to long-run
% equilibrium change if wages are very sticky?
%
%Answer.  In some countries, institutions can make the wage very slow to adjust.
%In that case, we might expect slower adjustment to the long-run
%level of output.

\item France.
We've seen that the employment ratio is lower in France than in the US.
Should France increase its money supply in an attempt to increase
employment and output?

Answer.  Good question, but the answer is no:
One suspects that the level of employment
associated with the long-run aggregate supply curve is lower than
markets would produce on their own.
%Why?
%Because this has been true for decades.
%You might consider using monetary policy to drive output beyond the
%long-run supply curve, but eventually you'll end up back there anyway.
%Put another way:  This is a micro issue and macro policy can't solve it.

\item Causality.
Do increases in consumption cause increases in output,
or the other way around?
That is, could the correlation between consumption and output be because
high output leads consumers to spend more, rather than the reverse?

Answer.  We observe data, not causality,
and we may not be able to distinguish between
alternative causal interpretations of the same events.
That's what makes economic analysis so much fun.
In some cases, we might be able to tell the difference,
and these cases might lead to more general insights.
For example, winning the lottery generates an increase in consumption,
and we can say confidently that the causality runs from
higher income to higher consumption.
Why?  Because the reverse argument is absurd: High consumption didn't cause the person to win the lottery.
In most cases, however,
we can have multiple plausible causal interpretations of events,
and there's not much we can do about it.
\end{enumerate}
\setlength{\leftmargini}{\oldleftmargini}


\section*{If you're looking for more}

Similar material is covered in greater depth
in most macroeconomics textbooks.

\section*{Symbols used in this chapter}

\begin{table}[H]
\centering
\caption{Symbol table.}
\begin{tabular*}{0.95\textwidth}{l@{\extracolsep{\fill}}l}
\toprule
Symbol & Definition\\
\midrule
$Y$    &Real output (=real GDP)\\
$A$    &Total factor productivity (TFP)\\
$K$    &stock of physical capital (plant and equipment)\\
$L$    &quantity of labor (number of people employed)\\
$\alpha$     &Exponent of K in Cobb-Douglas production function \\
        &(= capital share of income)\\
$P$    &Price level\\
$Y^*$    &Equilibrium (or potential) output\\
AS    &Short-run aggregate supply\\
AS$^*$    &Long-run aggregate supply\\
AD    &Aggregate demand\\
AD$'$    &Aggregate demand after a shock\\
AS$'$    &Aggregate supply after a shock\\
$M$    &Money stock\\
$V$    &Velocity of money\\
$C$    &Private consumption\\
$I$    &Private investment (residential and business investment)\\
$G$    &Government purchases of goods and services (not transfers)\\
$X$    &Exports\\
$M$    &Imports\\
$\NX$    &Net exports ($=X-M$)\\
\bottomrule
\end{tabular*}
\end{table}


