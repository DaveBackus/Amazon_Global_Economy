\chapter{International Trade}\label{chp:intr}
\hypertarget{trade}{}

\textbf{Tools:} Ricardo's model of trade; consumption and production possibility frontiers.

\textbf{Key Words:} Absolute advantage; comparative advantage; autarky.

\textbf{Big Ideas:}
\vspace{-0.1in}
\begin{itemize}
\item Trade is a positive-sum game:  both countries benefit.
\item Gains from trade are similar to increases in TFP:  trade increases aggregate consumption opportunities.
\item Trade creates winners and losers, but the winners win more than the losers lose.
 Trade affects the kind of jobs that are available, not the number of jobs.
%\item Opposition to trade is often thinly disguised self-interest.
\end{itemize}
\rule{\textwidth}{1pt}

Virtually all economists, liberal or conservative, believe that
free (or ``free-er'') trade is a good thing:
good for consumers, good for workers,
good for all countries involved.
Why?  Because consumers are able to buy products
from the cheapest vendor,
which forces production to the highest productivity firms,
which, in turn, supports the highest wages for workers.

No one else seems to believe that.
Most are convinced that one side of trade is unfair, that one country
is gaining at the other's expense.
The purpose of this chapter is to outline the logic for trade.
The logic is mathematical, by which we mean it's clear and precise,
if a little abstract.
You can decide for yourself whether you find it persuasive.


\section{Ricardo's theory of trade}

David Ricardo was one of the most influential economists of the early
nineteenth century,
but he came to economics by accident.  Born to a
Jewish family in Amsterdam, he left the Netherlands and broke off
relations with his family (and they with him) to avoid an arranged
marriage, and married a Quaker instead.  He set himself up in
London as a government securities dealer and became, in his words,
``sufficiently rich to satisfy all my desires and the reasonable
desires of all those about me."  Looking for something to occupy
his time, he developed the modern theory of international trade.

Many people in Ricardo's day (and ours) regarded trade as a
zero-sum activity:  If {you} gain from trade, then I must lose.  His
insight was that both sides typically benefit, even if it appears
that one has an \textit{absolute productivity advantage} over the
other. In his words, each country has a \textit{comparative
advantage}.

We develop Ricardo's theory in a particularly simple setting:
Two countries produce and consume two products, and both products
are produced with labor alone.  In many respects, this version of
the theory is unrealistic, but the lack of realism is exactly what
makes the analysis simple and understandable.
None of the simplifications are essential to the argument.

To be specific, let us call the countries the US (country 1) and
Mexico (country 2) and the products apples and bananas.  (Yes, we
know neither the US nor Mexico produces many bananas, but we like
the letters $a$ and $b$.)  We start by specifying the productivity
levels: the quantities of product (either $a$ or $b$) in country
$i$ (either 1 or 2) produced with one unit of labor.
We'll use specific numbers, which we report in Table~\ref{tab:prod}.
Let us say, also in the interest of simplicity,
that the labor force is the same in the two countries: $L_{1}=L_{2}=100$.

%*****************************************************************************************************
\begin{table}[h]
\centering
\caption{Productivities for trade example.}
\begin{tabular}{lcc}
\toprule
                    &        Apples       &      Bananas    \\%

\midrule
US (country 1)      &   $\alpha_{1}=20$   & $\beta_{1}=10$  \\%
Mexico (country 2)  &   $\alpha_{2}=5$    & $\beta_{2}=5$   \\%
\bottomrule
\end{tabular}
\label{tab:prod}
\end{table}
%*****************************************************************************************************


The productivities mirror the discussion you hear about trade
between a rich country (the US here) and a poor one (Mexico).
For example, one unit of labor produces more in the US
whether it's used to produce apples or bananas.
We would say the US has an {absolute advantage} in producing both goods.
A number of factors might play a role here. Perhaps the weather is
better, labor is better educated, the distribution system is
more efficient, or the institutions are better.


The question is: Would Mexico and the US both benefit from
completely free trade, relative to a position of \textit{autarky}
(no trade at all)? The answer is yes, but let's run through
the argument. Suppose Mexico had high enough tariffs or other
barriers to kill off trade altogether. Then Mexico would likely
produce both products. How much of each? It could produce apples
in quantity $a=L_{2}\alpha_{2} = 100\times 5 = 500$ or bananas in
quantity $b=L_{2}\beta_{2} = 100\times 5 = 500$. It could also
produce any combination in between, as shown in
Figure~\ref{fig:gains} (the solid line). We call the solid line
the production possibility frontier for Mexico, since every point on the line
represents a possible production/consumption combination. In this example,
the line has a one-for-one tradeoff between apples and bananas,
implying a relative price of $ q=p_{b}/p_{a} =
\alpha_{2}/\beta_{2} = 1$.

What happens if Mexico and the US allow trade? It depends on the
relative price $q$. Suppose that Mexico can export bananas at a
relative price of $q> 1$ apples for each banana. Then Mexico will
produce only bananas.  Why? Because it can produce each at the
same cost ($1/5=0.2$ units of labor), but bananas sell for more on
the world market. As a country, it faces strictly better
possibilities if it trades rather than producing both goods
itself. If it produces only bananas ($b = 500$) and then
trades some for apples at a rate of $q$ apples for every banana,
it does better than the one-for-one tradeoff it got from
producing apples itself.  (See the dashed line in Figure \ref{fig:gains},
which is above the solid line.)  [As a check on your
understanding: How would this work if $q < 1$? What would Mexico
produce?  What would its possibility frontier look like? Would
Mexico still benefit from trade?]

%%%%%%%%%%%%%%%%%%%%%%%%%%%%%%%%%%%%%%%%%%%%%%%%%%%%%%%%%%%%%%%%%%%%%%%%%%%%

\begin{figure}[h]
\caption{Gains from trade in Mexico.} \label{fig:gains}
%\begin{center}
\centering
\setlength{\unitlength}{0.095em}
\begin{picture}(300,200)(-30,-20)%
\thicklines

\put(-30,0){\vector(1,0){300}}%
\put (0,-20){\vector(0,1){200}}%
\put(265,-14){$a$}%
\put(-15,175){$b$}%

\put (110,80){Possibilities with trade}%
\put(5,10){Possibilities without trade}%

\put(0,150){\line(1,-1){150}}%
\qbezier[100](0,150)(100,75)(200,0)%

\put(-23,148){$500$}%
\put(150,-18){$500$}%


\end{picture}
%\end{center}
\end{figure}

%%%%%%%%%%%%%%%%%%%%%%%%%%%%%%%%%%%%%%%%%%%%%%%%%%%%%%%%%%%%%%%%%%%%%%%%%%%%

In short, trade benefits Mexico, even though it is less productive
than the US for both products.  Similar reasoning shows that the
US would benefit from trade, too.  [Another check:  What is the
possibility frontier for the US if there's no trade?
Trade with $q>1$?  Trade with $q<1$?]

Ricardo had a rationale for these gains from trade:  Even though
Mexico is less productive absolutely ($\alpha_{1}>\alpha_{2}$ and
$\beta_{1}>\beta_{2}$), it is comparatively more productive in
bananas than the US
$(\beta_{2}/\alpha_{2} > \beta_{1}/\alpha_{1})$.
Conversely, the US is comparatively more productive in apples
$ ( \alpha_{1}/\beta_{1} > \alpha_{2} /\beta_{2} )$.
If each country produces the good for which it is comparatively
more productive, then world productivity rises and both countries
benefit.
For this reason, Ricardo referred to this as the theory of comparative
advantage.


\section{Digging a little deeper}

Moving to free trade is similar to an
increase in productivity because when you shift production to high
productivity products, aggregate productivity rises.  The impact
is similar to our discussion of labor and financial markets.
Countries with good labor and financial markets allocate inputs
more effectively and increase aggregate productivity as a
result.  This is a natural feature of trade models, but it takes
some effort to work out the details, even in a setting as simple
as our example.  If you're averse to math, you might
skip to the next section the first time through.

Our goal is to compare production and consumption in two cases:
one with no trade, and one with completely free trade (no tariffs
or transportation costs).  The comparison is somewhat extreme, but
the hope is that it will give us the flavor of less-extreme moves
toward freer trade.  In each case, we need to find the competitive
equilibrium.  Competitive means that consumers and producers
take prices as given.  (No monopolies allowed here!)
Formally, a competitive equilibrium is a set of prices and
quantities that satisfies three conditions:
%
\begin{enumerate}
\item Consumers are on their demand curves;  they buy what they
want at the given prices.

\item Producers make zero profits (the effect of competition).

\item Total production equals total consumption for each product.
\end{enumerate}

Finding an equilibrium can be difficult, particularly if you have
a low threshold for algebra, but we can readily verify a
proposed equilibrium by checking the three conditions.

\textbf{Consumers}. The citizens of each country consume apples
and bananas.
They also work for the firms, getting a wage $w$ for each
unit of labor. Each consumer (we can index them by $i$) earns an
income $y_{i} = wl_{i}$ (for simplicity, we assume that $l_{i}$ is
given). Obviously, $L=\sum_{i}l_{i}$. How do consumers spend their
income? Like any of us, they receive satisfaction (or {\it utility\/})
from consuming both apples and bananas.  Let us say that
their utility from consumption is given by the following function:
\[
    U(a,b) \;=\; a^{s}b^{1-s}.
\]
Given her income and prices for apples and bananas, each consumer
will make spending decisions that maximize her utility.
Simple calculations show that a consumer will spend a fraction $s$
of her income on apples and the complementary fraction $1-s$ on
bananas. Summing across all consumers, we find that a
fraction $s$ of national income $Y$
($Y=\sum_{i}y_{i}=w\sum_{i}l_{i}=wL$) is spent on apples,
and the remainder is spent on bananas:
\begin{eqnarray*}
    p_{a}a &=& s Y  \\
    p_{b}b &=& (1-s)Y.
\end{eqnarray*}
These are (effectively) the demand functions for the two products.
We'll assume below that $s = 0.75$ in both countries.

\textbf{Producers.}  Consider producers in a specific country.
Let's say that labor sells for $w$ per unit,
with $w$ potentially differing across countries.
%Consider producers in an arbitrary country $i$.
A producer of apples (say) will hire labor at cost $w$ per unit and sell
apples, getting a profit of
\[
    \mbox{Profit} \;=\;  a (p_{a} - w/\alpha) ,
\]
where $\alpha$ is apple productivity in the country we're examining.
If $p_{a} < w/\alpha$, the price is too low and no apples will be
produced. If $p_{a} > w/\alpha$, competition among apple producers
will drive the price down until $p_{a} = w/\alpha$.  In short, if
apples are produced, their price will be $p_{a} = w/\alpha$.
Similarly, if bananas are produced, their price will be $p_{b} =
w/\beta$. If both apples and bananas are produced (and they need
not be), their relative price will be $q = p_{b}/p_{a} = \alpha
/\beta $.

\textbf{Equilibrium without trade.} If there's no trade, then each
country will produce both products.  Let us say that the wage rate
is $w = 1$ in both countries (but not comparable, because they may
be measured in different units).  Since the total labor input
is 100 in either country, national income is $Y = wL = 1\times 100
= 100$ in both Mexico and the US (again, the units are not comparable).
In the US, prices will be
\begin{eqnarray*}
    p_{a} &=& w/\alpha \;=\; {1}/{20} \;=\; 0.05\\
    p_{b} &=& {w}/{\beta} \;=\; {1}/{10} \;=\; 0.10\\
    q &=& {p_{b}}/{p_{a}} \;=\; 2.
\end{eqnarray*}
At these prices, the demands for apples and bananas are, respectively,
\begin{eqnarray*}
    a &=& {sY}/{p_{a}} \;=\; {0.75\times 100}/{0.05} \;=\; 1500\\
    b &=& {(1-s)Y}/{p_{b}} \;=\; {0.25\times 100}/{0.1} \;=\; 250 .
\end{eqnarray*}
Total utility is, therefore, $U = a^{0.75}b^{0.25} = 958$.

What about Mexico? Using similar methods, we find that prices are
%
\begin{eqnarray*}
    p_{a} &=&  {w}/{\alpha} \;=\; {1}/{5} \;=\; 0.20 \\
    p_{b} &=&  {w}/{\beta} \;=\; {1}/{5} \;=\; 0.20 \\
    q &=& {p_{b}}/{p_{a}} = 1.
\end{eqnarray*}
Demands are $a = 375$, $b = 125$. Utility is $U = 285$.  The
numbers are summarized in Table \ref{tab:price_quant} for future
reference.


\textbf{Equilibrium with trade}.
The complete solution is reported in Table \ref{tab:price_quant},
but let's see where it comes from.
It's moderately complicated,
so skip directly to the next section unless you're
incredibly curious.
The objective is to find prices and wages that equate
supply and demand for apples, bananas, and labor in both countries.
We'll focus on bananas; if the banana market clears,
so do the others.


%*****************************************************************************************************

\begin{table}[h]
\caption{Prices and quantities with and without trade.}
\label{tab:price_quant}
\centering
\begin{tabular*}{0.8\textwidth}{l@{\extracolsep{\fill}}cc}
\toprule
                                  &      Free Trade     &    No Trade    \\%
\midrule
                                  & \multicolumn{2}{c}{US} \\
\midrule%
Price of apples $p_{a}$           &         0.05        &      0.05      \\%
%
Price of bananas $p_{b}$          &         0.0667      &      0.10      \\%
%
Wage $w$                          &          1          &      1 (dollar)      \\%
%
Consumption of apples $a$         &          1,500      &      1,500      \\%
%
Consumption of bananas $b$        &          375        &     250      \\%
%
Utility                           &          1,061      &      958      \\%
\midrule
                                  & \multicolumn{2}{c}{Mexico} \\
\midrule%
Price of apples $p_{a}$           &         0.05        &      0.2      \\%
%
Price of bananas $p_{b}$          &          0.0667     &      0.2      \\%
%
Wage $w$                          &          0.3333     &      1 (peso)      \\%
%
Consumption of apples $a$         &          500        &      375      \\%
%
Consumption of bananas $b$        &          125       &       125      \\%
Utility                           &          354       &       285      \\%
\bottomrule
\addlinespace
\end{tabular*}
\begin{minipage}{0.8\textwidth}
\footnotesize{In the no-trade case, the wages are normalizations;
they're in different units and are not comparable across countries.}
\end{minipage}
\end{table}


%*****************************************************************************************************


Let's guess (we made up the example, so our guesses are pretty good)
that the US produces only apples and  Mexico produces only bananas.
In this way, the two countries specialize in the
production of the good in which they have a comparative advantage.
We'll verify this guess later.
Let's think about how the banana market works.
Supply is whatever Mexico produces given its
available labor $L_2$ and productivity $\beta_2$:
\[
    \mbox{Supply of Bananas}  \;=\;  L_2 \beta_2  .
                %\;=\; 100 \times 5 \;=\; 500.
\]
What about demand?  This is more complicated.
Since each country spends a fraction $1-s$ on bananas,
total demand is
\[
    \mbox{Demand for Bananas}  \;=\;  (1-s) Y_1/p_b + (1-s) Y_2/p_b ,
\]
where $Y_1$ and $Y_2$ are incomes in the US and Mexico,
respectively.
Competition in labor and output markets will equate
income paid to workers to the value of output they produce:
\begin{eqnarray*}
    Y_1 &=&  L_1 \alpha_1 p_a \\
    Y_2 &=& L_2 \beta_2 p_b .
\end{eqnarray*}
Why?  Because competition drives profits to zero.
That gives us
\[
    \mbox{Demand for Bananas}  \;=\;  (1-s) L_1 \alpha_1 p_a/p_b
                + (1-s) L_2 \beta_2  .
\]
Equating supply and demand and doing some algebra gives us
\begin{equation}
    p_b/p_a  \;=\;  \frac{ (1-s) L_1 \alpha_1 }{ s L_2 \beta_2 } .
    \label{eq:equilprice}
\end{equation}
Plugging in numbers, we find $ p_b/p_a = 4/3 $.
If you're unusually curious,
you can show (using the same approach)
that supply and demand are equal for apples at the same price.


We can now get a sense where the relative price
of bananas comes from:  supply and demand!
On the supply side,  higher productivity for bananas (higher $\beta_2$)
drives the price down.
This is the usual shift out of the supply curve.
Similarly, higher apple productivity (higher $\alpha_1$)
makes apples relatively less expensive.
On the demand side,  lower  $s$ indicates a lower desire for apples and a higher
desire for bananas and, thus,  drives the price of bananas up.
This is essentially a rightward shift of the demand curve.




\begin{comment}
If we're wrong, we'll find out shortly. We'll guess the following
prices and show that they work: set $w = 1$ for the US and
\begin{eqnarray*}
    p_{a} &=& 0.05 \\
    p_{b} &=& ({4}/{3}) p_{a} \;=\; 0.0667 \\
    q &=& {4}/{3}
\end{eqnarray*}
for both countries (since there's trade, any other prices in
Mexico would lead to an arbitrage opportunity).  At these prices,
the US will produce $L_{1} \alpha = 100\times 20 = 2,000$ apples
and consume $a = {sY}/{p_{a}} = 1,500$ and $b =
{(1-s)Y}/{p_{b}} = 375$. The total level of utility is 1,061.

What about Mexico?  At these prices, Mexico produces only bananas,
as we guessed.  Total production is $L_{2}\beta = 500$ bananas.
The Mexican wage rate solves $p_{b}\beta = w$ or $w = 0.067\times
5 = 0.33$. (Why is the wage lower than in the US?  Because productivity
is lower.) Mexican income is therefore $Y = wN = 33.3$.
Consumption is $a = 500$ and $b = 125$. Utility is 353.6.



{\it The remainder of this section works through the
solution in greater detail.
It's only for people who want to see where all the pieces come from.
Strangely enough, there generally are some people like this,
but if you're not one of them go immediately to the next section.\/}

We can find the equilibrium by looking at the supply and demand
for apples.
We'll assume for now that the US produces only apples,
Mexico only bananas.
We'll verify this later.
If that's the case, then output of apples is the amount of labor in
the US times apple productivity:
\[
    \mbox{Supply of Apples}  \;=\;  L_1 \alpha_1
                \;=\; 100 \times 20 \;=\; 2000.
\]
What about demand?  We need demand by both countries.
Since each country spends a fraction $s$ on apples,
demand is
\[
    \mbox{Demand for Apples}  \;=\;  s Y_1/p_a + s Y_2/p_a .
\]
Note that US income is $ Y_1 = w_1 L_1 $.
\end{comment}



Finally, we verify that the US produces only apples, Mexico only
bananas.
How do we show this?
At these prices and wages,
US banana producers lose money --- so they don't produce any.
Ditto Mexican apple producers --- the wage rate supported by banana
production is too high for apple producers to break even,
so they won't produce either.
This is really a good thing for Mexican workers, as
producing bananas supports a higher standard of living.

\section{Wages and productivity}

In the US you sometimes hear:  ``US workers can't compete
with Mexican workers, because their wages are so low.''
In Mexico, you sometimes hear:
``Mexican workers can't compete with US workers,
because their productivity is so much higher.''
Who is right?
The answer, of course, is neither.
In our model, wages reflect productivity.
Mexican wages are lower because Mexican workers are less productive.
Their wage is low enough to (just) make up for their lower productivity.
Ditto American workers:  Firms hire them despite their higher wage
because their productivity is higher.
The value of labor to a firm is a balance between the two forces:
price and productivity.

We can be more specific about the connection between productivity
and wages.
As a rule, the wage ratio will be somewhere between
the productivity ratios for the two products.
In this case,
the ratio of the US to the Mexican wage will be between
2 ($=10/5$, the ratio of banana productivities)
and 4 ($=20/5$, the ratio of apple productivities):
\[
  2 = \beta_1/\beta_2   < w_1/w_2  < \alpha_1/\alpha_2 = 4.
\]
If we were to (somehow) force up the Mexican wage
above the upper bound, we would simply make Mexican bananas
more expensive to Americans than producing them locally would.
Demand for Mexican labor would dry up.
Similarly, if we were to force down the Mexican wage below
the lower bound, Mexico would find it profitable to produce
both goods.
However, demand for Mexican labor would exceed supply,
which you'd expect to increase its price.

Overall, wages are connected to productivities.
Between the two bounds, demand plays a role, as we've seen.
If people have a stronger desire for bananas, that tends
to benefit the Mexican workers who produce them
by increasing the price of bananas,
as we see in equation  (\ref{eq:equilprice}).


\section{Bottom line}

Let's think about the calculations summarized in
Table \ref{tab:price_quant} from a nontechnical perspective.
The numbers make several points that
extend to more-general settings:

\begin{itemize}

\item \textbf{Trade makes consumers (=workers) better off.}
In the US, consumption of apples stays the same and consumption of
bananas increases. As a result, utility rises from 958 to 1061. In
Mexico, consumption of bananas does not change, but consumption of
apples is larger. Therefore, utility rises from 285 to 354. In
more-realistic models, the impact of trade is typically small,
but both countries gain, as they do here.
It's a byproduct of Adam Smith's invisible hand (aka the first
theorem of welfare economics), which you might recall from Firms
and Markets.

\item \textbf{Trade changes production.}  In this
case, Mexico shifted out of apples into bananas, and the US did the reverse.
In other models, the change in production may not be
so extreme, but it's generally true that they predict that every
country will stop producing some products and import them
instead.  The result is a more efficient system of production,
as each country produces those goods for which its relative
productivity is the highest.

\item Both effects show up in macroeconomic data as increases in
productivity.  If we were NIPA people, we might compute GDP this way:  sum production of apples and bananas, valued at a
consistent set of prices.  In this case, we'll use the free-trade
prices, which is similar to PPP adjustment (apply the
same prices in every country).  GDP at world prices is
%
%*****************************************************************************************************
%
%\begin{table}[h]
\begin{center}
\begin{tabular}{lcc}
\toprule
                    &      Free Trade     &    No Trade    \\%

\midrule%
US                  &         100.0       &      91.7      \\%
%
Mexico              &          33.3       &      27.1      \\%
\bottomrule
\end{tabular}
\end{center}
%\end{table}
%
%*****************************************************************************************************
%
Once trade shows up in GDP, it shows up in aggregate
productivity, too.  We don't have capital in this model, so the
production function is $Y = AL$.  Since $L$ is unchanged across
trade regimes, the change in $Y$ reflects an increase in TFP.

\item \textbf{No jobs were lost --- or found.}
In our example, every unit of labor was used whether trade was possible or not.
This is only a little extreme:  No trade models suggest that trade will have much
impact on employment.  Any effect there might be comes from the
impact on labor supply of an increase in the wage.
So when you read the newspaper, especially in an election year,
remember that trade has an impact on what the jobs are, not on how
many there are.
\end{itemize}

\section{Winners and losers}

From what we've seen, trade is a wonderful thing.  Who could
be against it?  In fact, lots of people seem to have a
passionately held view that trade and globalization are a
plague on the world.  What could they be thinking? What follows is
a short list of their arguments.
In practice, our experience is that most arguments against trade
are simply self-interest in disguise.


\textbf{Externalities}.  This is a classic ``failure" of markets,
the (unpriced) impact of one person's decision on another's
utility. For example, a polluting producer may inflict bad air on
you and reduce your welfare. When talking about trade, people
often refer to positive external effects on productivity.  Are
there advantages to having a local industry beyond the profit and
loss? Could it help others to increase their efficiency? This is a
legitimate argument, but probably not a good one in most cases.
Moreover, it's typically used by firms and industries looking for
special deals from their governments. For example,
European car makers used this argument when seeking government protection from
Japanese and Korean imports. Their argument was that the domestic
producers generated technology spillovers that benefited
related industries.

\textbf{Differences among residents of a country.} We rushed over
it, but built into our example was that all citizens of a
country have the same tastes and the same productivity in the
workplace. In practice, this is not true, and trade will affect
each person differently. In the example summarized in
Figure~\ref{fig:gains}, all Mexican consumers are better off. Now
suppose that Mexicans differ in how much they like apples and bananas
(i.e., the parameter $s$ is not the same across
individuals). In this case, the ones who like apples less and
bananas more may be worse off since the relative price of bananas
has gone up with free trade. In short, there can be losers. What
the theory says, however, is that the winners win a lot more than
the losers lose --- Mexicans gain, on average.
In principle, you might want to take some of the winners' gains and give them to the
losers, but in practice this isn't that easy to do.

Another example shows up regularly in the press: people who lose their
jobs when production adjusts to trade.  In this case, suppose that you
worked for an apple producer and lost your job. The long-term
answer is:  Get a job working for a banana producer, since their
productivity is higher. But in the short run, there's no question
that you suffer a loss from losing your job. Also, if working for a
banana producer requires skills that you do not have, you might
have to retrain yourself. Again, the winners should be able to
compensate the losers and still be better off, but in practice it
rarely happens. More importantly, people lose jobs all the time for lots of
reasons, and trade is unlikely to be a major factor in most cases.


\section*{Executive summary}

\setlength{\leftmargini}{0.5\oldleftmargini}
\begin{enumerate}
\item International trade allows consumers to buy products more cheaply
and workers to take jobs where their productivity is highest.

\item There can be both winners and losers from trade,
but, in theory, the gains outweigh the losses in every country.
\end{enumerate}
\setlength{\leftmargini}{\oldleftmargini}

\section*{Review questions}

\setlength{\leftmargini}{0.5\oldleftmargini}
\begin{enumerate}
\item Gains from trade. In Mexico, how does consumption
of apples and bananas change when we move from No Trade to Free Trade?
Are Mexican worker better off?
Who in Mexico loses?

Answer.  We read from Table \ref{tab:price_quant}.
Consumption of apples rises from 375 to 500,
and consumption of bananas stays the same at 125.
So Mexicans (who are both workers and consumers) are better off.
No Mexicans lose here, but you might imagine in a different world
that the former Mexican producers of apples suffer.
What we know in general is that the gains outweigh any such losses.

\item Trade politics.
Although the logic for trade is clear,
politicians all over the world complain about ``unfair'' competition
from abroad.
Why?

Answer.
It's hard to see this as anything but protection of their supporters.
In addition, people don't vote in other countries' elections.
In our example, you can imagine Mexican apple producers asking their
government to protect them from US competition.
What they should do here is switch to bananas, but the
the political process often favors the well-connected over the
average worker or consumer.


\item Changing demand.
In the example, show that an increase in $s$ to 0.78
is good for US workers and bad for Mexican workers.
Why might that be?

Answer.  This increases the price of the product produced by US workers,
which improves their situation.  The opposite is true for Mexican workers.
We say that the ``terms of trade'' (relative price of their export good) have moved against them.
Think of an oil-exporting country:  An increase in the price of oil
is good for them, bad for importers.  [Duh!]

\item Could there be losers?
If trade eliminates apple-producing jobs in Mexico,
could apple producers and workers be worse off?

Answer.  Yes!  But the gains for others are typically
larger than these losses,
so we should be able to compensate the losers and leave
everyone better off.
This is trickier than it sounds, though.

\item Food prices and trade.
When food prices rose sharply in 2008, India restricted
food exports to keep prices down.
Who would you expect to benefit from this policy?  Lose?
Is the overall impact on the Indian economy likely to be
positive or negative?

Answer.  You might expect this to keep prices down in the short run
because we've reduced the demand for locally produced food.
(You could also express this as an increase in supply to the domestic market, but it's cleaner this way.)
Who gains?  Domestic food buyers and foreign producers.
Who loses?  Domestic sellers/producers and foreign buyers.
Generally, any market intervention like this is a net loss.
You could show this formally using a supply and demand diagram.

%A related issue is that high prices might be expected to encourage
%an increase in production.
%We missed out on that by keeping prices low.
%In fact, we would expect supply to fall, which makes things worse.
%Not part of the question:
%We would expect poor people to be among the producers,
%so it's not clear why this policy was so popular.
\end{enumerate}
\setlength{\leftmargini}{\oldleftmargini}

\section*{If you're looking for more}

The personal information about Ricardo comes from
the profile on the New School's
History of Economic Thought
\href{http://homepage.newschool.edu/~het/profiles/ricardo.htm}{web
site}.
Doug Irwin's
``\href{http://www.econlib.org/library/Columns/Irwintrade.html}
{History of trade policy}'' is a nice overview of two centuries of thought on trade issues.

% ?? Symbol table


%\newpage
%\section*{Appendix (optional)}
%
%The demand functions used in our example come from the following utility maximization problem.
%To understand it, you'll have to polish off your calculus.
%The consumer chooses quantities $(a,b)$ of apples and bananas to maximize the utility function
%\begin{eqnarray}
%    u(a,b) &=&  a^{s} b^{1-s}
%    \label{objective}
%\end{eqnarray}
%subject to the budget constraint
%\begin{eqnarray}
%     p_{a}a + p_{b}b &=& Y .
%     \label{constraint}
%\end{eqnarray}
%One way to approach this problem is to solve (\ref{constraint}) for $b$ and
%substitute in (\ref{objective}),
%this leaving us with the single variable $a$.
%Doing this leads to
%%
%\[
%    u(a,b) \;=\; \left( \frac{s Y}{p_{a}}\right)^{s} \left(\frac{(1-s) Y}{p_{b}}\right)^{1-s}
%        \;=\; Y \left( \frac{s}{p_{a}}\right)^{s}\left(\frac{1-s}{p_{b}}\right)^{1-s}.
%\]
%
%Therefore the sum of utility across all consumers is
%
%\[
%w\left[\frac{s}{p_{a}}\right]^{s}\left[\frac{1-s}{p_{b}}\right]^{1-s}\sum_{i}l_{i}=w\left[\frac{s}{p_{a}}\right]^{s}\left[\frac{1-s}{p_{b}}\right]^{1-s}L=\left[\frac{swL}{p_{a}}\right]^{s}\left[\frac{(1-s)wL}{p_{b}}\right]^{1-s}.
%\]


