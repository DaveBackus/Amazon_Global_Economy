\part{Crises and Other Topics}
\chapter*{Crisis Overview}
\hypertarget{}{}

\rule{\textwidth}{1pt}

This outline covers key concepts from the third part of the course:
macroeconomic crises and other topics.
It is not exhaustive, but is meant to help you
(i)~anticipate what is coming and
(ii)~organize your thoughts later on.

\medskip
\textbf{\hyperref[chp:tax]{\underline{Taxes}} and \hyperref[chp:dbdf]{\underline{Government Debt and Deficits}}}

\textbf{Tools:} Welfare triangles; government budget constraint; debt dynamics.

\textbf{Key Words:} Tax wedge; deadweight loss; primary deficit/surplus.

\textbf{Big Ideas:}
\vspace{-0.1in}
\begin{itemize}
\item Tax systems should be (i) administratively simple and transparent and (ii) have a broad tax base.
\item Government spending must be paid for, either now through taxes, or in the
future by running primary surpluses.
\item Changes in the ratio of debt to GDP have three sources:
interest, growth, and primary deficits.
\end{itemize}


%\begin{comment}
\hyperref[chp:bop]{\textbf{\underline{International Capital Flows}}}

\textbf{Tools:} Balance of international payments; dynamics of net foreign assets.

\textbf{Key Words:} Current account; net exports; capital account; capital flows; net foreign assets.

\needspace{4\baselineskip}
\textbf{Big Ideas:}
\vspace{-0.1in}
\begin{itemize}
\item A current account deficit means that a country is borrowing from the rest of the
world; a current account surplus means it is lending to the rest of the
world.
We refer to the former as a capital inflow and the latter as a capital outflow.
\item A capital inflow (borrowing) can lead to problems if
it does not support productive activities.
\end{itemize}
%\end{comment}


\textbf{\hyperref[chp:fxf]{\underline{Exchange-Rate Fluctuations}} and \hyperref[chp:fxr]{\underline{Exchange-Rate Regimes}}}

\textbf{Tools:} Arbitrage arguments; central bank \index{central bank} balance sheet.

\textbf{Key Words:} Real and nominal exchange rates; purchasing power parity; covered/uncovered interest parity;
spot and forward exchange rates; the carry trade; convertibility; capital mobility; capital controls \index{exchange rate regime!capital controls}; fixed (pegged) vs. flexible (floating) exchange rate regimes; foreign exchange reserves; sterilization; speculative attack.

\needspace{4\baselineskip}
\textbf{Big Ideas:}
\vspace{-0.1in}
\begin{itemize}
\item Short-run movements in real exchange rates are largely unpredictable.
\item Countries adopt different exchange rate regimes:  fixed, floating, and in between.
The trilemma limits our policy options:  we can choose only two of
(i)~fixed exchange rate, (ii)~free flow of capital,
and (iii)~discretionary monetary policy.
\item Fixed exchange rate regimes must be defended through open market operations and are vulnerable to speculative attack.
\end{itemize}

\textbf{\hyperref[chp:cris]{\underline{Macroeconomic Crises}}}

\textbf{Tools:} Crisis triggers and indicators.

\textbf{Key Words:} Sovereign default; bank runs and panics; refinancing (rollover) risk; leverage; conditionality;
solvency and liquidity.

\textbf{Big Ideas:}
\vspace{-0.1in}
\begin{itemize}
\item Common triggers of macroeconomic crises are sovereign debt problems, financial fragility,
and fixed exchange rates.

\item Measures related to these triggers can help identify countries in trouble:
debt and deficits, financial weakness, exchange rate regime, and so on.

\item The goals of crisis prevention and crisis management are often at odds.
\end{itemize}
