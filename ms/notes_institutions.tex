\chapter{Institutions and Policies}\label{chp:insp}
\hypertarget{institutions}{}

%No extra line here.
\textbf{Key Words:} Institutions; governance; time consistency; property rights; markets.

\textbf{Big Ideas:}
%\vspace{-0.1in}
\begin{itemize}
    \item Cross-country differences in productivity (TFP) are connected to differences in institutions that shape productivity and policy.
    \item Good institutions include good governance; time consistency; rule of law; property rights; open and competitive markets.
\end{itemize}

\rule{\textwidth}{1pt}

The enormous international differences in GDP per person
reflect, in large part, enormous differences in
productivity.
But where do these differences in productivity come from?
It's tempting to attribute them to the ability and dedication
of the people who live there,
but (on second thought) there are smart, dedicated people everywhere.
We now believe that productivity reflects
the quality of local institutions and policies.
Stated more concretely:
it's not Steve Jobs who makes an economy productive;
it's the institutions and policies that allow and encourage someone like Jobs to
operate effectively.
Some countries have environments that encourage productive
activity, and others do not.
What's striking is not that this is true,
but how big a difference it seems to make.


\section{Good institutions}

So what do we mean by good institutions? \index{institutions}
The world's a complicated place, and it doesn't come
with any simple recipes.
But countries with good economic performance
 share some features.
We would say good institutions are social mechanisms
that facilitate good economic performance. Here's a short list.

\textbf{Good governance. \index{governance}}
It's essential that the government be strong enough
to guarantee the security and safety of the country
and people,
but not so strong that those in power abuse others for their own benefit.
It's a delicate balance,
but most productive economies have both strong governments
and clear limits to the government's power.

\textbf{Time consistency.\index{time consistency|textbf}}
\phantomsection
\label{sec:time_cons}Policy consistency over time reduces uncertainty and supports economic
growth. Institutions that allow governments to commit credibly to
good long-run policies (low inflation, fiscal prudence, etc.) help
reduce risks and allow businesses to plan with confidence.\index{inflation}

If governments can easily renege on promises (say, to keep
inflation and taxes low) when it suits them,
economic performance suffers. Finn Kydland \index{Kydland, Fynn}
 and Edward Prescott \index{Prescott, Edward}
shared the 2004 Nobel prize partly for their analysis of this ``time
consistency" problem, which arises not just in economics
but in many walks of life, from child-rearing to diplomacy,
to military strategy.

In formal research, the lack of time consistency is known as the
``dynamic inconsistency of intertemporal plans,'' which arises when a
future policymaker is likely to be motivated to break a current policy
promise. Institutions and practices that help governments pre-commit
to future policies in a credible way --- such as the announcement of inflation\index{inflation} targets
by independent central banks \index{central bank} or the constitutional prioritization
of debt payments by state governments --- help overcome
the time-consistency problem.

Such pre-commitments typically involve the introduction of
rules that limit \emph{policy discretion}. You might think that allowing
future policymakers complete discretion would result in the best possible
policies. However, in these notes you will find numerous examples in which the
ability to pre-commit results in better economic outcomes
(such as keeping inflation low or fostering greater investment).
The reason is that a commitment to prudent policies has a favorable influence
on the expectations and behavior of households and businesses today. When
economists incorporate the analysis of time consistency into
their assessment of various policy approaches, the age-old choice between
policy rules and policy discretion usually tips in favor of rules.

\textbf{Rule of law.\index{rule of law}}
It's also important that the legal system enforce the law:
that the police and judiciary are honest and
enforce the laws of the land.

\textbf{Property rights. \index{property rights}
}
We sometimes take this for granted,
but the laws should be clear about who owns what.
Without that, effective economic activity is impossible.
How can you sell something you don't own?
Imaginative people may be able to do just that,
but it's not a sound basis for a productive economy.
How can you get a mortgage if you can't establish
that you own real estate?
Why would anyone lend on those terms?

\textbf{Open and competitive markets. \index{competitive markets}}
You often hear about ``free markets,''
but what seems to work best are honest, open, flexible, competitive markets \index{competitive markets}
for products as well as capital and labor.
That's different from what you might term business-friendly governments,
those who protect sellers from competition or fraud.
The idea is not to protect producers,
but to allow them to compete honestly.

We'll give examples of each in class, but you might try to think of your own.
%Should we protect farmers?
%Newspaper publishers?


\section{Institutions or policies?}

Institutions bring to mind the difference between North and South Korea.
The two countries have the same culture --- and the same history until 1950.
At that time, living standards were similar,
probably a little higher in the North.
Today, best estimates indicate that GDP per capita in the South
is more than 15 times that of the North.
The huge difference in performance surely reflects the huge difference
in institutions between the countries:
the form of government and the nature of economic activity.


In other cases, policies may play an important role.
We think of policies as less fundamental aspects of
the economic environment than institutions.
An honest judicial system is an institution,
but tax rates and government spending are policies.
There's a fuzzy line between the two,
but the idea is that policies are more easily
changed than institutions.

Peter Henry  \index{Henry, Peter}
 (our dean) and Conrad Miller \href{{http://www.aeaweb.org/articles.php?doi=10.1257/aer.99.2.261}}{illustrate}
the role of policies in a comparison
of Barbados and Jamaica.
We'll draw liberally from their paper.
They note that the two countries have similar backgrounds and institutions:
\begin{quote}
Both [are] former British colonies,
small island economies,
and predominantly inhabited by the descendants of Africans....
As former British colonies, Barbados and
Jamaica inherited almost identical political,
economic, and legal institutions: Westminster
Parliamentary democracy, constitutional protection of property rights,
and legal systems rooted in English common law.
\end{quote}
Nevertheless,
Barbados grew 1.3 percent a year faster between 1960 and 2002,
giving it a substantially higher standard of living.
(This difference is larger than it looks --- the power
of compound interest and all that.)


One clear difference between the two countries was their
macroeconomic policies.
In the 1970s, Jamaica increased government spending
on job creation programs, housing, food subsidies, and many other things.
When tax revenue failed to keep up, the government found itself
with large, persistent budget deficits, which they financed by
borrowing from the central bank. \index{central bank}
This, in turn, led to inflation\index{inflation} rates of 20 percent and higher.
A fixed exchange rate raised the price of Jamaican goods relative to imports,
which led to restrictions on imports and wage and price controls.\index{government budget!budget deficit@budget (or government) deficit}

Barbados also had a fixed exchange rate,
but combined it with fiscal discipline, monetary restraint,
and openness to trade.
The result was a very different macroeconomic outcome.
It's possible other factors played a role, too,
but in this case policies were arguably
as important as institutions.

\begin{comment}
\section{Other factors}

[??]

Democracy, resources, education,....

Summarize evidence.  Mention Barro, Easterly...

Cause or effect?
\end{comment}

\section*{Executive summary}

\begin{enumerate}
\item Good institutions are the primary source of good economic performance.
\item A short list would include:  governance, rule of law,
property rights, and open competitive markets.
\item Stable and predictable macroeconomic policies matter, too.
\end{enumerate}

\section*{Review questions}

\setlength{\leftmargini}{.5\oldleftmargini}
\begin{enumerate}
\item Foxconn's next frontier. 
Hon Hai Precision Industry Co. Ltd. (``Foxconn'') is a Taiwan-based manufacturer that makes
products for Apple, Intel, Sony, and others.
Known for its plants in China, including one in Shenzhen that makes iPads,
it also has operations in Brazil, Malaysia, Mexico, and other locations.


\begin{table}[h]
\centering
\tabcolsep = 0.05in
\begin{tabular}{lrrr}
\toprule
Indicator & China & Thailand & Vietnam \\
\midrule
\multicolumn{2}{l}{\it General} \\
GDP per capita  (2005 USD) &  8400 & 9200 & 3500  \\
Doing Business overall (percentile) & 50.8  &90.3 & 46.5 \\
World Economic Forum overall (percentile) & 80.0 & 73.6 & 47.9\\
\midrule
\multicolumn{2}{l}{\it Governance} \\
Political stability (percentile)  &  25.0 & 16.5 & 52.8 \\
Govt effectiveness (percentile)   &  60.7 & 59.7 & 45.0 \\
Regulatory quality                & 45.5  & 56.4 & 29.4\\
Rule of law                       & 41.8 & 48.8 & 39.9 \\
Control of corruption (percentile) & 30.3 & 43.6 & 33.6  \\
\midrule
\multicolumn{2}{l}{\it Labor} \\
Minimum wage (USD per month) &  204 & 118 & 65 \\
Severance after 10 years (weeks of pay) & 43 & 50 & 43 \\
Labor market efficiency (percentile) & 71.5 & 47.2 & 64.6 \\
Literacy (percent of adults)        & 94 & 94 & 93 \\
Years of school (adults)        & 8.2 & 7.5 & 6.4 \\
\midrule
\multicolumn{2}{l}{\it Infrastructure and trade} \\
Infrastructure quality (percentile)  & 66.7 & 68.1 & 34.0 \\
%\midrule
%\multicolumn{2}{l}{\it International trade} \\
Export documents required (number) & 8 & 5 & 6\\
Export delay (days) &  21 & 14 & 21  \\
Export cost (USD per container) &  580 & 585 & 610 \\
\bottomrule
\end{tabular}
\caption{Institutional indicators for China, Thailand, and Vietnam.
Percentiles range from 0 (worst) to 100 (best).
Sources:  Penn World Table, World Economic Forum, World Bank, Doing Business.}
\label{tab:ctv}
\end{table}

With wages rising rapidly in China, Foxconn is exploring other locations.
As a private consultant, you have been asked to write a short report
outlining the advantages and disadvantages of locating in Thailand and Vietnam
and to compare both to China.
You collect the information in Table \ref{tab:ctv} and begin your report.

\begin{enumerate}
\item Which of these indicators are most important to your venture?
How do the two countries compare on them?
\item Which country or countries would you recommend to your clients?
What are the primary challenges they would face?
\end{enumerate}

Answer. 
This is a qualitative question, but here's an outline
of what an answer might look like.
A good answer should put some structure on the analysis,
not simply list what's in the table.

\begin{enumerate}
\item If you build a plant in another country, you'll be concerned
with overall institutional quality,
property rights (whether the government might steal the plant),
labor cost and quality,
labor market institutions,
and the challenges of exporting your product.
There's no clean link to the indicators, but you might guess that
property rights would be related to the governance indicators,
esp political stability and the rule of law.
The labor indicators obviously address concerns with labor.
And infrastructure and trade address the challenges of exporting.

As a rough guide:
\begin{itemize}
\item Overall:  It's interesting that Doing Business rates
Thailand highest, but the World Economic Forum rates China highest.
And the differences are large.  In the real world,
this would call for a closer look.
Ditto the source of political instability in Thailand.
\item Property rights and overall:  Thailand looks a bit better than the
others on Control of Corruption and Rule of Law, Vietnam looks better
on Political Stability.
\item Labor cost and quality:  Vietnam is considerably cheaper than the other two,
if we use GDP per capita or the minimum wage as rough guides to wages.
Literacy is similar in the three countries, China is highest, and Vietnam lowest,
on education.
\item Labor institutions:  The World Economic Forum ranks China highest,
and Thailand lowest, on overall labor market efficiency.
Another thing that's worth a closer look.
Severance looks similar.

\item Exporting:  cost and delay look similar, but Vietnam
has the worst infrastructure.
You'll want to look into this, see what aspects of the infrastructure
are likely to affect you.
\end{itemize}

\item 
To me, they both look like reasonable candidates.
For Thailand, I'd want to look closer at political stability,
see what that represents and think about how it would affect me.
(And that's an understatement!) 
For Vietnam, I'd want to look closer at infrastructure.
\end{enumerate}


\item Investing in China and India.
You work at a British asset management company and have been asked to
assess the potential of starting a country fund:
a mutual fund for UK investors
that would invest in China or India.
You realize that both countries are growing rapidly,
China more so to date than India,
but you wonder whether there are important
differences in the institutional environment
that might also be relevant.

\begin{table}[h!]
%    \tabcolsep = 0.2in
    \centering
    \begin{tabular}{lccc}
    \toprule
    Indicator    &  China   &  India  &    UK    \\
    \midrule
    GDP per capita (USD) &  5,300  &  2,700 &  35,300   \\
    GDP growth (\%)    &  11.2  &   8.4   &  2.9   \\
    Competitiveness  &  4.6    &  4.3   &  5.4   \\
    Regulatory quality &  4.8 &  4.9  &  9.8   \\
    Rule of law   &  4.6  &  5.8  &  9.3    \\
    Investor protection & 5  & 6  &  8   \\
    Financial sophistication  &  3.3  &  4.9  &  6.2  \\
    Macro stability    &  6.0  &  4.2  &  5.2   \\
    Control of corruption    & 3.8  &  5.3  &  9.4  \\
    \bottomrule
    \end{tabular}
    \caption{Measures of performance and institutional quality
    in China, India, and the UK.
    Competitiveness index is an overall measure of institutional quality.
    Sources:  CIA Factbook, Economist Intelligence Unit, World Economic Forum, 
    World Bank.}
    \label{tab:institutions}
\end{table}


Your summer intern collects the data in Table \ref{tab:institutions}
and explains what each of the indicators means.
In addition, she points out that
the World Economic Forum (WEF) collects survey responses about
the biggest problems faced by businesses.
In China they are:  access to financing, bureaucracy, corruption, and policy instability.
In India:  infrastructure, bureaucracy, labor regulations, and corruption. And in the UK:  taxes, education of workforce, and bureaucracy.

Based on this information and your own experience,
which country would you recommend? Why?

Answer.  
This is a relatively unstructured question,  but also a realistic one.  
A good answer probably touches on the following points:
%
\begin{itemize}
\item Country performance.
The guess is that returns will reflect country performance.
To the extent China is growing faster,
it's probably the better bet.

\item General institutions.
Institutions are helpful for predicting future performance,
and for indicating whether that growth will be claimed
by the people who produce it.
If you look at ``competitiveness,'' the WEF's overall measure of
institutional quality, China ranks (slightly) higher.
Most measures will find little difference between them,
this one favors China by a small amount.
Corruption is an issue in both places, although there's
some indication that India controls it better.
Bureaucracy is an issue in both countries.
Political instability is mentioned as an issue in China,
and could be relevant in the sense that changing regulations
are difficult to deal with.

\item Investment-specific institutions.
There are specific institutions that pertain directly to financial
markets; as we've seen, it takes a lot of regulatory infrastructure
to make financial markets work well, even in developed countries.
Here India looks somewhat better than China.
Overall regulatory quality is better,
as are investor protection, rule of law, and financial sophistication. Access to financing is an issue in China,
but that's irrelevant to this endeavor.

\item Bottom line.  
India has, in some respects, more developed
institutions for capital market activity.
It's partly a matter of history, partly of how the countries
have evolved over the last 30 years.
It takes a fairly sophisticated set of institutions to get
bond and equity markets to work effectively,
and China probably has further to go in this dimension right now.
\end{itemize}


\end{enumerate}
\setlength{\leftmargini}{\oldleftmargini}


\section*{If you're looking for more}

The comparison of Barbados and Jamaica comes from Peter Henry and Conrad Miller,
``\href{{http://www.aeaweb.org/articles.php?doi=10.1257/aer.99.2.261}}
{A tale of two islands}.''

Here are some other good reads, in order of increasing length:
\begin{itemize}
\item Ben Bernanke\index{Bernanke, Ben},
``\href{http://www.federalreserve.gov/newsevents/speech/Bernanke20110928a.htm}
{Lessons from emerging markets}.''
Nice short summary of what good institutions and policies look like.

\item Nicholas Bloom and John Van Reenan,
``\href{http://www.aeaweb.org/articles.php?doi=10.1257/jep.24.1.203}
    {Management practices across firms and countries}.''
They connect productivity to management practices, including
monitoring, targets, and incentives.
Some find this obvious, but we find it reassuring that
good management has a measurable difference
on performance.

\item Bill Easterly,
\href{http://www.amazon.com/Elusive-Quest-Growth-Economists-Misadventures/dp/0262550423}
{\it The Elusive Quest for Growth}.
Essentially a collection of essays on topics related to helping poor countries,
unusually witty for an economist.

\item David Landes,
\href{http://www.amazon.com/Wealth-Poverty-Nations-Some-Rich/dp/0393318885}
{\it The Wealth and Poverty of Nations}.
Less witty than Easterly, but he gives us an interesting historical
perspective on the major countries of the world:  Europe, India, China, etc.
\end{itemize}

The idea of good institutions has been around forever, or close to it,
but we now have better measures of institutional quality than we used to.
One of the leading sources is the World Bank's Doing Business, available at

\vspace*{\parskip}
\centerline{\url{http://www.doingbusiness.org/}.}

The reports of the Economist Intelligence Unit are thoughtful aggregators
of this kind of information.
%We'll discuss other sources in class.
