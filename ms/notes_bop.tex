\chapter{International Capital Flows}\label{chp:bop}
\hypertarget{bop}{}

%No extra lines here.
\textbf{Tools:} Balance of international payments; dynamics of net foreign assets.

\textbf{Key Words:} Trade balance; net exports; current account; capital account; capital flows; net foreign assets.

\textbf{Big Ideas:}
%\vspace{-0.1in}
\begin{itemize}
\item A current account deficit implies that a country is borrowing from the rest of the
world; we refer to this as a capital inflow.
A current account surplus --- a capital outflow --- implies that a country is lending to the
rest of the world.
\item
A capital inflow (borrowing) can lead to problems if
it does not support productive activities.
For this reason, analysts often focus on the reasons for capital flows as well as their magnitude.

\item The dynamics of net foreign assets are analogous to government debt dynamics.
\end{itemize}

\rule{\textwidth}{1pt}

International trade in goods and assets are at all-time highs all
over the world.
In these notes, we describe the measurement system
used to track such trades:  the balance of payments (BOP), a close
relative of the National Income and Product Accounts (NIPA) focusing
on international transactions.  This is simply accounting, in the
sense that we're counting things in a consistent way and not
applying any particular theoretical framework.  Nevertheless, an
important idea emerges: Countries that run trade deficits can also
be thought of as attracting foreign investment or borrowing from
abroad. This connection between flows of goods and flows of assets
gives us a new perspective on issues such as persistent trade deficits.


\section{Trade in goods, services, and income}

The balance of payments starts with a measurement system
for trade in goods and services and related flows of income.
See Table~\ref{tab:usbop}.


\begin{table}[b!]
\centering
\caption{US balance of payments. }
%\begin{tabular*}{0.75\textwidth}{l@{\extracolsep{\fill}}r}
%\toprule
%% Net exports of goods   &  \\
%% Net Exports of services & \\
% Net exports of goods and services & --559.8 \\
% Net labor income from ROW  & --8.0 \\
% Net capital income from ROW &   235.0  \\
% Net taxes and transfers from ROW & --133.0 \\
% {\bf Current account}  & {\bf --465.8} \\
% \\
% Net direct investment in US &  --185.3 \\
% Net purchase of securities  &  210.4  \\
% Net loans and other  & 531.2  \\
% {\bf Capital and financial account (�inflows�)} &  {\bf 556.3}  \\
% \\
% {\bf Statistical discrepancy} &  {\bf 89.2} \\
%\bottomrule
%\addlinespace
%\end{tabular*}
%Created by hand from the workbook tab_bop.xlsx
%We don't have a good way to automate this yet

\begin{tabular*}{0.80\textwidth}{l@{\extracolsep{\fill}}r}
\toprule
Net Exports of Goods and Services       & $  -508.3$ \\
Net Labor Income from ROW               & $    -9.4$ \\
Net Capital Income from ROW             & $   247.4$ \\
Unilateral Current Transfers from ROW   & $  -119.2$ \\
\textbf{Current and Capital Account}    & $\mathbf{  -389.6}$ \\
\addlinespace
Net direct investment in US             & $  -225.4$ \\
Net Purchase of Securities              & $    167.0$ \\
Net loans and other                     & $   243.7$ \\
Financial derivatives                   & $54.4$\\
\textbf{Financial Account (inflows)}     & $\mathbf{   239.6}$ \\
\addlinespace
\textbf{Statistical Discrepancy}     & $\mathbf{    149.9}$ \\
\bottomrule
\addlinespace
\end{tabular*}

\begin{minipage}{0.80\textwidth}
\footnotesize{The numbers are for 2013, billions of US dollars.
ROW means ``rest of world.''
There are modest differences between these balance of payments measures and quarterly NIPA measures.}
\end{minipage}
\label{tab:usbop}
\end{table}

Two closely-related measures are commonly reported.  The {\it
merchandise trade balance\/} is similar to net exports
, but includes only trade in goods (``merchandise'').  It is reported monthly, and
so is more readily available than the quarterly NIPA data, which are broader.
Service trade includes such things as foreign tourists visiting the US (hotels,
restaurants), consulting services provided by US firms for foreign
clients, and foreign students attending US universities. Since the
US currently runs a modest surplus in service trade, the merchandise
trade deficit (slightly) overstates the deficit for trade in goods
and services (net exports).
The {\it current account\index{current account|textbf} balance\/} is a broader concept than net exports;
it consists of net exports plus net receipts of capital income, labor
income, taxes, and transfers from abroad
({\it net foreign income\/} for short).
Mathematically,
\[
    \CA \;=\; \NX + \mbox{Net Foreign Income} ,
\]
where $\CA$ is the current account and $\NX$ is (still) net exports.

Net foreign income includes such items as payment of interest
on US government bonds\index{bond} owned by foreign central banks \index{central bank} (a negative entry),
salaries received by American consultants working in Tokyo
(a positive entry),
and salaries paid to Russian hockey players in the US
(a negative entry).
We see in Table~\ref{tab:usbop} that 
the US was a net recipient of capital income and a net payer
of labor income.


The current account balance is, thus, the broadest measure of a country's flow
of ``current'' payments to and from the rest of the world. In the
US, the difference between net exports and the current
account usually is modest. In other countries, the flows
of labor and capital income may play larger roles.

%In many cases, there's little difference between the trade balance
%and the current account.
%We refer to them generally as {\it external balances\/}.


\section{Trade in assets}

There are also flows related to capital and financial
transactions.
You can see in Table~\ref{tab:usbop} that the US in 2012 was the net recipient of
\$446.3b of capital and financial ``inflows,'' meaning that
foreigners' purchases of US assets were greater than US nationals'
purchases of foreign assets by this amount.
By convention, this is reported as a
positive entry, even though it corresponds to an accumulation of
liabilities with respect to the rest of the world.  Foreigners' purchases
of domestic assets consisted of direct investment (a controlling
interest in a US business); purchases of equity and bonds\index{bond} issued by
US corporations; purchases of US government and agency issues; loans to US borrowers; and
some other minor items we won't bother to enumerate.

The central insight we gain from the balance of payments
is that these asset transactions must match the current transactions:
\[
    \mbox{Current Account} + \mbox{Capital and Financial Account} \;=\; 0.
\]
It's not quite true in the data, because the numbers are not
entirely accurate.
We add a balancing item
(``statistical discrepancy'' or ``errors and omissions'') to make up the
difference. The point is that any deficit in the current account\index{current account}
must be financed by a capital inflow:
selling assets or accumulating liabilities with
respect to the rest of the world.  The same accounting
 truism applies to a firm or an individual.  If your expenditures exceed
your receipts, you need to sell assets or borrow to finance the
difference. Firms do this regularly when they make major additions
to plant and equipment. And households often do the same when they
buy houses.

The interesting thing about this accounting
 identity is that it gives us a different perspective on current account deficits.  If we
run a current account\index{current account} deficit as a reflection of a trade deficit, as
in the US right now, we're tempted to look at imports and exports as
the reason. Perhaps foreign countries are keeping our goods out of
their home markets, or pushing down their exchange rates to encourage
exports.  That's the first reaction most people have.  But now we
know that a current account deficit must correspond to a capital and
financial inflow: Foreign investors are buying our assets.  This
perspective leads us to think about the investment opportunities in
the US and elsewhere in the world that might lead to this. Are US
assets particularly attractive? Or are foreign assets unattractive?
Both perspectives are right, in the sense that they're true as a
matter of accounting  arithmetic, but the second one captures more
clearly the dynamic aspect of decisions to invest.


\section{Net foreign assets}

The capital and financial account measures net flows of financial
claims: changes in asset position, in other words. The balance-sheet
position of an economy is referred to as its net international
investment position (NIIP) or, simply, net foreign assets (NFA).
If a country's claims on the rest of the world exceed
their claims on it,
then it has positive net foreign assets and is said to be a net
creditor. If negative, a net debtor. The position changes over time,
as indicated by the capital and financial account.
Mathematically, we would say
%
\begin{eqnarray}
        \mbox{\it NFA\/}_{t}  &=&
        \mbox{\em NFA\/}_{t-1}  +   \mbox{\em NX\/}_t %\nonumber \\
              \nonumber \\
        &&   + \; \mbox{Net Foreign Income} + \mbox{Asset Revaluations} .
        \label{eq:niip}
\end{eqnarray}
As in most accounting
 frameworks, there's a connection
between the income statement (the ``flows'' in economics parlance)
and the balance sheet (the ``stocks'').

An analogous relation for an individual might go something like this:
Suppose you
start with no assets or liabilities and then borrow 50,000 for the
first year of your MBA.
You spend the entire 50,000 and have no other source of funds,
so you have a cash-flow deficit of $-50,000$ for the year.
At the end of the year, you have a net asset position of
$-50,000$.
The bookkeeping is analogous to equation (\ref{eq:niip}),
with $\NFA$ analogous to your net worth,
$\NX$ analogous to your annual cash-flow surplus or deficit,
and the last two terms ignored to keep things manageable.
If we added interest on the debt, that would show up in
Net Foreign Income.


Why do we need asset revaluations?  By tradition, we
measure international investments at market value, so if the value
of an asset changes, we need to account for it in NFA.
In international investments, asset revaluations occur both through the usual change in
prices of equity and bonds\index{bond} and through changes in exchange rates
for instruments denominated in foreign currencies.


\begin{table}
\centering
\caption{US net international investment position.}
\begin{tabular*}{0.8\textwidth}{l@{\extracolsep{\fill}}r}
\toprule
{\bf US-owned assets abroad}            &  23,709.8 \\
\hspace{5mm}Direct investment           &  7,080.1 \\
\hspace{5mm}Corporate equity            &  6,444.2 \\
\hspace{5mm}Bonds\index{bond}           &  2,738.8 \\
\hspace{5mm}Loans and other             &  4,178.6 \\
\hspace{5mm}Reserves \& govt            &  448.3 \\
\hspace{5mm}Financial Derivatives       & 2,819.8 \\
\addlinespace
{\bf Foreign-owned assets in the US}    & 29,092.8 \\
\hspace{5mm}Direct investment           & 5,790.6 \\
\hspace{5mm}Corporate equity            & 5,821.5  \\
\hspace{5mm}Corporate bonds\index{bond} &  3,886.8 \\
\hspace{5mm}US govt (treasuries, currency, official) &  5,794.9  \\
\hspace{5mm}Loans and other             &  5,052.8      \\
\hspace{5mm}Financial Derivatives       & 2,746.3 \\
\addlinespace
{\bf Net international investment position} &  $-$5,383.0  \\
\bottomrule
\end{tabular*}
\begin{minipage}{0.8\textwidth}
\footnotesize{%
\smallskip
The numbers are for 2013 yearend, billions of US dollars.}
\end{minipage}
\label{tab:usniip}
\end{table}

We report recent numbers for the US in Table~\ref{tab:usniip}.
There we see that the US has a net financial asset position of
--\$5,383.0b, meaning that foreign claims on the US exceed US claims
on the rest of the world by this amount.
The table gives a complete accounting of these positions.\index{current account}


\section{Sources of external deficits}
We'll talk more about the difference between the trade balance and
the current account shortly,
but for now, let's ignore the difference and consider a trade deficit.
If we have a large deficit, should we be worried?
Is it a sign that the economy is in trouble?
In this and many other cases, it's helpful to consider an analogous situation
for a firm.
Suppose that a firm is accumulating liabilities.
Is that a bad sign?
The answer is that it depends what the liabilities are used to finance.
If they finance productive investments,
then there should be no difficulty servicing the liabilities.
In fact, the ability to finance them suggests that someone
thinks the investments will pay off.
But if the money is wasted (surely you can think of examples!),
then investors might be concerned.
The same is true of countries --- it depends where the funds go.

Consider the flow identity that we saw in Chapter \ref{chp:macd}:
\[
    S  \;=\; Y - C - G \;=\; I + \NX.
\]
Typically, this is expressed as a ratio to GDP,
with everything measured at current prices:
\[
    \frac{S}{Y}  \;=\; \left( 1 - \frac{C+G}{Y} \right)
        \;=\; \frac{I}{Y} + \frac{\NX}{Y}.
\]
If we run a trade deficit ($\NX < 0$),
it must (as a matter of accounting )
reflect some combination of low saving
and high investment (high $I$).
If we borrow from abroad to
finance new plant and equipment, and the plant and equipment lead to
higher output, we can use the extra output to cover the liabilities.
If the investment is ill-considered, then we face the same issue
as a firm in a similar situation.

What if we finance household consumption
or government purchases?
We have to answer the same question:  Was the
expenditure worthwhile?  Here there is room for concern,
but a serious answer would depend on the nature of the expenditures.

The Lawson Doctrine, named after British government official Nigel Lawson,
makes a distinction between public and private sources of deficits.
Recall that we can divide saving into private  and government components,
so that
\begin{eqnarray*}
    S_p + S_g  &=& I + \NX.
\end{eqnarray*}
In Lawson's view, a trade (or current account)\index{current account} deficit that financed
a difference between private saving and investment
is fine.
But if the external deficit (trade or current account)
stems from a government deficit,
it's worth a more careful look.
In practice, emerging market crises often stem from government
deficits that are financed abroad.


\section{Debt dynamics and sustainability \index{government debt!sustainability}}

\begin{comment}
There's a natural source of dynamics in net foreign assets,
just as there was with government debt.
Since assets accrue interest, NFA tends to grow over time unless
something is done to counteract it.
In an international context, NFA plays the role of debt.
This is something of a misnomer
because in many countries, the claims of countries on others
include equity and direct investment as well as debt.
File that away for later.
We generally look at NFA relative to GDP, so the question
is which is growing more rapidly.
If the current situation leads the ratio of NFA to GDP to explode,
we say the situation is not sustainable.
\end{comment}

The net foreign asset position evolves through time, just as
government debt does.
As with government debt, the focus is traditionally on
the ratio to GDP, which can change through either the numerator or denominator.
We've seen that NFA changes like this:
\begin{eqnarray*}
    \mbox{\it NFA\/}_{t} &=& \mbox{\it NFA\/}_{t-1} + \mbox{\it NX\/}_{t} +
                \mbox{Net Foreign Interest Income}  \\
                        &=& (1+i_t) \mbox{\it NFA\/}_{t-1} + \mbox{\it NX\/}_t  .
\end{eqnarray*}
Note that everything here is nominal, including the
interest rate $i_t$ on the net foreign asset position.
Here, we're skipping asset revaluations and
the non-interest component of net foreign income,
but we could add them back in later if we thought they were relevant.
If the growth rate of nominal GDP is $g_t + \pi_t$, we can write
\begin{eqnarray*}
    Y_{t} &=& (1+g_t+\pi_t) Y_{t-1} .
\end{eqnarray*}
With these inputs, we see that $\NX/Y$ evolves like this:
\begin{equation}
    \frac{\NFA_{t}}{Y_{t}} \;\approx\;
                \frac{\NFA_{t-1}}{Y_{t-1}}
                +(i_t-\pi_t)  \frac{\NFA_{t-1}}{Y_{t-1}}
                -g_t  \frac{\NFA_{t-1}}{Y_{t-1}}
             +    \frac{\NX_{t}}{Y_{t}} .
\end{equation}
The logic is identical to our analysis of government debt
in equation (\ref{eq:debtdynamics}).


How does the ratio of NFA to GDP change over time?
The first issue is what real interest rate $i_t-\pi_t$
we pay on our borrowing.
Typically the rate is positive, which tends to increase a positive
net foreign asset and decrease a negative one.
The second issue of real GDP growth $g_t$.
High growth reduces the ratio of net foreign assets to GDP by increasing
the denominator.
Finally, a trade surplus or deficit carries over directly to the net foreign asset position.

If a country has a large current account deficit and
a large and growing net foreign liability position,
it's sometimes said to be {\it unsustainable\/}.
\index{government debt!unsustainable}
But if it's unsustainable, what happens?
The theory doesn't say, but we can imagine some possibilities:
The trade deficit turns to surplus; the country defaults on
some or all of its foreign liabilities; and so on.
More commonly, this is used to project the growth
of NFA over the next few years.
If this leads to a large ratio of NFA to GDP, then
investors may start to wonder whether they'll be repaid.
How large does it have to be to generate concern?
It depends on the country and its institutions
--- just as we learned in studying government debt in Chapter \ref{chp:dbdf}.

History tells us, however, that we see deficits and net liabilities in both
countries on the brink of trouble (Argentina in the late 1990s)
and countries that are performing well (Australia over most of its history).
Most analysts would check further and find out what the deficit
was financing (plant and equipment or government spending)
and how it was structured (debt or equity).
If debt, then the maturity and denomination are also relevant.


\section{Big picture}

The bottom line is that the current account\index{current account} deficit
and net foreign asset position are important indicators
of the state of an economy.
Important, yes, but it's not always clear what to make of them.
Take a current account deficit.
Is a deficit is bad (it sounds bad!)
or good (look, people want to invest in our country!)?
We need to look at the overall picture and come up with a judgment.
It's another piece of the puzzle to consider when deciding
whether a country is a good opportunity.


\section*{Executive summary}

\begin{enumerate}
\item There are several measures of current transactions with other countries:
\begin{itemize}
\item The merchandise trade balance measures exports minus imports of goods. \item Net exports
includes trade in services, as well. \item The current account includes net international factor
income, taxes, and transfers.\index{current account}
\end{itemize}
We refer to them collectively as ``external'' balances (deficits or surpluses).

\item The current account is mirrored by an equal and opposite capital and financial account
measuring net asset transactions.

\item The net international investment position measures our current net claims on the rest of the world.

\item The flow identity tells us that the external deficit
reflects some combination of personal saving, government saving,
and investment.

\end{enumerate}


\begin{comment}
\section*{Review questions}

\begin{enumerate}

\item What if we write off debt?

\item Why no interest payments in (\ref{eq:niip})?

\end{enumerate}
\end{comment}

\section*{If you're looking for more}

For more information:
%
\begin{itemize}
\item In the US, international transactions are reported along with the National Income and
Product Accounts by the Bureau of Economic Analysis.  See their
\href{http://www.bea.gov/bea/di1.htm}{International Economic Accounts}.

\item The International Monetary Fund's
\href{http://ifs.apdi.net/imf/ifsbrowser.aspx?branch=ROOT}{International Financial Statistics} is
the best single source of balance of payments and international investment data.

\item International standards for BOP data are set by a working committee of the International
Monetary Fund.  Their \href{http://www.imf.org/external/np/sta/bop/bop.htm}{web site} includes
discussions of both conceptual and measurement issues. The
\href{http://www.imf.org/external/bopage/arindex.htm}{annual reports} are a good overview.  One of
the recent highlights:  In 2007, the world trade balance was \$108b, meaning that countries
reported \$108b more exports than imports.  Since every export must be someone else's import, this
can't really be true, but it points to some of the difficult measurement issues faced by the
people putting these accounts together.

\end{itemize}


\section*{Symbols used in this chapter}
\begin{table}[H]
\centering
\caption{Chapter \ref{chp:bop} symbol table.}
\begin{tabular*}{0.95\textwidth}{l@{\extracolsep{\fill}}l}
\toprule
Symbol & Definition\\
\midrule
$\CA$    &Current Account\\
$\NX$    &Net exports\\
$\NFA$    &Net foreign assets\\
$S$        &Saving\\
$Y$        &Gross domestic product (= Expenditure = Income)\\
$C$           &Private consumption\\
$I$            &Private investment \\ %(including residential and business investment)\\
$G$           &Government purchases of goods and services (not transfers)\\
$S_p$    &Private saving $(=Y-T-C)$\\
$S_g$    &Government saving $(=T-G)$\\
$i$        &Interest rate on net foreign assets\\
$g$        &Discrete compound growth rate of GDP\\
\bottomrule
\addlinespace
\end{tabular*}
\begin{minipage}{0.95\textwidth}
\footnotesize{In this chapter, we have dealt only with \textit{nominal} variables.}
\end{minipage}
\end{table}


\section*{Data used in this chapter}
\begin{table}[htb]
\centering
\caption{Chapter 18 data table.}
\begin{tabular*}{0.9\textwidth}{l@{\extracolsep{\fill}}l}
\toprule
Variable & Source\\
\midrule
Current Account (BOP)    &BOPBCA\\
Current Account (NIPA)    &NETFI\\
Net exports of goods and services (NIPA)    &NETEXP\\
Nominal GDP    & GDP\\
Foreign-owned assets in US (+ equals increase)    &BOPI\\
U.S.-owned assets abroad (+ equals decrease)    &BOPOA\\
Income payments (total)    &BOPMIT\\
Income payments on foreign assets in US    &BOPMIA\\
Income receipts (total)    &BOPXRT\\
Income receipts on assets abroad    &BOPXR\\
Statistical discrepancy (BOP)    &BOPERR\\
\bottomrule
\addlinespace
\end{tabular*}
\begin{minipage}{0.9\textwidth}
\footnotesize{To retrieve the data online, add the identifier from the source column to \url{http://research.stlouisfed.org/fred2/series/}.  For example, to retrieve the GDP, point your browser to \url{http://research.stlouisfed.org/fred2/series/GDP}}
\end{minipage}
\end{table}
