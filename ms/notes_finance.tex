\chapter{Financial Markets}\label{chp:fnmk}
\hypertarget{finance}{}

%No extra line here.
\textbf{Key Words:} Time consistency; information asymmetry.

\textbf{Big Ideas:}
%\vspace{-0.1in}
\begin{itemize}
    \item Effective financial markets require strong institutional support.
    \item Good institutions deal with information asymmetries and time consistency issues.
\end{itemize}

\rule{\textwidth}{1pt}

Some of the most important markets for aggregate economic performance
are those for labor and (financial) capital,
which affect every industry and product.
Countries differ markedly in their treatment of both markets,
with (evidently) different outcomes as a result.

Our focus here is on financial markets,
which are, perhaps, the most difficult markets to manage effectively.


\section{Features of effective financial markets \index{financial markets}}

Financial markets are central to economic performance,
because they facilitate (if they work well)
the allocation of resources to the most-productive firms.
In the US today,
some firms borrow from banks,
others issue bonds \index{bond} and equity in capital markets,
and still others raise money through venture capitalists.
Countries differ widely in how they do this,
but they all have ways of channeling funds from
households (savers) to firms (borrowers).


The primary issue with financial markets is information,
and we know that markets sometimes handle information poorly.
Here, investors need to understand the risks faced by borrowers,
but borrowers typically know more about themselves than others do.
A bank, for example, needs to know enough about its borrowers to
assess the risk of default, and its depositors need to know enough about the bank's ability to
do this well to assess the risk to their deposits.\index{credit risk!default risk}
A bank (or other financial institution) is, therefore, in the information business:
Its goal is to process information efficiently so that it
can assess and manage risk.

None of this is easy to do.
All of these financial arrangements
require institutional support.
An effective financial system requires some version of:
%
\begin{itemize}
\item\textbf{Creditor protection. \index{creditor protection}}
If A lends money to B, it's essential that A's claim be honored.
That requires a legal system that makes
the creditor's rights clear and enforces them if necessary.
(You might think about ``property rights'' and the ``rule of law''
about now.)
Without this, people will either not make loans or will make loans only to friends and relatives.
Weak, ineffective financial systems follow naturally
when creditors are not protected,
and economic performance suffers as a result.

\item \textbf{Corporate governance. \index{corporate governance}}
The laws of most countries give creditors some
say over the management of firms.
Equity investors, for example,
are represented (in principle) by boards of directors.
There's endless debate about how best to do this,
but there's no question that doing it well is important.

\item \textbf{Disclosure. \index{disclosure}}
When people invest in securities, they need to understand what
they're buying.
In most countries with active securities markets,
the law dictates disclosure of relevant financial information.
Again, some countries do this better than others.

\item \textbf{Central banks. \index{central bank|textbf}}
Most countries have central banks. If run well, they play an important role in the economy,
particularly as lenders of last resort during financial crises.
\end{itemize}

Measures of these things are available from a number
of sources,
including the World Bank's
\href{http://www.doingbusiness.org/}{Doing Business}
website,
particularly the categories Getting Credit and Protecting Investors.


\section{Financial regulation \index{financial regulation} and crises}

From the perspective of a country, one of
the challenges of managing financial markets is that they
can cause enormous collateral damage if something goes wrong.
If a farmer goes bankrupt, you buy milk from someone else.
But if a large financial institution goes under,
it can slow down the whole economy.
The question is how to manage financial markets to get the benefits
of a thriving financial system with the least risk.

No one yet has come up with a perfect answer.
An unregulated financial system may work well most of the time,
but will experience occasional crises.
A more tightly regulated system may (it's not a sure thing)
have lower crisis risk, but the regulation may distort
the allocation of capital.
Most approaches to financial regulation face a tradeoff of this sort.

Consider deposit insurance. \index{deposit insurance}
In the US, bank panics were a common occurrence
up through the 1930s.
During the Depression,
thousands of banks went under.
Some of them were insolvent.
Others closed because depositors demanded their money back
for fear that the bank would go under:
what we call bank runs.
It's a consequence, in part, of people having imperfect
information about the bank's soundness.

The solution --- or, rather, one solution --- was to provide deposit insurance.
Milton Friedman \index{Friedman, Milton}
 and Anna Schwartz called federal
deposit insurance ``the most important structural change''
made in the 1930s to deal with bank runs.
And it worked --- bank runs pretty much ended.

But like many solutions, it raised new problems.
The problem with deposit insurance is what economists call ``moral hazard'' \index{moral hazard}
and others might call the ``other people's money'' problem.
Since depositors don't face the risk of losing their money,
banks don't face the risk of withdrawal,
and they have less reason to control the risk of their investments.
Or to put it differently, their borrowing costs don't reflect the risk
of their loan portfolios.
So they take excessive risk, which is hardly what we're looking for.
Therefore, we add to deposit insurance some regulatory oversight intended
to limit banks' ability to take risks.
We know from bitter experience that it's hard to get this right,
and we're still trying.

A related challenge is the ``too big to fail" \index{too big to fail}
dilemma, a classic version
of the \hyperref[sec:time_cons]{time-consistency problem} discussed in Chapter \ref{chp:insp}.\index{time consistency}
Policymakers insist that they
will never bail out failing banks, but everyone
knows in advance that a failed behemoth can topple the financial system
(think Lehman or AIG).
So the promise lacks credibility: a future policymaker is likely to bail them out anyway.
Investors know this, and reward the largest intermediaries with low funding costs,
thereby subsidizing excessive risk taking.

\needspace{4\baselineskip}
\section*{Executive summary}

\begin{enumerate}
\item Financial markets work best when based on effective institutions.
\item It's hard to get that exactly right.
\end{enumerate}


\begin{comment}
\section*{Review questions}

\begin{enumerate}

\item ...
\end{enumerate}
\end{comment}

\section*{If you're looking for more}

The logic and operation of financial institutions is a huge subject
in its own right.
Among the courses we have on the topic are Professor Schoenholtz's
``Money and Banking,'' course ECON-GB.2333,
and ``Financial Crisis and Policy,'' course ECON-GB.2343.
Or see his book:
Stephen Cecchetti and Kermit Schoenholtz,
\href{http://www.amazon.com/Banking-Financial-Markets-Stephen-Cecchetti/dp/007337590X/}
{\it Money, Banking and Financial Markets\/}.
Ben Bernanke's \index{Bernanke, Ben} testimony to Congress (search ``Bernanke \index{Bernanke, Ben}
 testimony'')
is a wonderful overview of financial regulation and the 2008 crisis.
He also did a series of lectures that are posted on the Fed's website.

Beyond that, financial crises make good reading, and there's no shortage
of good books on the subject.
One of the best reads is Edward Chancellor's {\it Devil Take the Hindmost\/},
a history of financial speculation.
On the most recent crisis,
we enjoyed David Wessel's
\href{http://www.amazon.com/FED-We-Trust-Bernankes-Great/dp/0307459691/}
{\it In Fed We Trust\/}
and Andrew Ross Sorkin's
\href{http://www.amazon.com/Too-Big-Fail-Washington-FinancialSystem/dp/0143120271/}
{\it Too Big to Fail\/}.
