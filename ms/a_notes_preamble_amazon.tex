\documentclass[8pt,twoside,pdftex]{book}

% nice book overview: http://www.math.mun.ca/~edgar/thesis.html
% see esp indexing w/ package makeidx
% more at:  http://en.wikibooks.org/wiki/LaTeX/Indexing

\usepackage{amsmath, amssymb, amsthm}
\usepackage{graphicx}
\usepackage{color}
\usepackage{comment}
\usepackage{layout}
\usepackage{booktabs}

% list spacing
\usepackage{enumitem}
\setitemize{leftmargin=*, topsep=-3pt, partopsep=0pt}
\setenumerate{leftmargin=*, topsep=0pt}

\usepackage{caption}
\usepackage{setspace}
\usepackage{float}

% for index
\usepackage{makeidx}
\usepackage[indentunit=2.0em, justific=RaggedRight, columnsep=2.5em]{idxlayout}

%Turn off ligatures to make pdf searchable, turn on ligatures to make font pretty
%\usepackage{microtype}
%\DisableLigatures{encoding = *, family = * }

% link colors etc:  black for Amazon, blue for the online version
\usepackage[colorlinks=true,
            allcolors=blue,
%            linkcolor=black,
%            citecolor=black,
%            urlcolor=blue,
            bookmarks=false,
            pdfstartview={FitV},
            pdftitle={The Global Economy},
            pdfsubject={A course in macroeconomics for business students},
            pdfauthor={The NYU Stern Economics Team}
            ]{hyperref}

% Adjust fonts for chapter and section headings
% old version
\usepackage[medium, compact]{titlesec}
\titleformat{\section}{\large\bfseries}{\thesection}{1em}{}
\titleformat{\subsection}{\normalsize\bfseries}{\thesection}{1em}{}

% LC  version
%\usepackage[medium, compact, sf, bf]{titlesec} % edited by LC
%\titleformat{\section}{\large\bfseries\sffamily}{\thesection}{1em}{}  % edited by LC
%\titleformat{\subsection}{\normalsize\bfseries\sffamily}{\thesection}{1em}{}  % edited by LC
%\titleformat{\part}{\bfseries\huge\sffamily\uppercase}{}{0pt}{Part \thepart\\}  % edited by LC
%\titleformat{\chapter}{\bfseries\huge\sffamily\uppercase}{}{0pt}
%        {\ifnum\thechapter>0\resizebox{!}{30pt}{\thechapter}\\ \fi}  % edited by LC

% to kill off awkward pagebreaks
\usepackage{needspace}
% example:  \needspace{4\baselineskip} makes sure we have four lines available before pagebreak

% for left spacing of problems, summary, etc
% no longer used
\newlength{\oldleftmargini}
\setlength{\oldleftmargini}{\leftmargini}

% for spacing of table of contents
% see http://tex.stackexchange.com/questions/49877/table-of-contents-spacing
% problem:  messes up spacing and fonts of table and figure lists in toc, so killed them off
\usepackage{tocloft}
\renewcommand\cftchapafterpnum{\vskip10pt}

%%%%%%%%%%%%%%%%%%%%%%Margins%%%%%%%%%%%%%%%%%%%%%%%%%%%%%%%%%%%%%%%%%%%%%%%
% http://en.wikibooks.org/wiki/LaTeX/Page_Layout
\usepackage{geometry}
\geometry{paperwidth=6in, paperheight=9in, outer=0.6in, inner=0.85in, bottom=0.6in}
%\usepackage[cam,letter,center,pdftex]{crop}

% indentation and para skips
\setlength{\parindent}{0in}
\setlength{\parskip}{\bigskipamount}  % this is diddled further in the front_matter
\raggedbottom

%%%%%%%%%%%%%%%%%%%%Tighten up the lists%%%%%%%%%%%%%%%%%%%%%%%%%%%%%%%%%%
\let\OLDdescription\description
\renewcommand\description{\OLDdescription\setlength{\itemsep}{-2mm}}

%%%%%%%%%%%%%%%%%%%%%%Headers and Footers%%%%%%%%%%%%%%%%%%%%%%%%%%%%%%%%%%
\usepackage{fancyhdr}
\pagestyle{fancy}
\fancyhead{}
\fancyfoot{}
\renewcommand{\chaptermark}[1]{\markboth{\thechapter.\ #1}{}}
\fancyhead[RO]{\leftmark \hspace{0.25in} \thepage}
\fancyhead[RE]{}
\fancyhead[LE]{\thepage \hfill Global Economy @ NYU Stern}

%%%%%%%%%%%%%%%%%%%%%%Special Characters%%%%%%%%%%%%%%%%%%%%%%%%%%%%%%%%%%%%
\newcommand{\GDP}{\mbox{\em GDP\/}}
\newcommand{\NDP}{\mbox{\em NDP\/}}
\newcommand{\GNP}{\mbox{\em GNP\/}}
\newcommand{\NX}{\mbox{\em NX\/}}
\newcommand{\NY}{\mbox{\em NY\/}}
\newcommand{\CA}{\mbox{\em CA\/}}
\newcommand{\NFA}{\mbox{\em NFA\/}}

\newcommand{\Def}{\mbox{\em Def\/}}
\newcommand{\DS}{\mbox{\em DS\/}}
\newcommand{\CPI}{\mbox{\em CPI\/}}
\newcommand{\POP}{\mbox{\em POP\/}}
\newcommand{\CU}{\mbox{\em CU\/}}
\newcommand{\RE}{\mbox{\em RE\/}}
\newcommand{\MB}{\mbox{\em MB\/}}
\newcommand{\MONE}{\mbox{\em M1\/}}
\newcommand{\MTWO}{\mbox{\em M2\/}}
\newcommand{\IP}{\mbox{\em IP\/}}
\newcommand{\RER}{\mbox{\em RER\/}}

%%%%%%%%%%%%%%%%%%%%%%%%%%%%Load Indexing Commands%%%%%%%%%%%%%%%%%%%%%%%%%%%%%%%
\makeindex

