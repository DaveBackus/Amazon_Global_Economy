\chapter{Macroeconomic Crises}\label{chp:cris}
\hypertarget{crises}{}

\input{\blurbpath Blurbs/blurbs_crises}
\rule{\textwidth}{1pt}

Economies periodically experience {\it crises\/}:
economic downturns that are not only larger than typical recessions
but qualitatively different.
The idea is now fresh in our minds,
but similar episodes have occurred throughout recorded history.
They're less common in modern, developed countries,
but they can happen anywhere.
Like snowflakes and business cycles, no two are exactly the same,
but they share some common features.


\section{Classic crisis triggers }

There are three classic triggers of macroeconomic crises:
sovereign debt, financial fragility, and a fixed exchange rate.\index{exchange rate regime!fixed exchange rate}

\textbf{Sovereign debt problems.\index{government debt!sovereign debt}}
If investors fear that a government may not repay its debt,
the market for debt collapses, often taking the economy with it.
In the old days, wars were the standard problem.
Wars are expensive, and if investors thought the expense was more
than the government was willing or able to bear,
they would stop buying the debt.
In modern times, governments spend money on many things besides wars,
but the possibility of default remains.
Argentina in 2002 and Greece today are recent examples.
These experiences remind us that sovereign debt need not be risk-free.

The central issue with government debts is sovereignty.
If a corporation defaults, the creditors take it to court
and claim the assets.
With governments, there's no such mechanism, and the process
is sloppier as a result.

\textbf{Financial fragility.}
We know from centuries of experience that when the
financial system freezes up,
economic activity slows down sharply.
We saw that in 2008,
but the same thing happened during the Bank of England Panic of 1825,
the Baring Crisis of 1890,
the US Panic of 1907,
Japan and Scandinavia in the 1990s,
and many other occasions.
It's a feature of even advanced financial systems that they
sometimes break.

In most cases, these financial problems follow from
poor investments (real estate is a common example),
which put the solvency of financial institutions in question.
The problems tend to snowball: Worries about the viability of one firm may lead others to
reduce their lending, leading to a cycle of retrenchment
that puts even sound firms in trouble.
The word ``panic'' is apt here and stems
from the imperfect information that investors have
when deciding where to put their money.


\textbf{Fixed exchange rates.} For whatever reason, fixed exchange rates periodically
must be defended in ways that undermine the economy.
Recent examples include the UK in 1992,
Mexico in 1994,
Korea in 1997,
and Argentina in 2000.

Many crises combine several of these elements.
The countries of the euro area face all three.
In Ireland and Spain, bailing out their banks
landed the governments in financial peril.
In other countries, bank positions in government debt
or in overpriced real estate put the banking systems in peril.
Finally, the common currency eliminates exchange rate changes
as a possible correction mechanism.


\section{Crisis indicators: the checklist\index{crisis!crisis indicators} }

Crises are inherently hard to predict.
Why?
Think about cardiologists.
We understand that they can identify risk factors
(weight, high blood pressure) but cannot predict
the date of a heart attack with any precision.
Crises are worse.
Once people see a crisis on the way,
their actions tend to reinforce it:
they sell government debt,
withdraw funds from banks,
or shift their money to foreign currency.
But like a cardiologist, we can use what we know
to identify signs of trouble.


Analysts differ in the details, but most would include
the following in their ``checklist'' of crisis indicators:

%\begin{itemize}
\textbf{Government debt and deficits.}
The primary issue here is the quality of governance.
That aside, common rules of thumb include:
Worry if government
deficit is more than 5 percent of GDP or debt is more than 50 percent of GDP.
Adjust upward for developed countries, downward for developing countries
and for regional governments.
And watch out for hidden liabilities:  pensions, health care, bailouts, etc.
These are often much larger than official liabilities.

Fine points:  Worry further if debt is short-term and/or denominated in foreign currency.
Short-term debt subjects the government to refinancing (``rollover'') risk;
markets may demand better terms or refuse to refinance.
Foreign-denominated debt subjects government to risk
if currency falls in value, making the debt larger in local terms.

\textbf{Banking/financial system.}
This isn't something we've discussed,
but  analysts track leverage, duration mismatch, exposure concentration,
risk-management processes,
and nonperforming loans.
The challenge is measuring them accurately from reported information.
Some of the most troubling situations come with low-quality data.


\textbf{Exchange rate and reserves.}
Rule of thumb:  Worry if the exchange rate is fixed, or close to it,
and the currency is significantly overvalued in PPP terms
(Big Macs cost 30 percent more than in other currencies;
the real exchange rate has risen more than 30 percent in the past 2-5 years).
Worry more if foreign-exchange reserves
are low or have fallen significantly.

\textbf{Political situation.}
%Rule of thumb:  Worry if the political system seems unable or unwilling
%to deal with problems that could lead to crises.
Crises are often more political than economic.
Countries with effective governments suffer fewer crises and
deal with those that occur more effectively.
Analysts therefore look for signs that the political
system is unable or unwilling to deal with problems that might turn into crises.
Zimbabwe's hyperinflation and Argentina's debt crisis are good examples.
Or the Weimar Republic in 1920s Germany.
%\end{itemize}

All of these things generate more concern
in countries with weak institutions.
It's not an accident that Greece is in worse trouble
than France or Germany.
%Analysts often find that local politics are more important
%than the numbers to the outcome.


\section{Crisis responses\index{crisis!crisis responses}}

What should a government do when faced with a crisis?
It depends on the trigger.
Standard advice includes:

\textbf{Sovereign debt crises.}
If the problem is that the government is borrowing too much,
the answer is to stop doing it --- run primary surpluses until the debt is manageable.
Default is also an option, and saves money in the short term,
but probably raises borrowing costs in the future.
And if you go through a default, it's helpful to resolve it as quickly as possible.

IMF support is often used to cushion the blow: Contingent on progress with the deficit,
the IMF lends the government money on more attractive terms
than the market would provide.
This ``conditionality,'' as it's called,
helps reduce moral hazard (you get the money only if you behave)
and provides cover for local politicians (the IMF made us do this).
Such conditional lending can be critical in a crisis,
when high borrowing rates exacerbate the government's debt problems.
(See:  debt dynamics.)


\textbf{Financial crises.}
If the financial system is fundamentally sound (solvent) but illiquid,
the longstanding advice is for the central bank \index{central bank} to lend aggressively.
The classic quote comes from Walter Bagehot,
a 19th-century businessman and journalist:
``To avert panic, central banks \index{central bank} should lend early and freely,
to solvent firms, against good collateral, and at high rates.''
%Put simply,
%the goals of crisis prevention and crisis management
%are often at odds.


If the financial system is insolvent,
it's important to get it recapitalized and operating again.
This advice comes with more than a little irony,
as governments sometimes find themselves bailing out
precisely those banks that triggered the crisis.
The trick is to do it in ways that inflict some pain
on the bank's management and creditors
(incentives for the future) and don't bankrupt the government.
These things happen fast, so it's hard to get everything right.


\textbf{Fixed exchange rates.} Let them float.
A more controversial approach is to impose capital controls   to inhibit the
response of capital markets to a possible drop in the exchange rate.
(See:  trilemma.)
capital controls   are a dangerous tool,\index{exchange rate regime!fixed exchange rate}
because the fear of future capital controls
can generate a crisis on its own,
as investors rush to get their money out of the country.
capital controls   on inflows may, for that reason, be more attractive
than controls on outflows.


\section*{Executive summary}

\setlength{\leftmargini}{.5\oldleftmargini}
\begin{enumerate}\itemsep=0.0in
\item The classic crisis triggers are
(i)~government debt and deficits,
(ii)~a fragile financial system,
and (iii)~a fixed exchange-rate system.

\item Crises are hard to predict,
but we nevertheless have useful indicators connected to each
of the triggers.

\item Politics and institutions are central.
\end{enumerate}
\setlength{\leftmargini}{\oldleftmargini}

%\end{document}

\section*{Review questions}

\setlength{\leftmargini}{.5\oldleftmargini}
\begin{enumerate}
\item Risk and opportunity in Ghana (May 2012).
You have been asked to prepare a risk assessment for
the West African country of Ghana.
Ghana is a former British colony that has been growing rapidly
in recent years after a period of unusually stable politics.
The Economist Intelligence Unit refers to it as a ``robust democracy.''
The World Economic Forum ranked Ghana 114th (of 133)
in their Global Competitiveness Report.
They continue:
``The country continues to display strong public institutions and
governance indicators,
particularly in regional comparison.''

The EIU's Country Risk Report adds:
\begin{itemize}
\item The December 2012 elections are expected to be close.
The president, John Atta Mills, came to power promising accountability
and transparency, but  has struggled to maintain party unity
while evidence emerges of financial impropriety of some government ministers.
\item The victor faces a challenging policy environment, particularly
the fiscal situation.
\item Expectations among the population are high as production
starts at the offshore Jubilee oil field.
\item The government's decision to allow use of 70\% of future
oil revenue as collateral for borrowing is a cause for concern
if the revenue is not managed properly.
\item The Bank of Ghana (the central bank \index{central bank}) faces the twin goals of
containing inflation\index{inflation} and fostering growth.
\item The currency  ---  the cedi --- floats with occasional heavy intervention.
\end{itemize}
%
Your mission is to assess the risks to Ghana  using
the information in the table, as well as
your own good judgement and analytical skills.

\begin{enumerate}
\item  You decide to start with a fiscal assessment.
What trend do you see in government revenues and expenses?

\item You notice that neither the primary deficit nor
interest expenses are reported separately.
How would you estimate them from the numbers in the table?
What are their values for 2011?

\item Using what you know about government debt dynamics,
compute the ratio of government debt to GDP for 2011.
What factors contribute the most to the change from 2010?

\item Overall, how would you assess the risks to Ghana's economy over
the next couple of years?
\end{enumerate}

{\small
\begin{tabular}{lrrrrr}
\toprule
        & 2008 & 2009 & 2010 & 2011 & 2012 \\
\midrule
GDP growth (\%) & 8.4 & 4.0 & 7.7 & 13.6 & 7.4 \\
Inflation (\%)  & 18.1 & 16.0 & 8.6 & 8.6 & 8.5 \\
Interest rate (\%) & 20.8 & 28.8 & 22.7 & 20.5 & 20.6  \\
Govt revenue (\% of GDP)  & 16.0 & 16.5 & 19.1 & 23.4 & 22.2 \\
Govt spending (\% of GDP) & 24.5 & 22.3 & 25.5 & 27.6 & 27.7 \\
Govt budget balance (\% of GDP) & --8.5 & --5.8 & --6.5& --4.2 & --5.5\\
Govt debt (\% of GDP) & 30.6 & 33.3 & 33.9 & {\bf } & {\bf } \\ %& 36.8 & 41.6\\
Real exchange rate (index) & 81.7 & 76.3 & 81.8 & 78.1 & 74.5\\
FX reserves (USD billions) & 1.8 & 2.9 & 4.3 & 4.4 & 4.8 \\
\bottomrule
\end{tabular}
}

\medskip
Data from EIU CountryData.
The government budget balance is a surplus if positive, deficit if negative.
The real exchange rate is the price of goods in Ghana relative to the rest
of the world;
the larger the number, the more expensive goods are in Ghana.
The numbers for 2011 and 2012 are estimates.

Answer.
\begin{enumerate}
\item Trends include:
(i)~revenues and spending both rising,
(ii)~spending still ahead of revenue (there's a deficit),
and (as a direct result)
(iii)~ratio of debt to GDP rising a little
(more on that to come).

\item This is a tricky one.
Remember that interest payments in year $t$ are $ i_t B_{t-1}/Y_t$.
We get what we want from:
\begin{eqnarray*}
    i_t B_{t-1}/ Y_{t} &=& i_t (B_{t-1}/ Y_{t-1}) (Y_{t-1}/ Y_{t}) \\
           &\approx&  i_t (B_{t-1}/Y_{t-1}) /(1+g_t + \pi_t) .
\end{eqnarray*}
That gives us interest payments in 2011 of 5.7\% of GDP and
a primary deficit of $-1.5$\% (that is, a surplus).

\item The key relation is this one:
\begin{eqnarray*}
    \Delta ({B_{t}}/{Y_{t}})
            &=&
                (i_t-\pi_t) ({B_{t-1}}/{Y_{t-1}})
                - g_t ({B_{t-1}}/{Y_{t-1}})
             +    ({D_{t}}/{Y_{t}})  .
\end{eqnarray*}
We refer to the components on the right as A, B, and C.
Doing the calculations gives us

\begin{center}
\begin{tabular}{lrr}
\toprule
        &  2010 & 2011   \\
\midrule
Interest payments  &  &  4.0  \\
Component A (interest)  &  &  4.0  \\
Component B (growth)            &   & --4.6  \\
Component C (primary deficit)   &   & --1.5  \\
Total change in $B/Y$       &       & --2.1   \\
Public debt (\% of GDP)     &  33.9 & 31.8   \\
\bottomrule
\end{tabular}
\end{center}

Over this period, the ratio of debt to GDP fell by 2.1\%.
The components contributed:
interest +4.0, growth --4.6, and the primary deficit --1.5.
Note especially the growth term, the result of unusually high GDP growth in 2011.

\item This is a call to look at the checklist of crisis indicators:
\begin{itemize}
\item Government debt and deficits.
We have deficits, but there's not
much sign yet of a growth debt to GDP ratio.
One future concern might be the possibility of borrowing now against future oil
revenue.  Will any debts incurred be spent wisely?
Will the oil revenue show up?
\item Banking system.  No information provided.
\item Exchange rate and reserves.  Reserves are modest,
but with the exchange rate floating there shouldn't be much
concern about that.
\item Politics.  Always an issue,
especially with a contentious election coming
and the promise of money from oil revenue.
It's an odd fact but a true one that revenue
from natural resources is more likely to cause problems than solve them.
\end{itemize}
\end{enumerate}
Update:  In August 2014, Ghana asked the IMF for help.
The chance of default remains low, since foreign debt is backed by  oil revenue,
but the promise of oil has turned into a curse,
as it often does.



% ==============================================================================
\item Don't Cry for Me Argentina.
(We know, it's a cliche, but so is their approach to policy.)
Argentina is a seemingly endless source of entertainment to economists,
yet its economy has done well in the recent past.
GDP growth fell to 0.9\% in 2009, during the global financial crisis,
but averaged over 9\% the next two years.
Most analysts attribute this success to
favorable commodity prices and strong global demand for Argentina's commodity exports.
Additional information is provided in Table \ref{tab:argentina}.

\begin{table}[h]
\centering
\tabcolsep=0.06in
\begin{tabular}{lrrrrr}
\toprule
                & 2010 & 2011 & 2012 & 2013 \\
\midrule
Official exchange rate (pesos per USD)  & 3.90 & 4.11 & 4.54 & 5.46  \\
Inflation (\%)              & 22.9 & 24.4 & 25.3 & 20.6 \\
Foreign currency reserves (USD billions) & 52.2 & 46.4 & 43.2 & 32.2 \\
Real GDP growth (\%)        & 9.2 & 8.9 & 1.9 & 5.2  \\
Govt revenue (\% of GDP)    & 24.3 & 23.6 & 25.4 & 27.3 \\
Govt spending (\% of GDP)   & 24.1 & 25.3 & 28.0 & 30.5  \\
Public sector surplus (\% of GDP) & 0.2 & --1.7 & --2.6 & --3.2 \\
Primary balance (\% of GDP) &  1.7 & 0.3 & --0.2 & --0.8   \\
Govt debt (yearend, \% of GDP)  & & & 44.8\\
Interest rate paid on debt (\%) & 4.0 & 5.5 & 6.7 & 6.5  \\
Money market interest rate (\%) & 9.1 & 10.0 & 9.8 & 12.7 \\
\bottomrule
\end{tabular}
\caption{Economic indicators for Argentina.  Source:  EIU.}
\label{tab:argentina}
\end{table}


At the same time, the government of President Cristina Fernandez de Kirchner
continues to adopt policies that befuddle outside observers, including:
taking over private pension funds,
restricting imports and purchases of foreign currency,
attacking the press,
nationalizing the Spanish-owned oil company YPF,
imposing price controls on electricity, natural gas, and public transportation,
and subsidizing energy consumption.

The Economist Intelligence Unit reports:
\begin{itemize}
\item A US court case may eventually leave
Argentina with the unpalatable choice of repaying the ``holdouts'' (creditors that
did not participate in the 2005 or 2010 restructurings) in full --- something that it
has sworn never to do --- or falling into default with its remaining creditors.

\item According to official data, consumer price inflation remains among the highest
in emerging markets, at 10.5\% in April 2013. However, the official data are
widely discredited and we are now using estimates produced by PriceStats,
which estimates that inflation in 2012 was 25\%.

\item Double-digit inflation has generated real peso appreciation.
Foreign-exchange controls have failed to prevent an erosion of
foreign exchange reserves,
heightening the risk of an eventual devaluation.

\item The Argentine peso floats in principle, but the central bank intervenes to limit
its rate of depreciation.
In addition, foreign currency transactions are subject to a variety of controls.
For the past couple of years, the government has
been gradually tightening the ``clamp,''
an unofficial policy of discouraging purchases of dollars.
As a result, the peso's official decline has been modest,
but the unofficial ``blue market'' price of the peso is considerably lower.

\item The poor banking sector risk rating reflects weak economic activity, expansionary
monetary policies that contribute to credit risk, high risk of exchange-rate
and interest-rate volatility, and increased currency convertibility risk.

\item The ruling party fared badly in the October midterm election,
 leaving the president without enough support in Congress
 to change the constitution and run for re-election.
 Focus will now shift rapidly to the 2015 presidential race.
 The president remains alienated from almost all of the country's most influential groups,
including the unions, the media, the Catholic Church and the traditional
leaders of the Peronist party. In this context, risks to political stability will be
high. An additional risk to stability is the president's health.
\end{itemize}
%
The question is what happens next:  Could another crisis be on the way,
or has Argentina put its problematic past to rest?
Use the information provided, 
and your own experience and good judgement,
to assess the risks to the Argentina economy over the next 2-3 years.
%
\begin{enumerate}
\item By ``real appreciation'' we mean an increase in the price
of local goods relative to foreign goods ---
what is sometimes called a decline in the real exchange rate.
Use the numbers in the table to demonstrate (or disprove) real appreciation
of the peso.

\item Why do you think the central bank's foreign exchange reserves have declined?

\item How do you see government debt evolving?
Compute, in particular, the ratio of government debt to GDP at year-end 2013.
What factors contribute the most to the change in the ratio?

\item Overall, how would you rate the risk of a macroeconomic crisis in Argentina?
What are the biggest sources of concern?
\end{enumerate}

\needspace{4\baselineskip}
Answer.
\begin{enumerate}
\item
The issue is the real exchange rate
$ \mbox{\em RER\/} = eP^*/P$, where $e$ is the exchange rate
(the peso price of one dollar),
$P$ is the price of Argentine goods,
and $P^*$ is the price of American goods.
So how is the real exchange rate changing?
In words:  the combination of high inflation and
more modest currency depreciation has made Argentine goods expensive
(equivalently, foreign goods cheap).

How would you show this?
Inflation is the rate of increase in $P$,
and we see the price of Argentine goods going up rapidly, roughly 20\% a year.
In contrast, $eP^*$ is going up less:
$P^*$ is roughly flat (1-2\% inflation in the US)
and $e$ is rising (if we compute its rate of change)
5\% in 2011 and 10\% in 2012.
Thus {\it RER\/} is rising, as Argentine goods get relatively more expensive.

\item Evidently people want dollars, not pesos, and the central bank supplies
them to maintain a relatively stable exchange rate.
One possible reason:  Argentine prices are rising,
and a substantial depreciation is one way to get that.
That makes pesos less attractive, since you'd lose (relative to dollars)
if the peso falls in value.

\item The debt dynamics equation is
\begin{eqnarray*}
   \Delta (B_t/Y_t)  &=&  (i_t - \pi_t)(B_{t-1}/Y_{t-1})
                - g_t (B_{t-1}/Y_{t-1}) + D_t/Y_t .
\end{eqnarray*}
The three terms are
\begin{eqnarray*}
    (i_t - \pi_t)(B_{t-1}/Y_{t-1}) &=&  -6.3  \\
    - g_t (B_{t-1}/Y_{t-1})   &=&  -2.3 \\
    D_t/Y_t  &=&  0.8 .
\end{eqnarray*}
Their total is --7.8, so the ratio of debt to GDP will fall to 37.0.
Note for later the negative contribution of the real interest rate:
they're getting a very good deal on their debt.
It's not hard to imagine that changing.

\item This is a call for the checklist:
\begin{itemize}
\item Debt and deficits.
(i)~The calculation shows the debt ratio is falling.
But the US court case could lead to default,
which isn't a good thing.
And the negative real interest rate is unlikely to continue.
If they paid a modest 2\% real rate on debt, the debt ratio would
go up about 4\% this year,
and higher rates are certainly possible.

\item Banks.
The EIU suggests that banks could suffer from a weak economy.

\item Exchange rates and reserves.
The real exchange rate continues to appreciate, making 
Argentine goods more expensive.  
At the same time, they're losing reserves as they sell dollars to support the peso.  
Both point toward a decline in the value of the peso.  

\item Politics.  Always an issue in Argentina.
There's some uncertainty given the president's lame duck status and health.
On the other hand, a change could make things better.
\end{itemize}
%
The fiscal situation, including the court case, the exchange rate and reserve position,
the banking system, and the political situation all shows signs of trouble.
Overall, they'll probably muddle through,
but there's a chance of serious trouble.
\end{enumerate}
Update:  Argentina defaulted in July 2014.
It's not clear how this will play out, but ``muddle through'' seems to be the likely outcome.
This hasn't had a large impact to date because Argentina
was already locked out of international financial markets for new issues.
The default doesn't change that, although it does make some international
transactions more difficult.



\end{enumerate}
\setlength{\leftmargini}{\oldleftmargini}


\section*{If you're looking for more}

You can find similar analyses in many places.
One of our favorites is the
Economist Intelligence Unit's Country Risk Reports.
Another is the IMF's
\href{http://www.imf.org/external/np/exr/facts/vul.htm}
{Vulnerability Indicators}.

There's no end of good descriptions of crises.
On the most recent crisis, Ben Bernanke's \index{Bernanke, Ben}
\href{http://www.federalreserve.gov/newsevents/testimony/Bernanke20100902a.htm}
{testimony} to the crisis commission is a good overview
from the perspective of the US
(search:  ``Bernanke\index{Bernanke, Ben}
 testimony crisis causes'').
Michael Lewis's Vanity Fair pieces are works of art
(search ``michael lewis vanity fair'').
Among the many books, we recommend
%
\begin{itemize}
\item Robert Bruner and Sean Carr,
\href{http://www.amazon.com/Panic-1907-Lessons-Learned-Markets/dp/0470452587/}
{\it The Panic of 1907\/}.
Good read, and short; you'll think it's about 2007.

\item Carmen Reinhart and Kenneth Rogoff,
\href{http://www.amazon.com/This-Time-Different-Centuries-ebook/dp/B004EYT932/}
{\it This Time is Different\/}.
The recent bestseller, covering 800 years and the whole world.

\item David Wessel,
\href{http://www.amazon.com/Fed-We-Trust-Bernankes-Great/dp/0307459683/}
{\it In Fed We Trust\/}.
Terrific book from the \emph{Wall Street Journal} writer.

%\item Paul Blustein,
%    {\it And the money rolled in (and out).\/}
%    Wonderful review of Argentina's 1999-2001 crisis.
%    He has another one, {\it The chastening\/}, about the Asian crisis
%    and the IMF.
%\item Andres Oppenheimer, {\it Bordering on chaos: Mexico's roller-coaster journey %toward prosperity.\/}
%    About Mexico in running up to the election and
%    subsequent crisis of 1994.
\end{itemize}
