\chapter{Taxes}\label{chp:tax}
\hypertarget{taxes}{}

%No extra line here.
\textbf{Tools:} Welfare analysis; triangles.

\textbf{Key Words:} Tax wedge; welfare loss; social cost; tax rate; tax base.

\textbf{Big Ideas:}
%\vspace{-0.1in}
\begin{itemize}
\item Tax systems should be (i) administratively simple and transparent and (ii) have a broad tax base.
\item A broad tax base minimizes the social cost of taxes.
\end{itemize}

\rule{\textwidth}{1pt}

Governments are a fact of life. They are the central player in building and enforcing
the institutional arrangements that make %economic and business
life as we know it possible.
They are also, in some cases, an obstacle to performance.
The difference between good and bad economic performance
is often the difference between good and bad government.

Our focus will be on the narrower issue of government revenues and expenses.
Governments around the world differ
in how much they spend (generally measured as ratios to GDP),
what they spend it on,
and how they finance their spending (taxes and borrowing).

This chapter is devoted to taxes.
Taxes are mind-numbingly complicated, but these three principles
describe good tax systems:
%
\begin{itemize}
\item \textbf{Tax \index{tax} revenue pays for spending.}
A government must collect enough tax revenue to pay for its expenses,
whatever they might be.

\item \textbf{Broad tax base. \index{tax!tax base}}
Most tax systems are riddled with exemptions.
The problem with exemptions is that they leave non-exempt activities to finance government spending.
With a narrower tax base, the rate must be higher on what's left.

\item \textbf{Administratively simple and transparent.}
The tax systems in some countries are so complex that people spend
days or weeks of their time, or hire professionals, to figure out
what they owe.
Worse, some countries assess taxes in ways that seem arbitrary,
leading to unpredictable tax expenses and endless disputes.
The best systems are simple (it's not hard to figure out what you owe)
and transparent (you know ahead of time the tax consequences of your actions).

\end{itemize}
%
We'll focus on the second principle, leaving the first for the next chapter
and the third to speak for itself.
But if you're interested in an example of a complex tax system
at work, search ``vodaphone taxes India.''


\section{Social cost of taxes}
\label{sec:triangle}

Taxes are a necessary evil;
governments, like people and businesses,
must finance their spending one way or another.
Governments generally do it with taxes.
However, the way in which governments collect tax revenue
can affect economic performance and welfare.
Our issue is not that taxes take purchasing power
away from individuals.
They do, but if government spending must be financed,
that's really a question of whether the purchasing power they take is put to good use.
We'll leave you to decide that for yourself.
Our issue is that taxes inevitably
discourage some activities relative to others.
%and these incentives may not be in the public interest.
Taxes on labor income may discourage work,
taxes on capital (or investment) income may discourage saving and investment,
and taxes on cigarettes may discourage smoking.
We'll leave cigarettes for another time,
but the general incentive effects of taxes are worth a closer look.

More formally,
taxes affect (``distort'') economic decisions.
They insert a ``wedge''  \index{tax!tax wedge} (a difference or discrepancy) between
private and social costs of various activities.
As a result, they generally lead to decisions that are socially inefficient:
We could reallocate the same resources and raise everyone's welfare.
The conditions for this invisible-hand result should be
familiar from your microeconomics class:
clear property rights,
competitive buyers and sellers (no monopolies),
complete information,
and absence of externalities
(no direct impact of one person's actions on another's welfare).
Under these conditions,
we might want to set tax rates
to generate the least disruption to resource allocation:
to minimize the adverse incentives built into taxes.

We can get a sense for how taxes affect decisions \index{tax!social cost}
in a traditional supply-and-demand setting like that
in Figure \ref{fig:tax}.
The demand curve (labeled $D$) represents purchasers of the product;
for any given quantity $Q$, it tells us how much buyers are willing to
pay --- hence the value to them (at the margin) of that number of units.
The supply curve (labeled $S$) represents sellers.
With competitive sellers, it tells us how much it costs to produce
a given quantity (at the margin).
The market clears at point A, where supply and demand are equal.

%%%%%%%%%%%%%%%%%%%%%%%%%%%%%%%%%%%%%%%%%%%%%%%%%%%%%%%%%%%%%%%%%%%%%%%%%%%%
%  Supply and demand diagram
\begin{figure}[h!]
\caption{The social cost of a tax.}
\label{fig:tax}
%
\centering
\setlength{\unitlength}{0.075em}
\begin{picture}(250,200)(0,-10)
%\footnotesize
\thicklines

% horizontal axis
\put(-30,0){\vector(1,0){300}}
\put(255,-16){$Q$}

% vertical axis
\put(0,-20){\vector(0,1){200}}
\put(-15,155){$P$}

% demand
\put(35,165){\line(4,-3){200}}\put(240,10){$D$}

% supply
\put(35,13){\line(5,3){200}} \put(240,130){$S$}
\put(35,33){\line(5,3){200}} \put(240,150){$S'$}
%\put(35,53){\line(5,3){200}} \put(240,170){$S''$}

% equilibrium labels
\put(157,77){\footnotesize A}
\put(130,98){\footnotesize B}
\put(138,64){\footnotesize C}
\put(4,97){\footnotesize E}
\put(4,60){\footnotesize F}
\put(-12,77){\footnotesize G}
%\put(133,92){\line(0,-1){92}}
% dotted line
\qbezier[31]{(133,0)(133,46)(133,92)}
\qbezier[45]{(0,92)(67,92)(133,92)}
\qbezier[45]{(0,72)(67,72)(133,72)}
\qbezier[50]{(0,80)(67,80)(145,80)}

\end{picture}
\begin{minipage}{0.7\textwidth}
\vspace{0.25in}
{\footnotesize The social cost of imposing a tax that shifts the supply
curve from $S$ to $S'$ is the triangle ABC.}
\end{minipage}
\end{figure}
%%%%%%%%%%%%%%%%%%%%%%%%%%%%%%%%%%%%%%%%%%%%%%%%%%%%%%%%%%%%%%%%%%%%%%%%%%%%


Now suppose we charge a tax of a fixed amount per unit.
From the perspective of buyers, the supply curve has shifted up
by the amount of the tax to $S'$.
Note that there is now a difference between the social cost
(the marginal cost of production in terms of resources used)
and the private cost (the price paid by buyers):  a wedge, in other words.
The market now clears at B for buyers and C for sellers.
This difference leads buyers and sellers to reduce the quantity
of resources allocated to this product,
leaving them to be used elsewhere in the economy.
Buyers, of course, buy fewer units, because the price has gone up.
Sellers offer fewer units for sale
because the price to them has fallen.
%The magnitude of the change in quantity
%depends on the slopes of the supply and demand curves.


The social cost of the tax (the reduction in welfare it causes)
is the area inside the triangle ABC.
The upper part of the triangle is the loss of consumer surplus \index{consumer surplus}
(the difference between what buyers pay and what the product is worth to them).
The lower part of the triangle is the loss of producer surplus \index{producer surplus}
(the difference between what sellers receive and the cost of production).
The sum is the social cost of the tax,
which economists refer to as the ``deadweight loss'' \index{tax!deadweight loss}
 or \index{tax!excess burden} ``excess burden.''
You may recall a similar argument against monopolies.
Both result in fewer resources devoted to the product
than we would like.

%\begin{comment}
If you're not familiar with this kind of analysis,
here's a more complete accounting
 of the welfare loss.
It's not essential to our story; feel free to skip to the next paragraph.
In the figure, the loss of consumer surplus \index{consumer surplus} is the area EBAG,
the cost to them of charging a higher price.
The loss of producer surplus is FCAG.
Adding them together gives us an area much larger than the triangle ABC.
The difference is the rectangle EBCF,
which is the amount of revenue collected by the government.
This revenue doesn't disappear, so it's not a welfare loss.
That leaves us with the triangle ABC as the welfare loss. \index{tax!welfare loss}
%\end{comment}

There's a fine point here about who pays the tax.
We could charge sellers or buyers with the same result.
Governments sometimes prefer taxes on firms to taxes on
people, in part because it makes the tax less visible to voters,
but the impact on resource allocation should be the same.


\section{The benefits of a broad tax base \index{tax!broad tax-base principle} \index{tax!tax base}
}


One objective of a good tax system is to minimize the social cost
of taxes:  to raise tax revenue with as little impact as possible
on resource allocation.
We sometimes say we're looking for a resource-neutral tax system ---
or as close to neutral as we can get.
This is a
complicated issue, both in theory and in practice,
but one principle is that we want a broad tax base.


%%%%%%%%%%%%%%%%%%%%%%%%%%%%%%%%%%%%%%%%%%%%%%%%%%%%%%%%%%%%%%%%%%%%%%%%%%%%
%  Supply and demand diagram
\begin{figure}[h!]
\caption{The social cost of doubling a tax.}
\label{fig:tax2}
%
\centering
\setlength{\unitlength}{0.075em}
\begin{picture}(250,210)(0,-10)
%\footnotesize
\thicklines

% horizontal axis
\put(-30,0){\vector(1,0){300}}
\put(255,-16){$Q$}

% vertical axis
\put(0,-20){\vector(0,1){200}}
\put(-15,155){$P$}

% demand
\put(35,165){\line(4,-3){200}}\put(240,10){$D$}

% supply
\put(35,13){\line(5,3){200}} \put(240,130){$S$}
\put(35,33){\line(5,3){200}} \put(240,150){$S'$}
\put(35,53){\line(5,3){200}} \put(240,170){$S''$}

% equilibrium labels
\put(157,77){\footnotesize A}
\put(130,98){\footnotesize B}
\put(138,64){\footnotesize C}
\put(115,108){\footnotesize D}
\put(108,47){\footnotesize E}

% dotted lines
\qbezier[25]{(133,0)(133,46)(133,92)}
\qbezier[35]{(0,92)(67,92)(133,92)}
\qbezier[35]{(0,72)(67,72)(133,72)}
%
\qbezier[25]{(118,0)(118,46)(118,103)}
\qbezier[30]{(0,102)(67,102)(118,102)}
\qbezier[30]{(0,62)(67,62)(118,62)}

\end{picture}
\begin{minipage}{0.7\textwidth}
\vspace{0.25in}
{\footnotesize We double the tax rate by
shifting the supply curve from $S$ to $S''$.
Note the social cost:  the triangle ADE is four times as big as ABC.}
\end{minipage}
\end{figure}
%%%%%%%%%%%%%%%%%%%%%%%%%%%%%%%%%%%%%%%%%%%%%%%%%%%%%%%%%%%%%%%%%%%%%%%%%%%%


The argument for a broad tax base goes like this. \index{tax!tax base}
Think about two ways of raising the same tax revenue:
a low tax rate on a broad base and a higher rate on a narrower base.
Which is better?
We'll give an answer using our supply and demand analysis.
Suppose we have two similar markets, each like the
one we described in Figure \ref{fig:tax}.
In the broad-base system,
we tax the products in both markets at the same rate.
The social cost is, therefore, double what we saw earlier:
the triangle ABC for each market.

Now consider a narrow-base system that taxes one market at twice the rate.
We'll use Figure \ref{fig:tax2} to see how this works.
There, we have drawn three supply curves:
$S$ refers to supply without the tax;
$S'$ refers to supply with a small tax (the broad-base system);
and $S''$ refers to supply with a tax rate double that in $S'$
(the narrow-base system).
What is the social cost of the narrow base?
Since the rate is higher,
the welfare triangle is larger;
it consists of the area ADE.
If you look at this long enough, you'll realize that
the area of ADE is four times that of ABC,
which makes the social cost twice as large
as in the broad-base system.

The point, in general, is that broad-based tax systems are better
because they allow you to raise a given amount of revenue with
a lower rate and (therefore) smaller social cost.
You'll hear lots of arguments for tax exemptions,
but you rarely hear that they result in higher taxes on other things,
which is the primary argument against them.

A corollary of this principle is that with similar goods ---
by which we mean goods with similar supply and demand curves --- we should aim for similar tax rates.
The reasoning is the same,
although it's harder to show in a diagram.
%(high and low being less obvious than double and zero).
If we have goods whose demand curves have different slopes,
similar logic would lead us to tax them differently,
but that's a subtle point we'd prefer to leave for another time.

%One common application is tax rates over time:
%since (say) consumption today is similar to consumption
%next year, it should have a similar tax rate.
%Thus large changes in tax rates over time
%violate the principle --- and, more importantly,
%lead to larger welfare losses than necessary.


\section{Applications}

Here are some practical applications
of the broad-tax-base/tax-similar-goods-the-same principle.

\textbf{The underground economy. \index{underground economy}}
One of the difficulties of an underground economy is that
unofficial businesses typically do not pay taxes,
thereby forcing all of the tax burden onto the rest
of the economy.
That violates our principle (low rate on broad base) and also
the corollary (tax similar goods at similar rates).
We've shown that this leads to an inefficient allocation of resources.


William Lewis ({\it The Power of Productivity\/}) argues that in
Brazil, it may also lower productivity.
His argument has several steps.
Brazil has a relatively large government
for a country at its stage of development (40 percent of GDP, which
is above the US
and significantly above what we see in most developing countries).
Financing government spending requires, therefore,
relatively high tax rates,
which creates a substantial incentive for tax avoidance.
Small firms (the story goes) are generally less productive than large firms
(economies of scale),
but many survive because they are able to avoid taxes.
Thus the tax system protects inefficient small firms,
thereby lowering overall productivity.
In Lewis's story, this is a direct result of large government.
%In our context, the point is that taxing two sectors at different
%rates is socially inefficient.


\textbf{Value-added taxes (VAT).\index{tax!value-added tax (VAT)}}
Before the VAT became popular, countries often had piecemeal
tax systems in which goods were taxed at every stage
of production.
This led, in some cases,
to very high taxes on intermediate and final products
simply because the taxes at each stage added up.
This violates our principle, specifically the corollary,
because a product made by a single vertically-integrated firm is taxed
at a lower rate than the same product made by several firms,
one at each stage.

Consider a product that has five stages of production,
each performed by a different firm.
If each stage is charged a moderate tax of ten percent,
what is the total tax paid in the production
of the product?
Let's say that total value added is five, with one
unit of value added at each stage.
The first-stage firm produces one unit of value.
This costs the second-stage firm 1.10
since it must pay the ten percent tax.
This firm also adds one unit of value,
and sells its output for a price (including taxes) of
\[
    \left( 1.10 + 1 \right) \times 1.10 \;\;=\;\; 2.31,
\]
so the implicit tax rate over the two stages is
$ 0.31/2 = 15.5$ percent.
If you work through all five stages,
you'll find that the price of the final product,
including all the taxes paid,
is 6.71 after the last stage, so the effective tax rate is 34.2 percent
[$= (6.71-5.00)/5.00$].
Note the large difference
in tax rates across the five stages of production.
The final stage only gets taxed once, so it pays a tax rate of 10\%,
but the first stage gets taxed five times, so
it's taxed at a rate of 61 percent [$1.10^5 = 1.61$]!
In contrast, a vertically-integrated firm pays only ten percent, at each stage and overall.

These differences in tax rates potentially lead to
inefficient production,
as firms look for substitutes for highly-taxed inputs,
or integrate vertically.
This is one of the arguments for a value-added tax system.
With a VAT, firms pay tax on only the value-added of their
stage of production, which eliminates differences in tax rates
paid by different stages.
A value-added tax system is equivalent to one
in which we tax only the final good;
we just arrange to collect the tax in pieces.


\textbf{Taxes on capital income. \index{tax!capital income tax}}
A high tax rate on capital income might be expected
to discourage saving and investment,
leading the economy to have less capital than otherwise.
This, in turn, would reduce wages, since
the marginal product of labor is lower if we have less capital.
So, what is an appropriate tax rate on capital income?
Some economists argue that taxes on capital income should be zero.
People would eventually pay tax on capital income indirectly
when they consume the proceeds,
but they should not be taxed before then.
The logic is similar to the argument for a value-added tax,
since taxes on capital income are effectively taxes on future
consumption and accumulate in a similar way.


Let's think about how households allocate their income over time.
Suppose that we have two dates (labeled ``0'' and ``1'').
If a household earns labor income $(Y_0,Y_1)$ at the two dates
and receives
a (real) interest rate $r$ on saving,
then saving is $ Y_0 - C_0$, and
consumption at date 1 must be
$ C_1 = (1+r) (Y_0 - C_0) + Y_1 $.
We can put the two together in the present-value relation:
\begin{eqnarray*}
    C_0 + C_1/(1+r) &=&  Y_0 + Y_1/(1+r) .
\end{eqnarray*}
This tells us, in essence, that the price of date-1 consumption
is $1/(1+r)$.
If we had more periods, we'd have a similar relation,
with prices $1/(1+r)$, $1/(1+r)^2$, $1/(1+r)^3$, etc.,
for consumption at dates 1, 2, 3, etc.\index{interest rate}


Now think about taxes.  If we tax interest income, this changes
the price of future consumption.
For a given real interest rate $r$,\index{interest rate!real}
a higher tax rate increases the price of future consumption,
which you might expect to encourage current consumption.
If the tax rate on capital income is $\tau$,
then the after-tax interest rate is $ (1-\tau) r$
and the price of consumption $n$ periods in the future
is $ 1/[1+(1-\tau)r]^n $.
This may not seem like a big deal,
but with the mythical power of compound interest,
it can increase the price of future consumption substantially.
Consider a numerical example with $r = 0.04$ (four percent a year)
and a tax rate of $\tau = 0.25$ (25 percent).
With no tax, the price of consumption one period in the future
is 0.9615 [$=1/(1+r)$].
With the tax, this increases to 0.9709 [$=1/[1+(1-\tau)r]$],
a modest difference.
But if the number of periods is large, the difference can also be large.
Suppose $n=25$; think of a 30-year-old consultant saving for
retirement.
Then, the tax raises the price of future consumption by 27 percent,
from 0.3751 to and 0.4776.
You can imagine that this could lead people to consume more now
and less later since future consumption has become relatively
more expensive.
It might also lead them to work less if working now
is intended to finance future consumption.


If people consume more now and less later, then
they are saving less.
And if they are saving less,
the economy will have less capital.
The cost has the same source as in our earlier analysis: The private benefits of saving are less than the social benefits,
so we do too little of it.
That's why some economists favor a consumption tax:  a tax on only
that part of income that is consumed.
In practice, many countries offer something of this sort
through tax-sheltered retirement and saving programs,
which avoid the period-by-period tax on investment income
of our example.

If a government promises not to tax capital too heavily in the future,
will investors believe it?
This version of the \hyperref[sec:time_cons]{time-consistency problem}\index{time consistency} can be severe. If investors doubt the commitment, they
will refrain from investment even if current tax rates are low. The issue arises
whenever governments are seen as willing to exhaust their borrowing capacity.
As we have seen in other examples, overcoming the challenge of time
consistency depends on the nature and quality of a society's
institutions. In the late 17th century, for example,
the empowerment of the British parliament helped persuade a
rising commercial class that the King would not arbitrarily
seize their wealth. The flow of savings and investment helped Britain
grow earlier and faster than other modern economies.
In many highly indebted countries today,
concerns about future tax burdens can reduce saving and investment now,
or encourage other means of tax avoidance,
possibly making the government's current budget situation worse.

\begin{comment}
{\it Taxes on labor income.\/}
In our diagram, a tax on labor (collected from either firms or workers)
would discourage work in the same way it discourages other activities
(see Figure \ref{fig:tax}).
The issue here is how sensitive labor supply is to the tax rate.
If people work the same amount whether the tax rate is high or low,
then the social cost is zero.
If they work less when the tax is higher, then the cost is positive.
Generally the flatter the supply curve the higher the cost.
You can show this in the figure, but it takes good draftsmanship.

How sensitive is labor supply?  Somewhat.
The most sensitive components of labor supply
seem to be spouses in two-income households
(does one stay home, or take early retirement?)
and poor people (many of whom would lose
means-tested benefits if they worked).
\end{comment}


\textbf{Changing tax rates. \index{tax!tax rate}}
Economist Edward Prescott writes
({\it Wall Street Journal\/}, December 20, 2005):
%
\begin{quote}
Let's drop the word ``cuts'' [when we talk about taxes].  The problem with advocating a cut in something is that you are necessarily going to stir up political trouble from someone who will want to increase it again.  So, even if you are fortunate enough to get your cut enacted, it is likely a matter of time before the political pendulum swings back and someone else gets their increase. \end{quote}
%
The argument against large changes in tax rates over time
follows from the corollary:  ``Tax similar markets at similar rates.''
In this case, the two markets are ``today'' and ``tomorrow.''
We could add the cost of the uncertainty created
by the process of changing tax rates.

\textbf{Deficits.} The same argument gives us some insight into deficits. We should finance whatever the government spends
with relatively stable tax rates.
Why?  Because low taxes now and high taxes later, or the reverse,
violates our principle.
Suppose, then, that the government is running a deficit.
Should it raise taxes?
It should aim at a stable level of tax rates
that finances government spending.
Typically, you would expect this to lead to deficits in recessions,
when the tax base is small, and surpluses in booms.
In practice, this is more complicated because we don't know either the level
of spending (what's the present value of future commitments to
Social Security and Medicare?)
or the base on which tax rates will be applied
(will the economy grow three or four percent a year over the next decade).
The principle remains:
finance government spending with stable tax rates.\index{government budget!budget deficit@budget (or government) deficit}

\begin{comment}
{\it Tax arbitrage. \index{tax! tax arbitrage}}
Differences in tax rates lead to obvious incentives to relabel
a high-tax item as a low-tax item.
In the UK, taxes are more favorable to capital expenditures
on equipment than structures
(equipment is expensed, structures are amortized),
so there's an incentive for firms to interpret equipment broadly.
In Ireland, the corporate tax rate is very low,
so multinational firms have an incentive to shift profits from
(say) the US to Ireland.
(How might they do this?)
In the US, corporate taxes place a cap on the deductability of
executive salaries, but not on other forms of compensation (stock options,
for example), which creates an incentive for firms to pay executives
through the latter.
\end{comment}


\section*{Executive summary}

%\setlength{\leftmargini}{.5\oldleftmargini}
\begin{enumerate}

\item All taxes have incentive effects.  In the absence of externalities and monopolies, the tax
systems that lead to the most efficient allocations of resources
(a)~apply low tax rates to a broad base
and (b)~tax similar products at similar rates.

\item The cost of exemptions is that non-exempt products must pay higher rates as a result.

%\item Taxes on labor and capital income probably discourage work and
%saving/investment, respectively.
\end{enumerate}
%\setlength{\leftmargini}{\oldleftmargini}


\section*{Review questions}

%\setlength{\leftmargini}{.5\oldleftmargini}
\begin{enumerate}

%\begin{comment}
\item  Welfare triangle review.
In Figure \ref{fig:tax}, identify the following:
\begin{enumerate}
\item The loss of consumer surplus. \index{consumer surplus}  Why does the tax leave consumers
with less surplus?
\item The loss of producer surplus.  Why does the tax leave producers
with less surplus?
\item Government revenue.
\item The total welfare loss.
\end{enumerate}

Answer.
\begin{enumerate}
\item The area EBAG.
consumer surplus before the tax is the area between the demand curve
and the line GA indicating the market price.
Some consumers are willing to pay more; the difference is their surplus.
When the price paid by consumers rises as a result of the tax, some of this surplus goes away.
\item The area FCAG.
Producer surplus before the tax is the area between the supply curve
(the cost of production) and the line GA indicating the market price.
When the price received by producers falls as a result of the tax, some of this surplus goes away.
\item The area EBCF.  this is the tax (EF) times the equilibrium quantity (FC).
\item The area ABC.  This is consumer surplus plus producer surplus
minus government revenue.
\end{enumerate}
See also the discussion at the end of Section \ref{sec:triangle}.
%\end{comment}

\item Tax systems.
Comment on the welfare impact of these aspects of the US tax system:
\begin{enumerate}
\item Sales tax exemption for food and clothing.
\item Sales tax exemption for goods purchased over the internet.
\item Sales tax exemption for medical care.
\item Income tax exemption for health insurance.
\item Sales tax exemption for education supplied by nonprofit institutions.
\item Elimination of the capital gains tax.
\end{enumerate}

Answer.
\begin{enumerate}
\item Probably bad because it means tax rates on other things must be higher,
which generates larger welfare losses.
Remember:  broad base, low rates.
One common justification is that it favors poor people,
since food and clothing are necessities,
but it's probably not an effective way to do this.
The best way is simply to give them money.

\item Also bad, and for the same reason.  It leads to such
things as sales tax on internet purchases from Barnes \& Noble (since
they have local outlets) but not on Amazon (since they do not).

\item Ditto.

\item Ditto.

\item Remember the principle:  tax similar products the same way.
There's no economic logic for taxing a product
differently just because its producer it has a different legal structure.
Remember: the NYSE was a nonprofit until recently.

\item  To the extent that it's a tax on capital or investment income,
this could be a good thing.
Further, capital gains reflect inflation as well as investment income,
which can result in potentially very high tax rates on real returns.
The solution here, though, is to index the tax system
(or keep inflation\index{inflation} low enough that it doesn't have much effect).
An important caveat is that there are no adverse incentive effects
involved in taxing capital gains that have already occurred;
the incentive argument works only going forward.
\end{enumerate}

\item Taxes without spending?
Suppose a hypothetical government has no expenditures to finance.
What tax rates should it set?

Answer.  Zero!  Why?  Nonzero taxes (even negative taxes or subsidies)
generate adverse incentives; the prices people pay for products
do not reflect their social cost of production.
Possible exception:
externalities, although even here there may be better choices than taxes.

\item  Small government.  Since government spending must be financed
with taxes, and taxes distort the allocation of resources, should we have a small government?

Answer.  This is a complex issue, but here's one take on it.  First,
you need a government. There are clearly important and necessary
roles for government: providing national and personal security,
defining and enforcing property rights, supporting competitive
markets, and so on. Without an effective government, you simply
can't have a productive economic system. Second, there's tremendous
variety across countries in the kinds of services provided by
government.
In many countries, governments supply educational
services, social insurance, and pensions, although the degree of
government involvement varies. The evidence is mixed.  Among
countries with high GDP per person, those with large governments are
not notably less productive than those with small governments.
Sweden, for example, is a productive and prosperous country despite very high
government spending. Among developing countries, the evidence is
stronger: Those with smaller ratios of spending to GDP have grown faster,
on average, over the last forty years. This may reflect the direct
effects of government or other factors --- it's hard to say.

\item Progressive taxes.  Questions often come up about the progressivity
of tax systems.
They aren't really review questions, but this seems as good a place to put them
as any.
\begin{enumerate}
\item How does a progressive tax square with taxing similar things at similar rates?
\item Since rich people own most of the assets, shouldn't we tax investment income
as a way to redistribute income?
\item If we rely heavily on VAT, as many countries do,
 how do we make the overall tax system progressive?
\end{enumerate}

\needspace{4\baselineskip}
Answer.
\begin{enumerate}
\item It doesn't.
For reasons we've seen, progressive taxes distort resource allocation more than a flat tax that collects
the same amount of revenue.
But they also redistribute income from rich to poor.
If you want the latter, you're stuck with some of the former.
\item Maybe.  It still has adverse incentive effects, but it's true that
wealth is much more unequally distributed than income.
\item A VAT is, by design, a flat tax:  products and people are treated the same way.
In practice there's some variation in rates by products, but it's harder to
treat buyers differently.
Most countries introduce progressivity through the income tax and means-tested benefits.
\end{enumerate}
\end{enumerate}
%\setlength{\leftmargini}{\oldleftmargini}


\section*{If you're looking for more}

The analysis of welfare triangles is standard economics.
See, for example, these links from Wikipedia:

\vspace*{\parskip}
\centerline{\url{http://en.wikipedia.org/wiki/Economic_surplus}}
\centerline{\url{http://en.wikipedia.org/wiki/Deadweight_loss}}

There's also a Kahn Academy video (``Taxation and dead weight loss'').

Real-world tax systems can be incredibly complicated.
Some good overviews are:
%
\begin{itemize}
\item The OECD's program on taxation  has an extensive
set of data and analysis for developed countries:

\vspace*{\parskip}
\centerline{\url{http://www.oecd.org/tax/}.}

\item The World Bank's Doing Business website includes information about
tax rates and associated administrative costs for mid-sized firms.
Be careful, however, of the definitions.
Total tax, for example, is reported as a percentage of
profit, even though some of the taxes
apply to labor.


\item The Economist Intelligence Unit's Country Commerce
and Country Finance reports contain information about
both business and individual taxes.
Here's what they say about corporate taxes in the US:
``Tax jurisdiction in the United States is divided among the federal
government, the 50 states plus the District of Columbia,
and local counties and municipalities. ...
There are no uniform rules on the definition of taxable income or on the
apportionment of income among the various tax jurisdictions.
Hence, the advice of a tax lawyer is practically indispensable
to any newcomer to multistate business.''

\item
 Myron Scholes, Mark A. Wolfson, Merle Erickson, Edward Maydew, and
 Terrence Shevlin's {\it Taxes and Business Strategy (4e)\/}
 is a wonderful, practical book on tax issues.
 (Earlier editions are cheaper, and probably as good for our purposes.)

\end{itemize}

\section*{Symbols used in this chapter}

\begin{table}[H]
\centering
\caption{Symbol table.}
\begin{tabular*}{0.6\textwidth}{l@{\extracolsep{\fill}}l}
\toprule
Symbol & Definition\\
\midrule
$P$     &Price\\
$Q$        &Quantity\\
$S$        &Supply\\
$D$        &Demand\\
$Y$        &Labor income\\
$C$        &Consumption\\
$r$        &Real interest rate\\
%$S$        &Saving ($=Y-C$)\\
$\tau$     &Tax rate (proportional)\\
\bottomrule
\end{tabular*}
\end{table}


%
%\pagebreak
%\subsubsection*{Appendix:  Taxes and employment (optional)}
%
%We go through a more formal approach to taxes on labor income.
%The slope of the labor supply curve shows up here (implicitly)
%in the discussion of income and substitution effects.
%If they're equal, then labor supply is vertical.
%One of the results:  it's not only taxes that matter,
%but how the money is spent.
%For example, income support programs,
%which give you income even if you are not working,
%generate income effects and therefore have an impact on labor supply.
%That's another dimension of policy that's relevant to
%individual work decisions.
%
%The question:  Is it possible that taxes on labor income reduce work?
%Suppose people like to consume goods and services,
%and they also enjoy leisure,
%defined as time spent not working.
%If they spend a fraction $L$ of their time working, then $1-L$ is spent not working.
%Their work decision is then a tradeoff of the two:
%they need to work to be able to afford consumption (a good thing),
%but work reduces their leisure (a bad thing).
%Their work decision then balances these costs and benefits,
%with the price of work (the after-tax wage) as the equilibrating force.
%We'll describe this tradeoff formally with a utility function,
%a mathematical representation of preferences.
%The idea is that combinations of consumption and leisure
%that generate high utility
%are better than those that generate low utility.
%A convenient example is
%\[
%    U(C,1-L) \;=\; \log C + \gamma \log (1-L) .
%\]
%This says that consumption and leisure are both good things
%($U$ increases as we increase $C$ and $1-L$).
%The parameter $\gamma$ represents the importance of leisure;
%high values represent strong preference for leisure,
%so we might expect them to lead to less work (other things equal).
%
%
%Let's see how the decision to work depends on the wage.
%If the wage is $w$, then a person who works amount $L$ would
%receive income $wL$.
%In our setting, this would all be consumed,
%since that's the point of working:
%$ C = wL$.
%Faced with this wage, our theoretical person would choose the work effort
%that maximizes her utility:
%\[
%    \log C + \gamma \log (1-L) \;=\; \log (wL) + \gamma \log (1-L).
%\]
%We can find the solution by setting the derivative with respect to $L$
%equal to zero:
%\begin{equation}
%    1/L  - \gamma/(1-L) \;=\; 0
%    \label{eq:labor-foc}
%\end{equation}
%or $ L = 1/(1+\gamma)$.
%
%
%You might notice two features of this solution.
%One is that the amount of work declines with $\gamma$:
%if people have a strong preference for leisure (large $\gamma$),
%they will work less.
%In principle, we could use this to explain lower employment rates
%in France than the US:  the French have higher $\gamma$s.
%In practice, most economists would be skeptical.
%We know, for example, that the French worked
%about as much as Americans in 1965.
%Did their $\gamma$ rise over the last thirty years?
%If so, why?
%You can see that this explanation raises as many questions as it answers.
%The second feature of the solution is that work does not depend on
%the wage.
%Does this make any sense?
%In fact, wages have gone up several times over the last century,
%with little impact on the amount of work effort.
%We have seen some changes ---
%women work more, children and men less ---
%but the total fraction of time spent working hasn't changed much,
%suggesting that this isn't a bad approximation.
%Theoretically, it represents a balance of income and substitution effects.
%With a higher wage, leisure is more expensive, so we consume less of it
%(and work more).
%But a higher wage also makes us richer,
%and part of this increase in our standard of living
%is used to consume more leisure (work less).
%In our example, the two effects are equal:
%they cancel each other out,
%leaving no impact of the wage on work.
%This feature is not demanded by theory,
%but it seems to us to be a reasonably good approximation
%of the world around us.
%
%
%Now consider the impact of taxes.
%If the tax rate on labor income is $\tau$,
%then the after-tax wage is $(1-\tau) w$.
%Changing the tax rate will have no effect on time spent working
%for the same reason that changing the wage had no effect.
%Were we wrong in our guess that taxes might reduce work effort?
%Maybe not.
%One route toward this outcome is to consider what the government does
%with its tax revenue.
%The government collects tax revenue from workers equal to $\tau w$.
%Suppose all or part is given back to people in the form
%of transfer payments:  unemployment insurance, parenthood benefits, welfare,
%and so on.
%In this case, the income effect of the transfer may lead people to work less.
%The details work like this.
%A typical person receives transfer payments $T$;
%let us say that they do not depend
%on anything a person might do.
%The transfer payment allows her to consume $ C = (1-\tau) w L + T$.
%We maximize
%\[
%    \log C + \gamma \log (1-L) = \log [(1-\tau) wL + T] + \gamma \log (1-L),
%\]
%which leads to the first-order condition,
%\[
%    (1-\tau) w/[(1-\tau) wL + T]  \;=\; \gamma/(1-L) .
%    \label{eq:labor-foc-trans}
%\]
%Let us say that transfer payments are a fraction $f$ of tax revenue:
%$ T = f \tau w L$.
%Then a little algebra (well, maybe more than a little) tells us
%$ L = 1/(1+\gamma^*) $, where
%\[
%    \gamma^* \;=\; \gamma \; \frac{1-\tau(1-f)}{1-\tau} \;\geq\; \gamma .
%\]
%In words:  an increase in the tax rate acts like an increase
%in preference for leisure and reduces work.
%If all the tax revenue is used to finance transfers ($f=1$),
%then $ \gamma^* =  \gamma/(1-\tau)$ and the result is clear.
%
%{\it Numerical example.\/}
%Let $w = 10$ and $\gamma = 4$.
%With no transfers,
%work effort satisfies equation (\ref{eq:labor-foc}), so
%\[
%    L \;=\; 1/(1+\gamma) \;=\; 0.2.
%\]
%Now let $\tau = 0.25$ and suppose that half of all
%tax revenue goes to transfers ($ T = \tau w L/2 $).
%Then $ L = 0.18$, a fall of 10\%.
%The impact is larger with higher taxes and larger fractions
%of tax revenue devoted to transfers.
%The point is that the quantitative impact
%of the tax can be large.
%
%
%\begin{comment}
%The bottom line:  taxes on labor income can reduce
%time spent working if the tax revenue is returned
%to households in the form of transfer payments.
%There's some debate about this, but it seems to us
%to be what we see in some countries.
%There are lots of other features of tax systems
%that may influence work decisions in practice,
%including benefits tied explicitly to not working.
%Some experts argue that loss of benefits
%makes marginal tax rates on labor income
%extremely high in some countries, especially for low-income workers.
%
%Problem.
%
%\item Is a tax on consumption similar to a tax on labor income?
%(This applies to the optional appendix.)
%
%Answer.  Yes and no.
%Why no?  Because a tax on consumption avoids the disincentive to saving
%built into a tax on income.
%Why yes?  Because most labor income is consumed,
%so the two taxes have similar effects.
%In the static setting of our labor income tax analysis
%(where there's no saving, so we don't need to worry about that),
%the two are equivalent.
%Suppose we charge tax rates of $\tau_L$ on labor income and $\tau_C$ on consumption.
%Then the budget constraint becomes
%\[
%    (1+\tau_C) C \;=\; (1-\tau_L) w L .
%\]
%We can rewrite this as
%\[
%     C \;=\; \left( \frac{1-\tau_L} {1+\tau_C } \right) w L
%        \;=\; (1-\tau^*) wL
%\]
%with
%\[
%    \tau^* \;=\; \frac{\tau_C + \tau_L}{1+\tau_C} .
%\]
%In words:  this is equivalent to a system with a tax $\tau^*$
%on labor income only.
%In practice, this is sometimes
%used to compute a single summary measure $\tau^*$
%of the tax burden.
%
%Example.  Suppose the two tax rates are $ \tau_C = \tau_L = 0.25$.
%Then they're equivalent to a labor tax rate of 40\% [$=0.50/1.25$]
%(not 50\%!).
%
%\item The appendix is based on a talk by Richard Rogerson,
%reprinted as ``Understanding differences in hours worked,''
%{\it Review of Economic Dynamics\/}, 2006.
%The theory is more technical than this class,
%but the graphs of work over time in Europe are really striking.
%Generally there's been a sharp drop in work in Europe over the last 40 years, but not in the US.
%The question is why.
%He argues that taxes are part of the answer.
%
%
%
%\end{comment}
%

