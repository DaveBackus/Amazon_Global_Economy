%No extra line here.
\textbf{Tools:} Open market operations; central bank balance sheet; Taylor rule.

\textbf{Key Words:} Real interest rate; nominal interest rate; expected inflation; real money balances; inflation targeting; interest-rate rules; rules vs. discretion; zero lower bound; quantitative easing; credit easing; policy duration commitment.

\textbf{Big Ideas:}
%\vspace{-0.1in}
\begin{itemize}
    \item In conventional practice, central banks use open market operations to manage interest rates. These policy actions are equivalent to managing the money supply directly.
    \item The Taylor rule provides a guide to how central banks manage their target interest rates in response to data on inflation and (real) GDP growth.
    \item When interest rates are at or near zero, a central bank can resort to unconventional monetary policy, including quantitative easing, credit easing, and policy-duration commitments. These policies have been in widespread use since 2008.
\end{itemize}
